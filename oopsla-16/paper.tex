%-----------------------------------------------------------------------------
%
%               Template for sigplanconf LaTeX Class
%
% Name:         sigplanconf-template.tex
%
% Purpose:      A template for sigplanconf.cls, which is a LaTeX 2e class
%               file for SIGPLAN conference proceedings.
%
% Guide:        Refer to "Author's Guide to the ACM SIGPLAN Class,"
%               sigplanconf-guide.pdf
%
% Author:       Paul C. Anagnostopoulos
%               Windfall Software
%               978 371-2316
%               paul@windfall.com
%
% Created:      15 February 2005
%
%-----------------------------------------------------------------------------


\documentclass[10pt,times,numbers]{sigplanconf}
\usepackage[1stsubmission]{oopsla2016}

% The following \documentclass options may be useful:

% preprint      Remove this option only once the paper is in final form.
% 10pt          To set in 10-point type instead of 9-point.
% 11pt          To set in 11-point type instead of 9-point.
% numbers       To obtain numeric citation style instead of author/year.

% template-provided usepackages
\usepackage{amsmath}
% custom ones copied from pldi template
\usepackage{url}
\usepackage{enumitem}
\usepackage[linesnumbered,ruled]{algorithm2e}
\usepackage{algpseudocode}
\usepackage{graphicx}
\usepackage{amsthm}
\usepackage[colorlinks=true,allcolors=blue,breaklinks,draft=false]{hyperref}
\newcommand{\doi}[1]{doi:~\href{http://dx.doi.org/#1}{\Hurl{#1}}}

\newtheorem{lemma}{Lemma}
\newtheorem{theorem}{Theorem}
\newtheorem{definition}{Definition}

% ???
%\newcommand{\cL}{{\cal L}}

\begin{document}

% TODO CAMREADY: Don't forget to run e.g. s/\\quicksand /\\quicksand~/
\newcommand\landslide{\textsc{Landslide}}
\newcommand\quicksand{\textsc{Quicksand}}
\newcommand\simics{\textsc{Simics}}
\newcommand{\sect}[1]{\S #1} 
\newcommand\hilight[2]{\color{#1}#2\color{black}}
\definecolor{orange}{RGB}{192,96,0}
\definecolor{olivegreen}{RGB}{0,127,0}
\definecolor{brickred}{RGB}{192,0,0}
\definecolor{commentblue}{RGB}{0,0,192}

\newcommand\numthrlibs{79}
\newcommand\numpintoses{78} % 'di' is basecode. durr
\newcommand\numstudence{157} % total pintoses plus p2s

\special{papersize=8.5in,11in}
\setlength{\pdfpageheight}{\paperheight}
\setlength{\pdfpagewidth}{\paperwidth}

\conferenceinfo{OOPSLA '16}{November 2--4, 2016, Amsterdam, the Netherlands}
\copyrightyear{2016}
\copyrightdata{978-1-nnnn-nnnn-n/yy/mm}
\copyrightdoi{nnnnnnn.nnnnnnn}

% Uncomment the publication rights you want to use.
%\publicationrights{transferred}
\publicationrights{licensed}     % this is the default
%\publicationrights{author-pays}

%\titlebanner{banner above paper title}        % These are ignored unless
%\preprintfooter{short description of paper}   % 'preprint' option specified.

\title{Stateless Model Checking with Data-Race Preemption Points}

\authorinfo{Ben Blum}{Carnegie Mellon University}{bblum@cs.cmu.edu}
\authorinfo{Garth Gibson}{Carnegie Mellon University}{garth@cs.cmu.edu}
%\authorinfo{\vspace{24pt}}{}{}

\maketitle

\begin{abstract}
Stateless model checking is a powerful technique for testing concurrent programs,
but suffers from exponential state space explosion when the test input parameters are too large.
Several reduction techniques can mitigate this explosion,
but even after pruning equivalent interleavings, the state space size
%for a fixed set of preemption points
is often intractable.
Most prior tools are limited to preempting only on synchronization APIs,
which reduces the space further, but can miss unsynchronized thread communication bugs.
%
Data race detection, another concurrency testing approach, focuses on suspicious memory access pairs during a single test execution.
It avoids concerns of state space size, but is prone to false positives:
spurious reports of access pairs that cannot be reordered to produce a bug.

We present an {\em Iterative Deepening} framework for stateless model checking,
which manages the exploration of many state spaces using different preemption points.
It uses state space estimation to prioritize jobs most likely to complete in a fixed CPU budget,
and it incorporates data-race analysis to add new preemption points on the fly.
Preempting threads during a data race's instructions
%These preemption points allow us to check whether each data-race can produce an observable failure, and
can automatically classify the race as buggy or benign,
and uncovers new bugs not reachable by prior model checkers.
%
It also enables full verification of all possible schedules when every data race is verified as benign within the CPU budget.
In our evaluation, Iterative Deepening
%with data-race preemption points
found 1.8x as many bugs and verified 4.3x as many tests compared to stateless model checking without data-race preemption points.
%Our evaluation shows this technique is
%more effective than single-state-space model checking
%%both at finding more bugs and at completing more state spaces when no bug exists.
%both at finding the same bugs faster and at finding new bugs entirely.
\end{abstract}

\category{D.2.4}{Software Engineering}{Software/Program Verification}
\category{D.2.5}{Software Engineering}{Testing and Debugging}
%
%% general terms are not compulsory anymore,
%% you may leave them out
%\terms
%term1, term2
%
%\keywords
%keyword1, keyword2

%%%%%%%%%%%%%%%%%%%%%%%%%%%%%%%%%%%%%%%%%%%%%%%%%%%%%%%%%%%%%%%%%%%%%%%%%%%%%%%%

\section{Introduction}

% blah blah trite opening sentence
%As parallelism becomes ever more important for achieving high performance in modern-day programs,
%so too do advanced concurrency testing techniques become important for verifying the correctness of those programs.
Concurrency bugs are notoriously hard to find and reproduce because they only appear in specific thread interleavings, which arise at random during normal program execution.
{\em Stateless model checking} \cite{verisoft} offers a method for finding such bugs,
or verifying their absence,
%by systematically executing a program along as many distinct interleavings as possible,
by forcing a program to execute each distinct interleaving,
capturing
%and controlling
this nondeterminism in a finite state space.
Unfortunately, these state spaces explode exponentially in the size of the input program.
%Reduction
Techniques such as Dynamic Partial Order Reduction \cite{dpor} and Maximal Causality Reduction \cite{mcr} expand the limits of feasible test completion,
and search ordering strategies such as Iterative Context Bounding \cite{chess-icb} \revision{encourage}~bugs to be found sooner in a given space should they exist.

However, all stateless model checkers to date are bound by a fixed set of {\em preemption points}: code locations that define the granularity at which threads interleave.
For example, \textsc{CHESS} \cite{chess} \revisionx{by default} preempts only on synchronization operations and library calls, which can miss lock-free shared memory races.
%
On the other hand, SPIN \cite{spin} %and Inspect \cite{inspect}
%are able to preempt
preempts threads before any shared memory access.
Such fine granularity would automatically check each data race for the possibility of failure, but risks timing out before the state space can be completed. %makes full state space completion intractable for even modestly-sized tests.
\revisionx{Some tools, such as CHESS and Inspect \cite{inspect} can strike a middle ground using compiler instrumentation to statically add preemption points on memory accesses.
Nevertheless,}
choosing preemption points is a tradeoff between schedule coverage and feasibility of completion:
%but
even with state-of-the-art reduction techniques,
\revision{fixing the degree of coverage}~in advance necessarily leaves some tests unaffordably large \cite{parrot,mcr}.
%This work shows how to avoid making that tradeoff decision in advance.

We present \quicksand,
a model checking framework for deciding at runtime which preemption points to test,
according to which resulting state spaces are most likely to fit a prescribed CPU budget.
It uses data-race analysis \cite{eraser} to dynamically find new preemption points which expose bugs not reachable by preempting on API calls alone.
When prior \revision{approaches}~would time out on large tests by trying several preemption points simultaneously,
\quicksand~\revision{identifies this pitfall in advance using state space size estimation \cite{estimation},
and instead}~tests smaller state spaces based on subsets of those preemption points.
\revision{Often, testing these}~smaller state spaces can even find the same bugs sooner.

%\revision{Hence, recent approaches favor
%heuristic
%%search ordering
%strategies \cite{chess-icb,randomized-scheduler}
%to prioritize fast bug-finding over ultimate completion time,
%%assuming state spaces will always be too big and
%%at the expense of ultimate completion time,
%which in turn sacrifices the potential for a full safety guarantee when no bugs exist.}
%%search ordering strategies such as Iterative Context Bounding \cite{chess-icb}
%%assume all state spaces will be too big,
%%and optimize the search to uncover bugs more quickly at the expense of ultimate completion time.}
%%/ by sacrificing the ability to complete a total verification quickly when possible.

On the other hand, when the CPU budget is large enough to fully test all data-race preemption points,
we prove that this constitutes a total verification of all possible thread schedules.
%Short of deciding in advance to preempt on every single instruction \cite{spin}, this was not previously possible without data-race preemption points.
\revision{To achieve the same level of verification, prior model checkers must decide}~in advance to preempt on every single memory access \cite{spin}, \revision{which is computationally prohibitive for
even moderately-sized
%larger
tests.
Our approach provides the best of both worlds:
by estimating the size of the test on-the-fly, \quicksand~can find bugs quickly in large tests {\em and} provide fast total verification for small ones.}


We evaluate \quicksand~by testing \numstudence~student thread libraries and kernels from the undergraduate operating systems classes at Carnegie Mellon University, University of California at Berkeley, and University of Chicago.
%We show many advantages over both conventional stateless model checking and single-pass data-race analysis.
%We find many of the same bugs found by the conventional approach faster using subsets of preemption points,
We find that data-race preemption points quickly expose many new bugs that prior model checkers could not find at all,
and that they enable full verification of many more tests than before.
%and that they enable full verification of many racy, but non-failing, tests.

Our contributions are as follows:
\begin{enumerate}
	\item {\em Iterative Deepening}, a new \revision{algorithm}~for combining data-race analysis with stateless model checking, and \quicksand, an open-source implementation;
	\item A proof of convergence \revisionx{for sequentially-consistent memory models}, showing that should it be possible in the given CPU budget,
		fully testing every discovered data-race preemption point is equivalent to testing all possible thread schedules;
	\item A new tactic for eliminating one class of false-positive data race candidates,
		which cannot soundly be used in a single-pass analysis,
		but which we prove correct when used with Iterative Deepening;
	\item A large evaluation in which \quicksand~compares favourably to stand-alone data-race detection and stateless model checking approaches, finding new bugs that would be missed by either alone.
\end{enumerate}

% Comment this out for space?
The rest of the paper is organized as follows.
\sect{\ref{sec:overview}} reviews the background material, %on stateless model checking and data-race analysis,
\sect{\ref{sec:design}} and \sect{\ref{sec:implementation}} discuss our design and implementation, %of Iterative Deepening,
\sect{\ref{sec:soundness}} provides our proofs of soundness, %convergence and of the soundness of our new false-positive data-race tactic,
\sect{\ref{sec:eval}} presents our evaluation,
\sect{\ref{sec:future}} discusses limitations and future work,
\sect{\ref{sec:related}} surveys the related work,
and \sect{\ref{sec:conclusion}} concludes.

\section{Background}
\label{sec:overview}

We review the background in stateless model checking and data-race analysis using the example program in Figure~\ref{fig:example}.

\subsection{Stateless Model Checking}
\label{sec:overview-sssmc}

Stateless model checking \cite{verisoft} is a testing technique for systematically exploring the possible thread interleavings of a concurrent program.
A stateless model checker executes the program repeatedly, each time according to a new thread interleaving, until the state space (or the CPU budget) is exhausted.
During each execution, it forces threads to execute serially, thereby confining the program's nondeterminism to controlled thread switches.

Rather than identifying suspicious conditions which may include false alarms, the approach of many static analyses \cite{racerx,coverity},
stateless model checkers focus on concrete observed failures such as assertions, deadlocks, segfaults, and use-after-free.
For example, data races \revision{do not always lead to failures}, but represent suspicious violations of common locking discipline.
Although C++ defines all data races as undefined behaviour \cite{boehm-memorymodels},
this work focuses on reporting races to the user only when accompanied by direct evidence of a failure.

\begin{figure}[t]
	\small
\begin{tabular}{rll}
	& \multicolumn{2}{c}{\texttt{int x = 0; mutex\_t mx;}} \\
	& {\bf Thread 1} & {\bf Thread 2} \\
	1 & \texttt{\hilight{orange}{mutex\_lock(\&mx);}} & \\
	2 & \texttt{int tmp = x;} &\\
	3 & & \texttt{atomic\_xadd(\&x, 1);} \\
	4 & & \texttt{\hilight{olivegreen}{yield();}} \\
	5 & & \texttt{atomic\_xadd(\&x, 1);} \\
	6 & \texttt{x = tmp + 1;} & \\
	7 & \texttt{\hilight{commentblue}{mutex\_unlock(\&mx);}} & \\
	8 & & \texttt{assert(x >= 2);} \\
	%1 & \texttt{\hilight{brickred}{x->foo = ...;}} & \\
	%2 & \texttt{\hilight{olivegreen}{free}(x);} \\
	%3 & & \texttt{\hilight{commentblue}{// x's memory recycled}} \\
	%4 & & \texttt{y~=~\hilight{olivegreen}{malloc}(sizeof *y);} \\
	%5 & & \texttt{\hilight{commentblue}{// ...initialize...}}\\
	%6 & & \texttt{publish(y);} \\
	%7 & & \texttt{\hilight{brickred}{y->bar = ...;}} \\
\end{tabular}
\caption{Example program with a data-race bug. In this interleaving, the assertion on line 8 will fail. Two data-race preemptions are required to expose the bug: one just before thread 1's line 6, and one just before thread 2's line 8.}
\label{fig:example}
\end{figure}

%\begin{figure}[t]
%	\begin{tabular}{c}
%	\includegraphics[width=0.42\textwidth]{tree-maximal-only.pdf}
%	\end{tabular}
%	\caption{\revisionz{The state space of possible interleavings is represented as a tree,
%	which stateless model checkers test exhaustively.
%	This state space arises when preempting on synchronization APIs used by the program in Figure~\ref{fig:example};
%	other state spaces are possible with more, or fewer, preemption points.}}
%	\label{fig:tree-maximal}
%\end{figure}

\begin{figure}[t]
	\includegraphics[width=0.48\textwidth]{trees.pdf}
	\caption{Iterative Deepening example.
		The minimal state space (a) includes only voluntary thread switches, such as {\tt yield()}. %or {\tt cond\_wait()}.
		Multiple further tests can be run: preempting on calls to {\tt mutex\_lock} alone (b), {\tt mutex\_unlock} alone (c), or both together (d).
Each option increases the state space size unpredictably, so multiple state spaces should be tested in parallel.
Estimation techniques~\cite{estimation} inform which state spaces to prioritize.
}
	\label{fig:id}
\end{figure}

The checker defines the granularity of thread interleavings by the {\em preemption points} it uses to switch threads.
Most model checkers \cite{chess,dbug-ssv} choose synchronization and thread library API boundaries for these points;
in our example program, these would be lines 1, 4, and 7.
Figure~\ref{fig:id} shows several possible resulting state spaces.
The approach of prior work is to enable all preemption points simultaneously, i.e., to test only the state space denoted (d).

To mitigate the exponential explosion,
Dynamic Partial Order Reduction \cite{dpor} identifies equivalent execution sequences according to Mazurkiewicz trace theory \cite{mazurkiewicz},
and tests at least one execution from each equivalence class.
Intuitively, if two thread transitions between preemption points do not conflict on any shared resource access, reordering them produces an equivalent interleaving, i.e., the same program behaviour.
Iterative Context Bounding \cite{chess-icb}, another popular technique, heuristically reorders the search to prioritize interleavings with fewer preemptions first, according to the insight that most bugs require few preemptions to uncover.
Nevertheless, state spaces are still exponentially-sized in the number of conflicting transitions.

This motivates {\em Iterative Deepening}, our new technique for heuristically adjusting the preemption points at runtime.
Rather than committing to one state space with every available preemption point enabled,
we will search among different {\em subsets} of the points.
%for example, ``preempt on all calls to {\tt mutex\_lock} but not on {\tt mutex\_unlock}''.
Hence, we will test all the state spaces shown in Figure~\ref{fig:id} in parallel,
and decide on-the-fly whether to pursue each test, or to defer it in favour of others.

\revisionx{
{\bf Recent advances.}
Like many prior checkers \cite{chess,dbug-ssv,inspect,portend}, ours implements DPOR for sequentially-consistent hardware.
In the worst case, these tools may all suffer false negatives as they miss weak-memory-only bugs.
Recently, Zhang et al. \cite{tsopso} introduced a new formalism with which DPOR can control weak memory nondeterminism,
such as reordering store buffers.
Iterative Deepening could use this new technique,
%We expect this technique would apply directly with Iterative Deepening,
provided a model checker which implements such reorderings (not yet available to us).
%We believe this assumption is reasonable for the scope of this paper because the main property we require of DPOR is that
%A recent paper extended DPOR to support weaker memory models \cite{tsopso}, so our proofs could likewise be extended with that version of DPOR in future work.

Maximal Causality Reduction (MCR) \cite{mcr}, a reduction algorithm which may replace DPOR,
has also recently shown promising performance improvements over prior model checkers.
We expect Iterative Deepening to be compatible with MCR,
and look forward to evaluating the combination when an MCR implementation becomes widely available.
}

\subsection{Data Race Detection}
\label{sec:overview-dr}

\begin{figure}[t]
	\small
\begin{tabular}{c}
	\revision{
\begin{tabular}{rll}
	& \multicolumn{2}{c}{\texttt{int x = 0; bool y = false; mutex\_t mx;}} \\
	& {\bf Thread 1} & {\bf Thread 2} \\
	1 & \texttt{\hilight{brickred}{x++;}~// A1} & \\
	2 & \texttt{mutex\_lock(\&mx);} & \\
	3 & \texttt{mutex\_unlock(\&mx);} & \\
	4 & & \texttt{mutex\_lock(\&mx);} \\
	5 & & \texttt{mutex\_unlock(\&mx);} \\
	6 & & \texttt{\hilight{brickred}{x++;}~// A2} \\
\end{tabular}
}
\\
\revision{\normalsize (a) True potential data race.}
\\
\\
\revision{
\begin{tabular}{rll}
	%& \multicolumn{2}{c}{\texttt{int x = 0; bool y = false; mutex\_t mx;}} \\
	%& {\bf Thread 1} & {\bf Thread 2} \\
	1 & \texttt{\hilight{brickred}{x++;}~// B1} & \\
	2 & \texttt{mutex\_lock(\&mx);} & \\
	3 & \texttt{y = true;} & \\
	4 & \texttt{mutex\_unlock(\&mx);} & \\
	5 & & \texttt{mutex\_lock(\&mx);} \\
	6 & & \texttt{bool tmp = y;} \\
	7 & & \texttt{mutex\_unlock(\&mx);} \\
	8 & & \texttt{if (tmp) \hilight{brickred}{x++;}~// B2} \\
	%8 & & \texttt{if (tmp)} \\
	%9 & & \texttt{~~~~\hilight{brickred}{x++;}~// B2} \\
\end{tabular}
}
\\
\revision{\normalsize (b) No data race in any interleaving.}
\end{tabular}
\caption{\revision{Data-race analyses may be prone to either {\em false negatives} or {\em false positives}.
Applying HB to program (a) will miss the potential race possible between A1/A2 in an alternate interleaving,
while using Limited HB on (b) will produce a false alarm on B1/B2.}}
\label{fig:hb-example}
\end{figure}

Data race analysis \cite{eraser} identifies pairs of unsynchronized memory accesses between threads.
Two instructions are said to race if
they both access the same memory address,
at least one is a write,
the threads do not hold the same lock,
and no synchronization enforces an order on the thread transitions \revision{(the {\em Happens-Before} relation)}.
In Figure~\ref{fig:example}, lines 3 and 5 each race with 2 and 6, and line 6 races with 8.

\revision{
A data race analysis may be either {\em static} (inspecting source code) \cite{racerx} or {\em dynamic} (tracking individual accesses arising at run-time) \cite{tsan}.
This paper focuses exclusively on dynamic analysis,
so although our example refers to numbered source lines for ease of explanation,
in practice we are actually classifying the individual memory access events corresponding to those lines during execution.
}

%\revision{Among these, only the races involving lines 3 and 5 are relevant to the possible assertion failure;
%permuting the order of line 8 alone
%% relative to 2 and 6
%will not change the program's behaviour.}

Though state-of-the-art model checkers preempt only on synchronization events,
many serious concurrency bugs are caused by data races leading to corrupted shared state.
Figure~\ref{fig:example}'s buggy interleaving is possible only with {\em data-race preemption points}:
preempting just before an instruction identified as part of a data race.
None of the state spaces in Figure~\ref{fig:id} contain this interleaving,
as none of the mutex/yield preemptions split lines 2 and 6 across different transitions.

	{\bf Variants of Happens-Before.}
	Most prior work focuses on {\em Happens-Before} (HB) \cite{lamport-clocks,djit,fasttrack} as the order relation between accesses.
\cite{predictive-dr} and \cite{hybriddatarace} identify a problem with this approach:
it cannot identify access pairs separated by an unrelated lock operation which could race in an alternate interleaving.
%as shown in the example program in Figure~\ref{fig:hb-example}(a).
\revisionx{Figure~\ref{fig:hb-example}(a) shows a contrived example program in which HB masks the potential race.}
We call such unreported access pairs {\em false negatives}.
However, consider the similar program in Figure~\ref{fig:hb-example}(b),
in which the access pair ceases to exist in the alternate interleaving.
O'Callahan et al. \cite{hybriddatarace} introduced the {\em Limited HB} relation,
which will report such potential races
by considering only blocking operations like {\tt cond\_wait} to enforce the order.
Limited HB will report all potential races, avoiding \revisionx{many} false negatives \cite{tsan},
but at the cost of necessarily reporting some such {\em false positives}.

Finally, the {\em Causally-Precedes} relation \cite{predictive-dr} %strikes a middle ground,
extends HB to additionally report a subset of potential races while soundly avoiding false positives.
It tracks conflicting accesses in intervening
critical sections to determine whether lock events are unrelated to a potential race.
Causally-Precedes will identify the potential race in Figure~\ref{fig:hb-example}(a), as the two critical sections do not conflict,
although it can still miss true potential races in other cases.

\revisionx{Being dynamic analyses, both HB and Limited HB may suffer false negatives when a racy access pair is not executed at all in a specific interleaving.
Limited HB offers the advantage of identifying a potential race as long as the access pair is observed under any interleaving,
rather than requiring the accesses to be adjacent in time, as HB would.}
While stand-alone data-race analyses must avoid inundating the user with false alarms \cite{racerx},
%However,
we incorporate data-race analysis in an internal feedback loop,
using model checking to automatically test each potential race
and report only directly observed failures to the user.
%Our convergence theorem requires identifying all potential data-races to provide total verification (\sect{\ref{sec:totalverif}})
%Hence, we employ Limited HB in this paper},
%accepting some overhead from false positives for the sake of total verification (\sect{\ref{sec:totalverif}}).
\revisionx{Hence, we can accept some overhead from Limited HB's false positives for the sake of finding data-race candidates more quickly.
In \sect{\ref{sec:eval}} we will evaluate how HB and Limited HB each influence \quicksand's bug-finding and verification speed.
}

% Old explanation.
%Prior work \cite{hybriddatarace} distinguishes between {\em pure happens-before} \cite{lamport-clocks},
%in which accesses with a lock release/acquire in between are not considered concurrent,
%and {\em limited happens-before},
%in which only blocking operations like {\tt cond\_wait} or {\tt join} enforce the order.
%Compared to the former, the latter may report false positives, but avoids false negatives \cite{tsan},
%in which a potential racing access pair is not reported, even though it could lead to a bug.
%We use the limited happens-before analysis, accepting some overhead from false positives for the sake of total verification (\sect{\ref{sec:totalverif}}).

\revisionx{
{\bf Philosophy of bugs.}
% two camps: all races are bugs; citations:
% and, races must be classified; citations: replay analysis, portend,
While there is a vast body of work on how to detect data races to begin with,
judging data races once found is a matter of philosophical debate unto itself.
Some recent tools classify races depending on how they impact program behaviour \cite{recordreplaydrs,portend}, overlooking {\em benign} races in search of more dangerous ones.
%In particular,
\cite{racerx} acknowledges that a program may have too many races for a user to worry about,
so bug reports must be prioritized by severity of effects. %how likely they may cause failures or security vulnerabilities.
%
However, other prior work argues that data races are always bugs \cite{miscompile-benign,data-races-are-evil},
largely due to the possibility of compiler or hardware reordering of racy accesses.
%
We take the former camp:
we consider a data race a bug only when it results in a visible failure state (e.g., crash or deadlock)\footnote{
C++ declares any race between two non-atomic locations to be undefined behaviour \cite{cpp-foundation}.
From a C++ perspective, we assume all concurrent accesses are implemented by {\tt std::atomic} loads and stores.}.
We bypass concerns of compiler reordering by checking programs at the executable level;
for a study of hardware reordering in the context of DPOR, we refer the reader to \cite{tsopso}.
}

\subsection{Terminology}

For the rest of the paper, we will abbreviate {\em preemption point} (PP),
\revision{{\em happens-before} (HB)},
{\em model checking} (MC),
{\em single-state-space model checking} (SSS-MC), % (i.e., the approach of prior work),
{\em Dynamic Partial Order Reduction} (DPOR),
{\em Iterative Context Bounding} (ICB),
and {\em state space estimate} (ETA).

SSS-MC indicates the approach of prior tools:
the set of PPs is fixed in advance, and the tool commits to testing every interleaving available with those PPs.
Many techniques can skip equivalent interleavings or order the search to uncover bugs faster \cite{dpor,demeter,chess-icb,gambit,smc-empirical-study},
but new PPs cannot be added, nor ineffective ones removed, by any dynamic analysis.

We distinguish between data-race {\em candidates} and data-race {\em bugs}.
\revision{We refer to racing (or potentially-racing) memory access pairs as}~{\em data race candidates}.
Should %a future interleaving,
preempting during such accesses
lead to \revision{an observable}~failure, then we report a {\em data-race bug}.
Otherwise, if the access pair can be reordered,
%\revision{to be simultaneous},
but does not produce a failure under any interleaving, it is a {\em benign data race} (with respect to the test input).
If they cannot be reordered at all, due to some other communication \revision{such as in Figure~\ref{fig:hb-example}(b)}, it is a {\em false positive}.

We also identify the {\em minimal} and {\em maximal state space} for each test.
The {\em minimal state space} includes only thread switches arising from no preemptions (Figure~\ref{fig:id}(a)).
The {\em maximal state space} is the one tested by SSS-MC: all statically-available PPs are enabled (Figure~\ref{fig:id}(d)).
%However, should new data-race PPs be added during a test, the new maximal state space will be the one including those as well.
%% don't say this ^ -- because when you add dr pps, they add in pairs and there would be multiple maximals, at least until they each get explored and subseuqnetly add a pp of the other of the pair.

\section{Design}
\label{sec:design}

Named after the analogous technique in chess artificial intelligence \cite{iterative-deepening-chess-ai},
Iterative Deepening
%likewise
makes progressively deeper searches of the state space until the CPU budget is exhausted.
In this context, the depth is the number of PPs used.
Hence, our tool \quicksand~schedules multiple MC instances in parallel to simultaneously test many different subsets of the available PPs
, which we refer to as {\em jobs}.
It prioritizes jobs based on number of PPs, ETA, and whether they include data-race PPs.
We rely on state-space estimation \cite{estimation}
to understand which jobs are likely to complete on time,
before actually testing each interleaving within.
The overall purpose is to decide automatically when to defer testing a state space,
so an inexpert user can provide only their total CPU budget as a test parameter,
and to enable completing appropriately-sized jobs within that budget.

Note that Iterative Deepening is a {\em wrapper} algorithm around stateless MC.
A MC tool is still used to test each state space, and other reduction techniques are still applicable.
Moreover, because Iterative Deepening treats the set of preemption points as mutable,
it can add new ones reactively based on any runtime analysis.
We focus on dynamic data-race detection~\cite{tsan} as the mechanism for finding new preemption candidates.

%%%%%%%%%%%%%%%%%%%%%%%%%%%%%%%%%%%%%%%%%%%%%%%%%%%%%%%%%%%%%%%%%%%%%%%%%%%%%%%%

\subsection{Initial PP configuration}

Iterative Deepening must be seeded with some initial state spaces,
which can be any number of subsets of the statically-available PPs that SSS-MC would use.
For testing user-space code, we begin with the four PP sets from Figure~\ref{fig:id}:
$\{yield\}$,
$\{yield,lock\}$,
$\{yield,unlock\}$,
and $\{yield,lock,unlock\}$,
By extension, these also introduce PPs on any other sync primitives implemented with internal locks,
such as condvars or semaphores.
Preempting on voluntary switches such as {\tt yield} is always necessary to maintain the invariant that only one thread runs between consecutive PPs.
%so the {\tt yield} PP is always implicitly enabled.

For kernel-level testing, we consider interrupt-disabling analogous to locking,
so we also preempt just before a disable-interrupt opcode ({\tt cli}) and just after interrupts are re-enabled ({\tt sti})\footnote{
%(to appropriate the names of the x86 instructions)
During data-race detection, {\tt cli}/{\tt sti} are treated as a single global lock.
%as {\tt cli}'d memory accesses can still race with others that have interrupts on.
Some kernels disable preemption without disabling interrupts,
which can be modelled the same way using manual annotations. %of that API.
This also assumes uni-processor scheduling; for SMP kernels, replace {\tt cli}/{\tt sti} with spinlocks.}.
\quicksand~is configured to begin with
$\{yield\}$,
$\{yield,lock\}$,
$\{yield,unlock\}$,
$\{yield,cli\}$,
$\{yield,sti\}$,
and $\{yield,lock,$ $unlock,cli,sti\}$.
As a heuristic, we don't test every intermediate subset such as $\{lock,sti\}$,
which could potentially be improved in future work (\sect{\ref{sec:future}}).

%%%%%%%%%%%%%%%%%%%%%%%%%%%%%%%%%%%%%%%%%%%%%%%%%%%%%%%%%%%%%%%%%%%%%%%%%%%%%%%%

\subsection{Choosing the best job}

% TODO: Move the 'seed sets' discussion up here, from implementation.

\newcommand\PendingJobs{\ensuremath{\mathcal{P}}}
\newcommand\SuspendedJobs{\ensuremath{\mathcal{S}}}
\newcommand\GetETA[1]{ETA(#1)}
\newcommand\GetPPSet[1]{PPSet(#1)}
With a limited CPU budget, we must avoid running tests that are likely to be fruitless.
Hence, we separate the available PP sets into a set of {\em suspended} jobs (partially-explored state spaces with high ETAs),
and a set of {\em pending} jobs (untested ones with unknown ETAs).
When a MC instance reports an ETA too high,
we compare other pending jobs to find another one more likely to complete in time.
%
Our method for doing so, listed in Algorithm~\ref{alg:shouldworkblock}, is the heart of Iterative Deepening.
%\footnote{
%Though its worst-case performance is $O(mn)$ in the
%%number of pending and suspended jobs,
%sizes of $\mathcal{P}$ and $\mathcal{S}$,
%in practice the non-constant portion beyond line 4 runs very infrequently
%and is negligible compared to the exponentially-sized state spaces.}.
Its main feature is understanding that if \GetPPSet{$j_1$} $\subset$ \GetPPSet{$j_2$},
and $j_1$ is suspended,
then $j_2$'s state space is guaranteed to be strictly larger, so $j_2$ will take at least as long.
Hence we should avoid testing $j_2$ unless $j_1$'s ETA improves over time.
%reveals that it might finish in time after all.
Similarly, when some job finds a bug, we cancel all pending superset jobs, as they would find the same bug.

\begin{algorithm}[t]
	\SetKwInOut{Input}{Input}
	%\textbf{Function} GetBestJob($j_0$, PendingJobs, SuspendedJobs): \\
	\Input{$j_0$, the currently-running job}
	%\Input{$eta$, $j_0$'s predicted completion time}
	\Input{\PendingJobs, the list of pending jobs, sorted decreasingly by heuristic priority}
	\Input{\SuspendedJobs, the list of already-suspended jobs, sorted increasingly by ETA}
	\If{\GetETA{$j_0$} $<$ HeuristicETAFactor $\times$ TimeLeft()}{
		// Common case: Job is expected to complete. \\
		return $j_0$
	}
	\ForEach{job $j_P \in$ \PendingJobs}{
		// Don't run a pending job if a subset of it is already suspended; its ETA would be at least as bad. \\
		\If {$\forall j_S \in$ \SuspendedJobs, \GetPPSet{$j_S$} $\not\subset$ \GetPPSet{$j_P$}}{
			return $j_P$
		}
	}
	%// no pending jobs; maybe resume a suspended job \\
	\ForEach{job $j_S \in$ \SuspendedJobs}{
		\If{\GetPPSet{$j_0$} $\not\subset$ \GetPPSet{$j_S$}
			$\land$
			\GetETA{$j_0$} $>$ \GetETA{$j_S$}}{
			// If a subset of $j_S$ is also suspended, don't run the larger one first. \\
			\If{$\forall j_{S2} \in$ \SuspendedJobs, \GetPPSet{$j_{S2}$} $\not\subset$ \GetPPSet{$j_S$}}{
				return $j_S$
			}
		}
	}
	return $j_0$ // ETA was bad, but no other job was better.
	\caption{Suspending exploration of a state space in favour of a potentially smaller one.}
	\label{alg:shouldworkblock}
\end{algorithm}

%
We also account for the inherent inaccuracy of ETA estimates.
Line 1 heuristically scales up the time remaining to avoid suspending jobs too aggressively
in case their ETAs are actually overestimated.
Lines 12-14 account for the
%bizarre
possibility that among two suspended jobs,
%given two jobs,
%%$j_1,j_2$,
\GetPPSet{$j_1$} $\subset$ \GetPPSet{$j_2$}
but
\GetETA{$j_1$} $>$ \GetETA{$j_2$}.
This can arise because estimates tend to get more accurate over time,
and $j_1$ perhaps ran much longer before suspending.
% In such scenarios,
We heuristically assume the smaller job's ETA is more accurate
to avoid repeatedly resuming larger jobs briefly while their ETAs only become worse
(it lets us avoid thrashing in \quicksand).

%%%%%%%%%%%%%%%%%%%%%%%%%%%%%%%%%%%%%%%%%%%%%%%%%%%%%%%%%%%%%%%%%%%%%%%%%%%%%%%%

\subsection{Data-race preemption points}
\label{sec:classifying}

During stateless MC, runtime data-race detection may find data-race candidates that we wish to investigate further.
Because data races indicate access pairs that can interleave at instruction granularity,
it is logical to re-execute the test and issue preemptions during those instructions to test alternate thread interleavings~\cite{racefuzzer,portend}.

\newcommand\AllJobs{\ensuremath{\mathcal{J}}}
\begin{algorithm}[t]
	\SetKwInOut{Input}{Input}
	\Input{$j_0$, the currently-running job}
	\Input{\AllJobs, the set of all existing jobs}
	\Input{$\alpha$, an instruction reported by the MC as part of a racing access pair}
	\If{$\forall j \in \AllJobs,$
	\GetPPSet{$j_0$} $\cup$ $\alpha$
	$\not\subseteq$
	\GetPPSet{$j$}
	}{
		AddNewJob(\GetPPSet{$j_0$} $\cup$ $\alpha$, HeuristicPriority($\alpha$))
	}
	\caption{Adding new jobs with data-race PPs.}
	\label{alg:handledatarace}
\end{algorithm}

With Iterative Deepening, this is a simple matter of creating a new state space with a PP enabled on the racing instructions by each thread, as shown in Algorithm~\ref{alg:handledatarace}.
Note that even though a data race may involve two different instructions, $\alpha$ and $\beta$, we add new state spaces with only one new PP at a time.
Rather than adding a single large state space, %configured to preempt on both involved instructions,
i.e., $AB =$ \GetPPSet{$j_0$} $\cup$ $\alpha$ $\cup$ $\beta$,
we prefer to add multiple smaller jobs which have a higher chance of completing in time, i.e.,
$A =$ \GetPPSet{$j_0$} $\cup$ $\alpha$ and
$B =$ \GetPPSet{$j_0$} $\cup$ $\beta$.
If $A$ and $B$ are bug-free, they will in turn add $AB$ later.
%We take care to avoid duplicating any superset state spaces with PPs on multiple data races.
The condition on line 1 ensures that we avoid duplicating any state spaces with multiple data-race PPs;
for example, $AB$ is reachable by multiple paths through its different subsets, but should be added only once.

Furthermore, we do not always strictly increase the number of PPs when we find a new data race.
%When \quicksand~receives a data race report, it adds two new jobs to its workqueue:
For each instruction involved in a data race, \quicksand~adds two new jobs:
a ``small'' job to preempt on that instruction only,
and a ``big'' job to preempt on that instruction as well as each PP used by the reporting job.
%
Hence,
%together with the logic in \sect{\ref{sec:classifying}},
each {\em pair} of racing accesses will spawn four new jobs, as shown in Figure~\ref{fig:new-dr-jobs}.
%
The rationale of spawning multiple jobs is that which will be more fruitful cannot be known in advance:
while the big job risks not completing in time,
the small job risks missing the data race entirely if the original PPs were required to expose it.
% TODO CAMREADY: Put numbers here.
In practice, we observed some bugs found quickly by these small jobs, and other bugs missed by the small jobs found eventually by the big jobs,
which motivates the need for Iterative Deepening to prioritize the jobs at runtime.

\begin{figure}[t]
	\includegraphics[width=0.48\textwidth]{dr-jobs-v2.pdf}
	\caption{\quicksand~manages the exploration of multiple state spaces, communicating with each MC instance to receive ETAs, data race candidates, and bug reports.
		When an access pair is reported as a data race, we generate a new PP for each access and add new jobs corresponding to different combinations of those with the existing PPs.}
	\label{fig:new-dr-jobs}
\end{figure}

The new spaces may expose a failure, in which case we report a data-race bug,
or complete successfully, which indicates a benign or false-positive data race.
They may also uncover a new data-race candidate entirely, %in some alternate interleaving,
in which case we iteratively advance to a superset state space containing PPs for both racing access pairs.
%Because Iterative Deepening is
Being constrained by a CPU budget,
we may time out before completing a data race's associated state space,
in which case it remains a potential false positive that the user must handle.

%When \landslide~detects a data race, it reports each of the two memory accesses involved in the race.
%Each report indicates the program counter value (PC) associated with the access, as well as some further conditions to help filter away unrelated executions of the same instruction on different data.
%(For example, many parts of a codebase might call {\tt list\_insert()}, but only one callsite does so without adequate locking.)
%Ideally, the PC would be qualified by a full backtrace, but tracing the stack is too expensive to do for each shared memory access.
%Instead, \landslide~qualifies the PC with
%(a) the current thread ID and
%% FIXME: We don't actually do this.
%(b) the most recent {\tt call} instruction.
%% (a crude approximation of a stack trace)
%% which are much cheaper, as we carry them around all the time already
%Note that we do {\em not} qualify data races by the shared memory address,
%which can change based on different interleavings of previous code
%(for example, depending on the result of {\tt malloc()}).
%% especially when malloc is involved.
%%Figure~\ref{fig:dont-filter-dr-by-address} shows example code where qualifying by memory address will miss the bug.

%%%%%%%%%%%%%%%%%%%%%%%%%%%%%%%%%%%%%%%%%%%%%%%%%%%%%%%%%%%%%%%%%%%%%%%%%%%%%%%%

\subsection{Heuristics}
% List of all heuristix:
% HOMESTRETCH - last 60sec of test, don't suspend
% ETA_THRESH - "to let its ETA stabilize"
% eta factor
% shold_reproduce -- small dr jobs are not allowed to add further instances of themself (why not? don't remember)
% priority change between suspected and confirmed dr
Algorithm~\ref{alg:shouldworkblock} allows heuristically scaling a job's ETA when comparing to the time budget,
to express how optimistic we are about the estimate's accuracy.
We use 2 as this scaling factor based on the results in \cite{estimation}.
%though we allow changing it via the command line.
We also include a heuristic to
%ignore ETAs entirely
never suspend jobs before they pass a certain threshold of interleavings tested,
for which we choose 32,
so that their ETAs have some time to stabilize.

We classify data-race candidates as {\em single-order} or {\em both-order} \cite{portend}
based on whether the MC observed the racing instructions ordered one or both ways in the original state space,
Jobs with both-order data-race PPs are prioritized higher,
because single-order candidates are more likely to be false-positives
(though before preempting during the access itself, we cannot say for sure, hence the heuristic).
%so we prioritize jobs with both-order data-race PPs.
For single-order races, we do not add a PP for the later access at all;
should it be needed, preempting on the first access will suffice to upgrade the race to both-order.


\section{Implementation}
\label{sec:implementation}

\subsection{Landslide}
\label{sec:landslide}

We chose \landslide~\cite{landslide} as our stateless model checker due to its ability to trace program execution at the granularity of individual instructions and memory accesses, which dynamic data-race detection requires.
\landslide~implements DPOR \cite{dpor},
%\cite{dpor},
state space estimation \cite{estimation}, and a hybrid lockset/happens-before data-race analysis \cite{hybriddatarace}.
% TODO: \cite{fairstatelessmc} instead or in addition?
It avoids state space cycles (e.g. ad-hoc synchronization with {\tt yield()} or even {\tt xchg} loops) with a heuristic similar to Fair-Bounded Search \cite{bpor}.
% this line can be cut if space is needed
It can test both user-level and kernel-level code, although is limited to timer-driven nondeterminism.
% joshua wants "segfault" to be "memory access error (i.e., segmentation fault, or bus error)"
Its bug-detection metrics include assertion failure, deadlock, segfault, heap checking (like Valgrind~\cite{valgrind}), and a heuristic infinite loop/livelock check.

{\bf Restricting PPs with stack trace predicates.}
Most model checkers (MCs) preempt indiscriminately on any sync API call, regardless of the call-site.
However, when testing a particular module in a large codebase,
the user is likely uninterested in PPs arising from other modules.
\landslide~provides the {\tt within\_function} configuration command for a user to identify which call-sites matter most.
Before inserting a PP, \landslide~requires at least one argument to {\tt within\_function} to appear in the current thread's stack trace.
%The {\tt without\_function} directive works similarly, but as a blacklist.
The {\tt without\_function} directive is the dual of {\tt within\_function}, indicating a blacklist.
Multiple invocations can be used; later ones take precedence.
%\cite{landslide} provides further detail on this feature.

{\bf Data races in lock implementations.}
Data race tools in prior work \cite{tsan,portend} recognize the implementations of sync primitives to avoid spuriously flagging memory accesses resulting from the lock implementation itself.
The assumption that the locks are already correct enables productive data-race analysis on the rest of the codebase.
Otherwise, with testing limited to one execution,
%even if one wishes to test for lock bugs,
data-race analysis would flag every access pair in the lock implementation, requiring human attention to verify.
However, Iterative Deepening can automatically verify a large quantity of data-race candidates as benign.
Hence, we extended \landslide~with a custom option to change the lock-set tracking to include accesses from {\tt mutex\_lock()} and {\tt mutex\_unlock()} in the analysis. (Accesses from other sync functions, such as {\tt cond\_wait()}, would either be included already, or be protected by an internal mutex.)

\subsection{Quicksand}

\quicksand~is an independent program that wraps the execution of several \landslide~instances.
The implementation is roughly 3000 lines of C.
It uses a thread pool to schedule the available state spaces,
sorting such jobs according to their status among a running queue, pending queue, and suspended queue.
Jobs are further prioritized by number of PPs, ETA, and whether they include data-race PPs.

{\bf Initial PPs.}
For testing user-space code, we seed the exploration with the four subsets of ``hard-coded'' locking API PPs that we showed in Figure~\ref{fig:id}:
$\{yield\}$,
$\{yield,lock\}$,
$\{yield,unlock\}$,
and $\{yield,lock,unlock\}$,
By extension, these also introduce preemptions on any other sync primitives implemented with mutexes,
such as condition variables, semaphores, and r/w locks.
Preempting on voluntary switches such as {\tt yield} is always necessary to maintain the invariant that only one thread runs between consecutive PPs.
%so the {\tt yield} PP is always implicitly enabled.

For kernel-level testing, we consider interrupt-disabling to be analogous to locking,
so we also preempt just before a disable-interrupt opcode ({\tt cli}) and just after interrupts are re-enabled ({\tt sti})\footnote{
%(to appropriate the names of the x86 instructions)
During data-race detection, {\tt cli}/{\tt sti} are treated as a single global lock.
%as {\tt cli}'d memory accesses can still race with others that have interrupts on.
Some kernels disable preemption without disabling interrupts,
which can be modelled the same way using manual annotations. %of that API.
This also assumes uni-processor scheduling; for SMP kernels, replace {\tt cli}/{\tt sti} with spinlocks.}.
\quicksand~is configured to begin with the subsets
$\{yield\}$,
$\{yield,lock\}$,
$\{yield,unlock\}$,
$\{yield,cli\}$,
$\{yield,sti\}$,
and $\{yield,lock,$ $unlock,cli,sti\}$.
As a heuristic, we don't bother with every intermediate subset such as $\{lock,sti\}$,
which could potentially be improved in future work (\sect{\ref{sec:future}}).

{\bf Communication protocol.}
The interface to \landslide~, which any similar MC could implement, has two parts.
First, when starting each job, \quicksand~creates a configuration file declaring which PPs to use,
% can lose this line due to space
among other options such as mutex-testing mode,
passed as an argument to \landslide.
Then, a dedicated \quicksand~thread communicates with the \landslide~process via message-passing. %on a FIFO pipe.
\landslide~messages \quicksand~after testing each new interleaving to report updated progress and ETA,
whenever a new data-race candidate is found, and whenever a bug is found.
\quicksand~in turn replies whether the test should suspend/resume due to too high ETA, or quit due to timeout.
We implement suspending jobs simply by making \landslide~wait on a message-passing call.
Should \quicksand~later resume a suspended job, we send a message to continue,
causing the \landslide~instance to resume exploring where it left off;
otherwise, we send a message only after time runs out, causing it to exit.

{\bf Heuristics.}
% List of all heuristix:
% HOMESTRETCH - last 60sec of test, don't suspend
% ETA_THRESH - "to let its ETA stabilize"
% eta factor
% shold_reproduce -- small dr jobs are not allowed to add further instances of themself (why not? don't remember)
% priority change between suspected and confirmed dr
Algorithm~\ref{alg:shouldworkblock} allows heuristically scaling a job's ETA when comparing to the time budget,
to express how optimistic we are about the estimate's accuracy.
We use 2 as this scaling factor based on the results in \cite{estimation},
though we allow changing it via the command line.
We also include a heuristic to
%ignore ETAs entirely
never suspend jobs before they pass a certain threshold of interleavings tested,
for which we choose 32,
so that their ETAs have some time to stabilize.

We classify data-race candidates as {\em single-order} or {\em both-order} \cite{portend}
based on whether the MC observed the racing instructions ordered one or both ways in the original state space,
Jobs with both-order data-race PPs are prioritized higher,
because single-order candidates are more likely to be false-positives
(though before preempting during the access itself, we cannot say for sure, hence the heuristic).
%so we prioritize jobs with both-order data-race PPs.
For single-order races, we do not add a PP for the later access at all;
should it be needed, preempting on the first access will suffice to upgrade the race to both-order.

% If there's room, mention the cant_swap mechanism for killing the top half of deferred jobs.

\subsection{Data-race preemption points}

When \landslide~detects a data race, it reports each of the two memory accesses involved in the race.
Each report indicates the program counter value (PC) associated with the access, as well as some further conditions to help filter away unrelated executions of the same instruction on different data.
(For example, many parts of a codebase might call {\tt list\_insert()}, but only one callsite does so without adequate locking.)
Ideally, the PC would be qualified by a full backtrace, but tracing the stack is too expensive to do for each shared memory access.
Instead, \landslide~qualifies the PC with
(a) the current thread ID and
% FIXME: We don't actually do this.
(b) the most recent {\tt call} instruction.
% (a crude approximation of a stack trace)
% which are much cheaper, as we carry them around all the time already
Note that we do {\em not} qualify data races by the shared memory address,
which can change based on different interleavings of previous code
(for example, depending on the result of {\tt malloc()}).
% especially when malloc is involved.
%Figure~\ref{fig:dont-filter-dr-by-address} shows example code where qualifying by memory address will miss the bug.

\begin{figure}[t]
	\includegraphics[width=0.48\textwidth]{dr-jobs.pdf}
	\caption{\quicksand~manages the exploration of multiple state spaces, communicating with each MC instance to receive ETAs, data race candidates, and bug reports.
		When an access pair is reported as a data race, we generate a new PP for each access and add new jobs corresponding to different combinations of those with the existing PPs.}
	\label{fig:new-dr-jobs}
\end{figure}


When \quicksand~receives a data race report, it adds two new jobs to its workqueue:
a ``small'' job to preempt on the racing instruction only,
and a ``big'' job to preempt on that instruction as well as each PP used by the reporting job.
%
Hence, together with the logic in \sect{\ref{sec:classifying}}, each {\em pair} of racing accesses will spawn four new jobs, as shown in Figure~\ref{fig:new-dr-jobs}.
%
The rationale of spawning multiple jobs is that which will be more fruitful cannot be known in advance:
while the big job risks not completing in time,
the small job risks missing the data race entirely if the original PPs were required to expose it.
In practice, we observed some bugs found quickly by these small jobs, and other bugs missed by the small jobs found eventually by the big jobs,
which motivates the need for Iterative Deepening to prioritize the jobs at runtime.
% TODO: Put numbers here.


\section{Soundness}
\label{sec:soundness}

In this section we present two theorems concerning Iterative Deepening's correctness.
Our full proofs,
available at \cite{quicksand-soundness},
%{\em [submitted as supplementary material; will be cited as a tech report in the final version of the paper]}
discuss our assumptions explicitly and include more formal definitions and structure.

\revisionx{These proofs are built on a DPOR algorithm definition which assumes sequentially-consistent memory hardware,
as discussed in \sect{\ref{sec:overview-sssmc}}.
We also assume the Limited HB definition
for the data-race analysis, as discussed in
\sect{\ref{sec:overview-dr}}.
%leaving the case for precise HB to future work.
}

\renewcommand\proofname{Proof Sketch}

%%%%%%%%%%%%%%%%%%%%%%%%%%%%%%%%%%%%%%%%%%%%%%%%%%%%%%%%%%%%%%%%%%%%%%%%%%%%%%%%

\subsection{Convergence to Total Verification}
\label{sec:totalverif}

Although Iterative Deepening's main purpose is to heuristically choose the most effective PP subsets to test
when the maximal state space is too large,
some tests may be small enough that even their maximal state spaces could be completed in time.
For such tests, preempting on every shared memory access \cite{spin,inspect} would provide a total verification of all possible thread schedules, if it could complete in time.
In this section, we show that Iterative Deepening provides a verification of the same strength if it completes the state spaces associated with every discovered data-race PP.
%In other words, contrapositively,
%if it is possible to find a bug with any sequence of preemptions on any instruction whatsoever,
%an equivalent thread interleaving will be reachable using only data-race PPs and synchronization API PPs.
A proof sketch of the contrapositive statement follows.

\newcommand\ppnext[1]{\ensuremath{\mathsf{next}(#1)}}
\newcommand\ppinstr[1]{\ensuremath{\mathsf{instr}(#1)}}
\newcommand\ppothers[1]{\ensuremath{\mathsf{others}(#1)}}
\begin{theorem}[Convergence]
If a bug can be exposed by any thread interleaving possible by preempting on all instructions during a specific test,
Iterative Deepening will eventually test an equivalent interleaving which exposes the same bug.
	\label{thm:convergence}
\end{theorem}
\begin{proof}
The proof has two parts:
first, we show that preempting on data-racing instructions and synchronization API boundaries suffices to test all possible program behaviour;
second, we show that Iterative Deepening will eventually detect all such data races.
Given a PP $p$, let $\ppnext{p}$ denote the next transition after $p$ executed by the thread which ran immediately before $p$,
let $\ppinstr{p}$ denote the first instruction of $\ppnext{p}$,
and let $\ppothers{p}$ denote the transitions by other threads between $p$ and $\ppnext{p}$.

\begin{lemma}[Equivalence of non-data-race PPs]
For any thread interleaving possible by preempting on any instruction,
there exists an equivalent interleaving which uses only data-race PPs and synchronization API PPs.
	\label{lem:relevant}
\end{lemma}

Let $p$ be the first PP in the given interleaving such that $\ppinstr{p}$ is not a data race with $\ppothers{p}$ nor is a synchronization API boundary.
Because $\ppinstr{p}$ is not a synchronization boundary,
no lock can be held during $\ppothers{p}$ that was also held by the first thread across $p$.
Hence, because $\ppinstr{p}$ is not a data race, it cannot be a shared memory conflict with $\ppothers{p}$ at all.
Let $i$ be the first instruction among $\ppnext{p}$ which is such a conflict, or a synchronization boundary.
If $i$ is a shared memory conflict, it must be a data race, for the same reasoning as above.
We modify the input interleaving by reordering $\ppinstr{p}$ until $i$, not including $i$, to before $\ppothers{p}$.
By the soundness of DPOR \cite{dpor}, this is equivalent to the input interleaving.
In other words, we have transformed $p$ into $p'$ such that $\ppnext{p'} = i$, which is a data race or synchronization boundary.
All PPs in the input trace can be inductively converted in the same manner.

\begin{definition}[Reachability]
A data race candidate, and its associated PPs, are reachable if it will be identified by an MC configured to preempt only on already-reachable PPs.
\end{definition}
Initially, the statically-available synchronization API PPs are reachable. Reachability of data-race PPs is transitive.

\begin{lemma}[Saturation of data-race PPs]
	Given any interleaving comprising only data-race PPs and synchronization API PPs, all involved PPs are reachable.
	\label{lem:saturation}
\end{lemma}

We induct on the PPs according to the order of their preemptions.
Given that the interleaving prefix preceding some PP $p$ is reachable,
we require that either $p$ is reachable, or a new data race among $\ppothers{p}$ will be newly reachable. %, not previously reachable, is now reachable.
The latter condition suffices because in a finitely-sized codebase, there must be finitely many unique racing instruction pairs.
%, so induction on the number of new data-race PPs among $\ppothers{p}$ will make $p$ itself reachable.

First, we must ``coalesce'' away $p$, as well as any other not-yet-reachable PPs in $\ppothers{p}$.
Consider the alternate interleaving in which the first thread executes past $p$ until the first already-reachable PP,
then the other threads among $\ppothers{p}$ execute the same way.
This interleaving's PPs are all reachable, so a state space $\mathcal{S}$ containing it will be tested.

If $p$ is a not-yet-reachable data-race PP,
it must be possible for some other thread to execute a data-racing instruction with $\ppinstr{p}$.
If this conflict was observed in the state space containing our coalesced interleaving, we have reached $p$.
Otherwise, we appeal to the soundness of DPOR:
If a program behaviour is possible by interleaving threads at the boundaries of the given transitions,
it will be tested in the containing state space.
By contrapositive, to expose this behaviour, one or more preemptions must occur in the middle of some transition, rather than at the boundaries.

We now show by contradiction there cannot be {\em multiple} data-race PPs which must all be enabled before either data race can be identified.
Assume there does not exist a single transition $t_1 \in \mathcal{S}$ which alone can be split into $\{t_1',t_1''\}$ by a PP $q$,
such that another thread's concurrent transition $t_2$ conflicts with $t_1''$.
By the soundness of DPOR, because all $t_2$s are independent with $t_1''$, $\mathcal{S} \equiv \mathcal{S} \cup q$.
Replacing $\mathcal{S}$ with $\mathcal{S} \cup q$ in the above assumption shows that no {\em pair} of new $q$s would expose new program behaviour, and inductively, no set of $q$s of any size, which contradicts the previous paragraph.

Hence, a single new not-yet-reachable data race is reachable in $\mathcal{S}$. Hence $p$ will be reached.
\\

To conclude,
for any possible interleaving, Lemma \ref{lem:relevant} provides an equivalent one with only data-race and synchronization PPs,
and Lemma \ref{lem:saturation} proves all involved PPs are reachable.
Hence, Iterative Deepening will eventually test a state space containing the equivalent buggy interleaving.
\end{proof}

%%%%%%%%%%%%%%%%%%%%%%%%%%%%%%%%%%%%%%%%%%%%%%%%%%%%%%%%%%%%%%%%%%%%%%%%%%%%%%%%

\subsection{Suppressing ``Malloc-Recycle'' False Positives}
\label{sec:recycle}

We identify a particular class of false positive data-race candidates \revisionx{under Limited HB} in which the associated memory was recycled by re-allocation between the two accesses.
Figure~\ref{fig:recycle} shows a common code pattern and interleaving which can expose such behaviour.
If the {\tt malloc} on line 4 returns the same address passed to {\tt free} on line 2, then lines 1 and 7 will be flagged as a \revision{potential}~data race.
We call this a {\em malloc-recycle data race \revision{candidiate}}.
To the human eye, this is obviously a false positive: reordering lines 4-7 before lines 1-2 will change {\tt malloc}'s return value, causing {\tt x} and {\tt y} to no longer collide.
Here, Thread 2's logic usually corresponds to an initialization pattern \cite{eraser}, but for generality we have added a {\tt publish} action on line 6.

%However,
When limited to a single test execution, suppressing any data race \revision{candidate}~matching this pattern is unsound.
Consider the more unusual program in Figure~\ref{fig:recycle-bug},
in which the memory is recycled the same way, but the racing access's address is not tied to {\tt malloc}'s return value.
Here, reordering lines 6-7 before line 3 will allow {\tt x} and {\tt x2} to race.
\revisionx{Such collisions could be avoided with a hacked allocator which never recycles memory, but this could unacceptably impact performance in {\tt malloc}-heavy tests.}

\begin{figure}[t]
	\small
\begin{tabular}{rll}
	& \multicolumn{2}{c}{\texttt{struct x \{ int foo; int baz; \} *x;}} \\
	& \multicolumn{2}{c}{\texttt{struct y \{ int bar; \} *y;~~~~~~~~~~}} \\
	\\
	& {\bf Thread 1} & {\bf Thread 2} \\
	1 & \texttt{\hilight{brickred}{x->foo = ...;}} & \\
	2 & \texttt{\hilight{olivegreen}{free}(x);} \\
	3 & & \texttt{\hilight{commentblue}{// x's memory recycled}} \\
	4 & & \texttt{y~=~\hilight{olivegreen}{malloc}(sizeof *y);} \\
	5 & & \texttt{\hilight{commentblue}{// ...initialize...}}\\
	6 & & \texttt{publish(y);} \\
	7 & & \texttt{\hilight{brickred}{y->bar = ...;}} \\
\end{tabular}
\caption{A common execution pattern with {\tt malloc()} that produces false positive data race candidates.}
\label{fig:recycle}
\end{figure}
\begin{figure}[t]
	\small
\begin{tabular}{rll}
	& {\bf Thread 1} & {\bf Thread 2} \\
	1 & \texttt{publish(x);} & \\
	2 & \texttt{\hilight{brickred}{x->foo = ...;}} & \\
	3 & \texttt{\hilight{olivegreen}{free}(x);} \\
	4 & & \texttt{x2 = get\_published\_x();} \\
	5 & & \texttt{\hilight{commentblue}{// x's memory recycled}} \\
	6 & & \texttt{y~=~\hilight{olivegreen}{malloc}(sizeof *y);} \\
	7 & & \texttt{\hilight{brickred}{x2->foo = ...;}} \\
\end{tabular}
	\caption{If a single-pass \revision{Limited HB analysis}~discarded candidates matching the malloc-recycle pattern,
it would miss the bug in this adversarial program.}
\label{fig:recycle-bug}
\end{figure}

Fortunately, when data-race detection is combined with DPOR and Iterative Deepening, pruning all malloc-recycle candidates is sound, even considering adversarial programs such as Figure~\ref{fig:recycle-bug}.
This makes it unnecessary to verify such \revision{candidates}~by actively adding more preemptions,
achieving a potentially combinatorial reduction in how many state spaces we generate.
%Intuitively, we need not worry about cases such as Figure~\ref{fig:recycle-bug} because,
%should they be true races,
%DPOR will reorder threads sufficiently for the malloc-recycle pattern to disappear.
We provide a proof sketch below.

\begin{theorem}[Soundness of eliminating malloc-recycle candidates]
	If a malloc-recycle \revision{candidate}~is not a false positive,
%DPOR will reorder threads such that
DPOR will test an alternate thread interleaving in which
%either
the accesses can race without fitting the malloc-recycle pattern.
%, or a use-after-free bug will be reported immediately.
\end{theorem}

\begin{proof}
%By definition of the malloc-recycle pattern,
Any such program must contain an access $a_1$ by one thread T1,
followed by a {\tt free} and a {\tt malloc} possibly by either thread,
followed by an access $a_2$ by the other thread T2. % not depending on the result of the middle malloc.
Without loss of generality, we say that T1 performs the {\tt free} and T2 the subsequent {\tt malloc}. %; the other cases are similar.
We also assume the only way for the program to get pointers to heap memory is through {\tt malloc};
hence, there must also be some ``publish'' action $p$ by T1 which communicates the address to T2.
Because this is a true \revision{potential}~data race, $p$ must occur before $a_1$, as $a_2$ cannot be reordered before $p$.

We require that a PP will be identified during T1 between $p$ and $a_1$.
The publish action must involve some thread communication, whether through a shared data structure or message-passing API.
If locking or message-passing is used, our set of hard-coded PPs suffices to provide a PP.
	Otherwise, $p$ (and the corresponding read by T2) will be a \revision{potential}~data race, although that may itself be a malloc-recycle \revision{candidate}.
In this case we use induction on the pointer chain leading to the shared address containing $p$:
in the base case, $p$ is communicated via global data or message-passing,
and in the inductive step, DPOR will reorder threads sufficiently to identify the PP on $p$.
Hence there will be a PP between $p$ and $a_1$ no matter the mode of communication.

With this PP, DPOR will reorder $a_2$ before $a_1$, while not changing $a_2$'s location.
As T2's {\tt malloc} now occurs before T1's {\tt free}, it will allocate different memory.
Hence $a_1$ and $a_2$ %will be in the same allocation;
can race without fitting the malloc-recycle pattern.
% Mario-man is very very hunger from not having enough plumming jobs, so his Quest for Eat and Dollars.
% This spells QED so we are done.
\end{proof}
\renewcommand\proofname{Proof}


We implemented a simple check in \landslide~to recognize the malloc-recycle pattern:
%We mechanically recognize when {\tt x} and {\tt y} correspond to different abstract allocations despite colliding on address,
%We implemented this check
%by adding a generation counter to \landslide's heap tracking:
%when tracking heap state,
each heap allocation is given a unique ID,
and when evaluating whether two heap accesses can race,
the IDs of their containing blocks must match.
Note that \revision{this proof does}~not require PPs on \revision{{\tt malloc}'s internal lock},
%the internal lock of {\tt malloc} or {\tt free},
which is an ideal candidate to ignore via {\tt without\_function} (\sect{\ref{sec:landslide}}) to reduce state space size.

\revision{This class of false positive is unique to heap-allocated memory among all ways threads could communicate.
By contrast, global memory has unlimited lifetime,
and message-passing primitives enforce an ordering which precludes the race. %(even under Limited HB).
Finally, note that as long as concurrent {\tt malloc} is implemented with an internal lock,
these false positives are of concern only under a Limited HB analysis (\sect{\ref{sec:overview-dr}}).
Nevertheless,
%our previous proof's
Theorem~\ref{thm:convergence}'s need for the Limited HB definition justifies our choice of it over full HB.
}

%%Because concurrent {\tt malloc} is often implemented with an internal lock, under a {\em pure} happens-before analysis,
%Note that under a {\em pure happens-before} analysis,
%these accesses are not considered concurrent % at all
%because of {\tt malloc}'s internal locking events,
%and would not result in such false positives.
%However, pure happens-before can miss many real bugs \cite{hybriddatarace,tsan},
%so in our context it is more appropriate to use the
%{\em limited happens-before} relation in a hybrid approach with lockset tracking.
%the hybrid approach combining lockset tracking and the {\em limited} happens-before relation is not vulnerable to false negatives,

% TODO: Abbreviate SMC, and fixed-PP approach.

\section{Evaluation}

Although \quicksand~presents Iterative Deepening and data-race PPs as interconnected techniques, they each could theoretically be employed alone in other model checkers.
For example, a single-state-space tool could use data-race candidates during immediately subsequent interleavings, essentially changing the state space on the fly.
Likewise, a message-passing-only tool could employ Iterative Deepening despite data races being absent from its concurrency model.
Hence, though many of our experiments compare \quicksand~to the state-of-the-art as a whole,
we also sought to evaluate each technique individually.
Our evaluation answers the following questions:
\begin{enumerate}

	\item Does \quicksand~improve upon state-of-the-art MC?
		\begin{enumerate}
			\item Does Iterative Deepening find bugs faster
				%than SSS-MC
				in subset state spaces, even without data-race PPs?
			% Probably not... ICB is state of the art here.
			% In large tests, can Iterative Deepening provide partial verification by completing smaller state spaces
			\item Do data-race PPs expose new bugs that couldn't be found with SSS-MC's fixed-PP-set approach?
				% Elaborate later:
				% Among those, how many were missed in a {\em completed} execution of the otherwise ``maximal'' state space?
		\end{enumerate}
	\item Does MC improve the accuracy of data-race detection?
		\begin{enumerate}
			\item Do we avoid false positives compared to a single-execution data race analysis?
				% Explain later as:
				% How many data-race candidates were verified as benign
				% But to be fair, you have to count how many DRs are reported as "couldn't test these, check yourself" at the end.
				% Also Include:
				% How many false positives does the free-re-malloc technique suppress?
				% TODO: If you have time, re-run all of the dr-only bug tests, with DR_FALSE_NEG enabled, and see how much fewer bugs get found (how many bugs get pushed past the time limit?)
			% TODO: This one's optional. You can give up on it.
			\item Do we find data-race bugs that would be false-negatives during a single-execution analysis?%Do we avoid false negatives compared to single-pass?
				%TODO: for this experiment, set EXPLORE_BACKWARDS=0
				% TODO: And disable false-neg malloc-free technique
				% TODO: And also disable the confirmed/suspected thing
				% (where mem.c waits for reorder observed before
				% messaging the latter half of the DR to QS)
		\end{enumerate}
\end{enumerate}

%%%%%%%%%%%%%%%%%%%%%%%%%%%%%%%%%%%%%%%%%%%%%%%%%%%%%%%%%%%%%%%%%%%%%%%%%%%%%%%%

\subsection{Test Suite}
% TODO: Maybe say how many lines of code total? How many lines on average per P2/pintos? (careful with pintos; lots of basecode)
Our test suite consists of \numthrlibs~``P2'' student thread libraries, from CMU's 15-410 OS class,
%across the Spring and Fall 2014 and Spring 2015 semesters;
and \numpintoses~``Pintos'' student kernels, from Berkeley's CS162 and U. Chicago's CS230 OS classes.
%
The P2 thread library comprises \texttt{thr\_create()}, \texttt{thr\_exit()}, \texttt{thr\_join()}, mutexes, condvars, semaphores, and r/w locks;
all implemented from scratch in userspace with a UNIX-like system call interface \cite{kspec,thrlib}.
%
The Pintos kernel project
involves implementing priority scheduling, \texttt{sleep()}, and user-space process management (\texttt{wait()} and \texttt{exit()})
using provided bare-bones mutex, context-switch, and virtal memory implementations
\cite{pintos}.
% P2 SLOC stats: 1807 avg; 1723 median; range 1181-4114.
% All numbers, obtained with:
% cd p2s; for i in */*; do wc -l $i/user/libthread/*.{c,h,S} $i/user/libthread/*/*.{c,h,S} ; done | grep total
% 1181 1192 1221 1230 1238 1240 1243 1261 1275 1307 1310 1318 1325 1334 1336 1345 1366
% 1388 1388 1403 1415 1416 1430 1451 1478 1498 1527 1589 1618 1635 1638 1654 1675 1676
% 1716 1719 1720 1723 1723 1727 1737 1743 1744 1751 1769 1777 1782 1789 1789 1812 1918
% 1926 1946 1994 2022 2043 2066 2077 2088 2099 2131 2164 2172 2190 2215 2227 2277 2282
% 2384 2387 2483 2486 2503 2514 2551 2597 2610 2665 4114
Though not ``real world'' programs, both projects are quite large: % maybe "complex"?
the P2s average 1807 lines of C and x86 assembly (stddev 489.5),
% Pintos SLOC stats: TODO
and the Pintoses average {\bf 9999999} % TODO

\newcommand\mxtest{\texttt{mx\_test}}
\newcommand\tej{\texttt{thr\_exit\_join}}
\newcommand\bct{\texttt{broadcast}}
\newcommand\paraguay{\texttt{paraguay}}
\newcommand\paradise{\texttt{paradise\_lost}}
\newcommand\rwl{\texttt{rwl\_test}}
We tested P2s with 6 multithreaded programs:
% from the 410 test suite % XXX: I would like to say this but this is a lie; figure out what else i can say instead
% each tailored to exercise a different part of the P2 project
\mxtest, for locking algorithm correctness, \tej, a test of thread lifecycle, \bct~and \paraguay, for condvars, \paradise~for semaphores, and \rwl~for r/w locks.
For \mxtest, \paradise, and \paraguay, we used {\tt without\_function} to blacklist {\tt thr\_create}, {\tt thr\_exit}, and {\tt thr\_join},
and for \mxtest~we enabled \landslide's mutex-testing option
(see \sect{\ref{sec:landslide}}).
\newcommand\prisema{\texttt{priority\_sema}}
\newcommand\waitsimple{\texttt{wait\_simple}}
% TODO: Add more test cases
We tested the Pintoses with 2 programs from the class test suite: \prisema, a test of the kernel scheduling algorithm, and \waitsimple, a test of process lifecycle system calls.
For all tests, we used {\tt without\_function} to blacklist PPs on the {\tt malloc} mutex.
% XXX: Some pintoses can't run all the tests. So this number is too high.
In total, this test suite comprises 632 unique state spaces.
All tests were run on 12-core 3.2 GHz Xeon machines with 12GB of RAM.

\begin{table}[t]
	\begin{tabular}{l|l|l}
			& QS bugs & SSS-MC bugs \\
		\hline
		\mxtest & eg 1000 & eg 0 \\
		\bct & & \\
		etc... & & \\
		\hline
		Total & & \\
	\end{tabular}
	\caption{Comparison of all bugs found, broken down by test case, among all P2s (top 6) and Pintoses (bottom 2)}
	\label{tab:allbugs}
\end{table}

\begin{table}[t]
	\small
	\begin{tabular}{l|l|l||l|l}
	& QS bug & \begin{tabular}{c} SSS-MC \\ completed\end{tabular}
	& QS bug & \begin{tabular}{c}SSS-MC \\ timeout \end{tabular} \\
		\hline
		\mxtest & e.g. 5 & 10 & 0 & 0 \\
		\bct & & & & \\
		etc... & & & & \\
		\hline
		Total & & & & \\
	\end{tabular}
	\caption{Bugs requiring data-race PPs to expose, found by \quicksand~but missed by the single-state-space approach.}
	\label{tab:drbugs}
\end{table}

%%%%%%%%%%%%%%%%%%%%%%%%%%%%%%%%%%%%%%%%%%%%%%%%%%%%%%%%%%%%%%%%%%%%%%%%%%%%%%%%

\subsection{Comparing Iterative Deepening to SSS-MC}
\label{sec:eval-sssmc}

% TODO: If you have time, rerun all the quicksand experiments JUST running the 4 base state spaces. Give it 10/4 hours of cpu budget on 4 cores. John's expt.

% mention exactly which state spaces we are comparing here
% mention partial verification in terms of state space completion, when SSSMC times out.

%In this section we show that using data-race PPs with \landslide~is more effective than either SSS-MC or single-pass data-race detection alone.

To compare to SSS-MC, we ran a control experiment for each test, running \landslide~on a single state space with all PPs on sync primitives enabled in advance (and no data-race PPs).
We gave each \quicksand~test 10 CPUs for 1 hour each. % XXX
% TODO figure out somewhere to mention what landslide's pps are: mx lock/unlock (aka sema up/down)
Though \landslide~does not implement parallel DPOR \cite{parallel-dpor}, we compensated by giving each control test 1 CPU for 10 hours,
%then dividing all associated times by 10 (simulating perfect parallelism).
and instrumenting \quicksand~to report total CPU-hours rather than wall-clock time.
%\quicksand's times by 10 to convert from wall-clock time to CPU-hours (even though it sometimes falls short of 100\% parallelism).
Figure~\ref{fig:dowefindbugsfaster} plots the bug-finding speed of SSS-MC against that of two different \quicksand~experiments:
%which we explain presently.

\begin{figure}[t]
	\includegraphics[width=0.48\textwidth]{dowefindbugsfaster.pdf}
	\caption{Comparison of bug-finding performance
	by several configurations of \quicksand~and the SSS-MC control.
	\quicksand~finds 169\% as many bugs with data-race PPs.}
	\label{fig:dowefindbugsfaster}
\end{figure}

{\bf Finding the same bugs faster.}
To show that Iterative Deepening is effective even for MC domains without data races, such as message-passing distributed systems,
we ran the test suite with \quicksand~configured to explore only subsets of the hard-coded mutex PPs (i.e., ignoring all data-race reports)\footnote{
Because \quicksand~is not yet instrumented to subset hard-coded PPs beyond the 4 ways shown in Figure~\ref{fig:id},
we ran these tests for 2.5 hours on 4 CPUs each.
Future work could parallelize QS-no-DR-PPs further; see \sect{\ref{sec:future}}.}.
%
The line QS-no-DR-PPs represents this experiment;
% TODO: rephrase based on result of expt
we see that even though SSS-MC mostly catches up to it by the end of the 10-hour budget,
QS-no-DR-PPs finds many more of the bugs much sooner.
From this we conclude that for smaller arbitrary CPU budgets,
Iterative Deepening is likely to find bugs SSS-MC will miss,
and programmers can be more confident in the verification provided when \quicksand~times out with no bug found.

{\bf Finding new data-race bugs.}
Though state-of-the-art MCs preempt only on synchronization events, many serious concurrency bugs are caused by data races leading to corrupted shared state.
The line QS-DRs represents our ``no holds barred'' \quicksand~tests:
we quickly pull ahead of SSS-MC, and ultimately conclude with 69\% more bugs in total.
The break-even point is at a negligible 45 seconds.

Furthermore, we plotted another line from this dataset, QS-no-DR-bugs,
which represents only the bugs found in state spaces without data-race PPs (like QS-no-DR-PPs, but even when data-race PPs are enabled).
Intuitively, this line shows that for programs with only benign data races,
\quicksand~can afford the extra overhead of verifying them while still slightly edging out SSS-MC\footnote{
The initial perfect overlap between QS-DRs and QS-no-DR-bugs indicates how long it takes before the first data-race bug is found.}.
%even after the extra overhead of verifying them, \quicksand~still slightly edges out SSS-MC

{\bf Partial verification guarantees.}
TODO % TODO

%Hence, in Table~\ref{tab:drbugs} we count how often \quicksand~uncovered a bug only in state spaces which included data-race PPs, while
%
%In Table~\ref{tab:allbugs} ....

%%%%%%%%%%%%%%%%%%%%%%%%%%%%%%%%%%%%%%%%%%%%%%%%%%%%%%%%%%%%%%%%%%%%%%%%%%%%%%%%

\subsection{Avoiding false positive data-race candidates}
\label{sec:eval-falsepos}
% Though we mechanically verify whether each data race candidate leads to a bug, each new PP can increase combinatorially..... obviously wish to avoid...

We also counted the number of malloc-recycle false positives that \landslide~suppressed.

%%%%%%%%%%%%%%%%%%%%%%%%%%%%%%%%%%%%%%%%%%%%%%%%%%%%%%%%%%%%%%%%%%%%%%%%%%%%%%%%

\subsection{Finding nondeterministic data-race candidates}
\label{sec:eval-falseneg}
Some memory accesses may be hidden in a control flow path that requires a nondeterministic preemption to be executed.
In such cases, a single-pass dynamic data-race detector
%could not achieve the coverage necessary
would fail
to identify a racing access pair as a candidate to begin with.
%
We counted how many such data-races, used as PPs, led to \quicksand~finding new bugs,
thereby making them {\em false negatives} of the single-pass approach.
% TODO: Put a figure here giving an example of where e.g. a data race only shows up during the contention path of a mutex.
We classified each data-race candidate according to whether \landslide~had reported them during the first interleaving, before any backtracking or preempting: if so, they were {\em deterministic data races} (hence could be found by single-pass).

To ensure a fair comparison, we disabled \landslide's {\em false-positive}-avoidance techniques during this experiment.
For example, we reported malloc-recycle data races during the first interleaving as {\em deterministic}, as a single-pass analysis must,
rather than waiting until future interleavings to confirm them (as explained in \sect{\ref{sec:recycle}}).

				% TODO: Argue:
				% It is fair to compare multiple pass DR-analysis under Landslide against just a single execution because prior work DR detectors, being not integrated with a MC, are not intended to uncover different results under subsequent runs.
				% Define a "stress tester" as a class of bug detectors where they are intended to [...]
				% Maybe say: Do there exist any stress-tester DR detectors where they are intended to produce new results under reruns?

% TODO: restructure this paragraph to account for RaceFuzzer/Portend. Be like, "Yes they explore multiple schedules, but only AFTER finding a race." And/or make this argument in prior work
One might also wonder: Why is it fair to compare the data race bugs \quicksand~finds (10 CPU-hours)
against the candidates found by a prior work's single execution (less than a minute)?
We argue this comparison is meaningful because prior work data-race tools, being not integrated with a MC,
are not intended to uncover different results under subsequent runs.
One could run a data-race tool repeatedly for 10 CPU-hours, but the advantage of stateless MC over stress testing is already well-understood.
% TODO TODO TODO get a big citation for teh above sentence.


% Figure out concretely what the data race tricks are that we do, so we can claim them as contributions in the paper. Then ACTUALLY EVALUATE THEM.
%         - Speculative DR PPs.
%                 Not a heuristic, rather how to make it work at all to begin with.
%                 (Cite MS thesis, claim on backwards explorating finding bugs faster)
%         - Free/re-malloc to eliminate some false positives. See #193.
%                 Measure how many false positives are eliminated.
%                 Check, ofc, to make ABSOLUTE SURE, that no bugs missed w/ this trick.
%                         If there are, it could be because of the implementation
%                         bug described in #193.
%         - Using tid/last_call filtering because whole stack traces are too expensive.
%                 Moderately optional, 1st priority since theoretically interesting:
%                 Turn on/off and measure how resulting DR bug #s change.
%         - Optional: Reprioritizing DRs based on "confirmed" / "suspected"
%                 Shouldn't be hard just make ID wrapper print "s" or "c"!
%                 Is it helpful for ID to put priorities on DR PPs?
%                         Test by inverting the priority and see if fewer buges are found.
%         // Super optional to talk about. Probably not worth the time.
%         // - "Too suspicious" (during init/destroy)
%         //      (Cite eraser, section 2.2)

%%%%%%%%%%%%%%%%%%%%%%%%%%%%%%%%%%%%%%%%%%%%%%%%%%%%%%%%%%%%%%%%%%%%%%%%%%%%%%%%

\subsection{Discussion and future work}
\label{sec:future}

%TODO: Run a mutex expt where "all atomic instrs" are PPs. See how many bugs are missed anyway.
% Wwe re-ran the \mxtest control experiment with \landslide~hard-coded to preempt on any atomic instruction
% (as well as on the mutex API boundaries).
% Still, this smarter configuration for SSS-MC found only 99999999999 bugs of \quicksand's 13.

{\bf Testing lock implementations.}
In \sect{\ref{sec:eval-sssmc}}, using data-race PPs compares favourably across the board to SSS-MC.
Observe in particular that in \mxtest, the control experiment found dramatically fewer bugs (just 1),
even compared to the other test cases\footnote{
% FIXME: "Aren't all lock impls assembly?" "Yes, but this one was ALL assembly."
	The one bug SSS-MC found was in a fully-assembly lock implementation. {\tt yield()}'s return value clobbered a value stored in {\tt \%eax}, which could lead to a failure after two repeated contentions. Preempting only on {\tt yield()} (in the contention loop) was sufficient.}.
%Intuitively, this is due to our control experiment being able to preempt only on the boundaries of the API which
%Though for many applications of MC, assuming a correct lock implementation is sufficient,
Though it suffices for many applications of MC to assume correctly-implemented locks,
we consider this strong evidence that any verification of low-level synchronization code must incorporate data-race PPs.

% TODO: get these numbers
{\bf Finer-grained PP subsets.}
\quicksand~was able to partially guarantee safety in {\large \bf 99999\%} of tests with large maximal state spaces.
However, in {\large \bf 1337} tests, no more than the minimal state space could be verified,
and in {\large \bf 42} tests, not even that much.
Larger state spaces often result from finer-grained locking,
which can indicate a more complicated concurrent algorithm requiring more rigorous verification than a program with a single global lock.
%Hence these corner cases are important to consider for future work.
While this work used {\tt within\_function} (\sect{\ref{sec:landslide}}) {\em statically} to restrict where PPs could arise in advance of the test,
we envision future Iterative Deepening implementations could incorporate this method to {\em dynamically} subset PPs further,
making partial verification of such large tests possible.
%enabling partial verification of such large tests. % if need space

{\bf Partial verification.}
We are not the first to provide a partial verification guarantee when timing out on too-large state spaces (\sect{\ref{sec:eval-sssmc}}).
While we guarantee safety when preempted on certain combinations of PPs,
CHESS
guarantees safety under no more than a certain number of preemptions \cite{chess-icb}.
%according to the maximum bound reached in the time limit.
We imagine these two guarantees could be each be useful to developers in different scenarios,
and are presently working to combine the two approaches to provide both at once.
One benefit of our technique is that {\tt within\_function}-based Iterative Deepening (discussed above)
would enable expert developers to configure custom subsets of PPs they are most interested in verifying,
according to which modules of a codebase they wish to test.

% Future work: Add parallel DPOR so you can fill your spare CPUs when there are fewer than the max number of jobs.

\section{Discussion}
\label{sec:future}

In this section, we discuss \quicksand's limitations and opportunities for future improvement.

% TODO CAMREADY: Measure how much subset overlap there is.
{\bf Avoiding redundant work.}
When we extend a small state space with more PPs, the new state space is guaranteed to test a superset of interleavings compared to the old one.
Any interleaving which does not preempt threads on any of the new PPs will be repeated work.
%Because we prioritize completing small state spaces before extending them with more PPs,
%the superset state spaces we run later will repeat each branch of their already-completed subsets.
%
%We measured the proportion of repeated work among completed state spaces across our test suite;
%on average, {\bf \large 999\%} of the interleavings in each test were repeated, with some tests as high as {\bf \large 9999\%}.
This may make us slower than SSS-MC to find certain bugs,
for example, if both {\tt lock} and {\tt unlock} PPs together expose a bug, but not either alone.
%Similarly, when pursuing total verification (\sect{\ref{sec:totalverif}}),
%if the state space resulting from preempting on every instruction could be completed,
%an SSS-MC tool such as \cite{spin} might achieve that verification faster,
%as Iterative Deepening will test many subsets of data-race PPs first.
Predicting whether an upcoming interleaving has already been tested is not straightforward,
but we believe future implementations
%of Iterative Deepening
could incorporate cross-job memoization
to prune some or all such repeated work.
\revisionx{Prioritizing the maximal state space in particular could also improve completion times:
whenever the maximal job finishes with no new data-races, future implementations could immediately prune all subset jobs and declare a total verification.}

{\bf Finer-grained PP subsets.}
\quicksand~was able to partially guarantee safety for some PPs in 89\% of tests with too-large maximal state spaces.
However, in 4 cases, no more than the minimal state space could be verified,
and in 6 others, no state spaces were completed at all.
%Larger state spaces often result from finer-grained locking,
%which can indicate a more intricate concurrent algorithm or an unnecessarily complicated design.
%Such programs may require even more rigorous verification than a program with a single global lock.
%Hence these corner cases are important to consider for future work.
While we used {\tt within\_function} (\sect{\ref{sec:landslide}}) {\em statically} to restrict where PPs could arise in advance of the test,
%we envision
future
%Iterative Deepening
implementations could use this mechanism to {\em dynamically} subset PPs further,
making partial verification of larger tests possible.
%enabling partial verification of such large tests. % if need space
%Because our current implementation does not avoid repeated interleavings across state spaces,
%as discussed above,
%we were limited to a small number of very basic PP subsets to statically seed the exploration

\revision{
{\bf Integration with static data-race analysis.}
In \sect{\ref{sec:eval-sssmc}}, we evaluated SSS-MC's ability to find data-race-induced failures
by configuring a static predicate to preempt on any non-stack memory access.
This introduced hundreds of new PPs on each new test execution, with a prohibitive performance impact.
While this performance could be improved by
%relaxing the preemption strategy,
instead using a static or single-pass analysis to find data-race candidate PPs in advance \cite{portend},
this strategy sacrifices the soundness of the verification guarantee, as shown in \sect{\ref{sec:eval-dr}}.
However, \quicksand~itself could employ static data-race analysis \cite{racerx} in future work.
Statically-identified data race candidates could heuristically be included in our ``seed'' PP sets (\sect{\ref{sec:initial-pp}}),
enabling \quicksand~to focus on the most suspicious races immediately, rather than waiting for them to be identified after potentially many iterations of MC.
}


%{\bf Small test cases.}

{\bf Partial verification.}
%We are not the first to provide a partial verification guarantee when timing out on too-large state spaces (\sect{\ref{sec:eval-sssmc}}).
While we guarantee safety when using certain combinations of PPs (\sect{\ref{sec:eval-sssmc}}),
%Iterative Context Bounding
ICB
guarantees safety under no more than a certain number of preemptions \cite{chess-icb}.
%according to the maximum bound reached in the time limit.
%We imagine
These guarantees could each be useful to developers in different scenarios,
and future work could combine the two approaches to provide both at once.
One benefit of our technique is that {\tt within\_function} %-based Iterative Deepening (discussed above)
would enable expert developers to
%configure custom subsets of PPs they are most interested in verifying,
%according to which modules of a codebase they wish to test.
restrict Iterative Deepening to only the modules of a codebase they wish to test.

Likewise, when full verification is not computationally feasible,
some jobs with data-race PPs will time out.
We cannot guarantee those races are
%false positives or
benign, even though no bug was found.
In the ``Untested DR PPs'' column of Table~\ref{tab:drbugs}, we show how many such candidates we could not verify (32\%).
For a more formal treatment of these cases, we refer the reader to the {\em k-witness harmless} metric introduced by \cite{portend},
which could be combined with \quicksand~in future work.
% Future work: Add parallel DPOR so you can fill your spare CPUs when there are fewer than the max number of jobs.

\section{Related Work}
\label{sec:related}

% TODO: ", and related data race analyses based on ????"

\subsection{Model Checking}

\landslide~uses many established model-checking techniques, dating back
% of course
to Verisoft, the original C model checker \cite{verisoft}.
%, and Eraser, the original data race detector \cite{eraser}.
%Our model checker
%\landslide~\cite{landslide} itself implements many techniques from prior work (\sect{\ref{sec:landslide}}).
%itself implements DPOR \cite{dpor},
%state space estimation \cite{estimation},
%and data-race detection \cite{eraser}.
We compare related tools by their treatment of shared-memory thread communication.

{\bf Synchronization events only.} CHESS \cite{chess} and dBug \cite{dbug-ssv} instrument the thread library API, and can preempt programs only during calls to this API.
Hence they will miss any bugs that require interleaving threads at instruction granularity during a data race. CHESS provides a data-race analysis to report any such violations of its concurrency model to the user, but does not incorporate data-race candidates as PPs in future tests.

{\bf Message-passing.} Other stateless model checkers, such as SAMC \cite{samc}, MaceMC \cite{macemc}, MoDist \cite{modist}, and ETA \cite{dbug-retreat}, limit thread communication to a message-passing API to more effectively test distributed systems.
This eliminates the need for data-race analysis, but restricts the class of programs that can be tested.
Nevertheless, Iterative Deepening is still applicable to these tools.

{\bf Preempting at instruction granularity} is a prerequisite for using data-race PPs.
However, the resulting state space explosion demands that any such tool either
choose a small subset of instructions to consider as PPs
or be limited to very small test inputs.
%However, every such prior tool we know of has serious drawbacks.
SKI \cite{ski} approaches kernel code by choosing in advance a random set of instruction offsets from the start of the test,
which is more similar to stress testing or fuzzing than to exhaustive state space exploration.
SPIN \cite{spin} specializes in verifying synchronization primitive implementations such as RCU, which is very similar to our \mxtest~experiment.
However, SPIN is stateful rather than stateless, and explicitly storing visited program states rather than using DPOR limits the size of programs that can be practically tested.
SPIN also requires code to be written in the PROMELA DSL.
%so cannot check implementations directly.

{\bf Other techniques.} Various improvements to DPOR have been proposed, such as Dynamic Interface Reduction \cite{demeter}, Maximal Causality Reduction \cite{mcr}, and DPOR for TSO/PSO \cite{tsopso}.
These are all orthogonal to our technique.
Parrot \cite{parrot} combines MC with a partially-determinizing runtime, but still, fewer than half the non-trivial state spaces in their evaluation could be completed.
%providing a strong argument for \quicksand.
Finally, Iterative Context Bounding (ICB) \cite{chess-icb} is most similar to Iterative Deepening,
as both approaches provide a partial verification on some subset of interleavings when full completion is intractable (\sect{\ref{sec:future}}).
However, ICB is limited to a fixed set of PPs, and so far no algorithm has been proposed to dynamically add data-race PPs during a test with ICB.

\subsection{Data Race Detection}

%Too many related projects to list have made contributions to the
Many advances have been made on the false-positive data race problem since it was first introduced in \cite{eraser}.
\cite{hybriddatarace} and \cite{tsan} combine the lockset and happens-before analyses into a hybrid technique, which we employ.
% TODO: any more?
DroidRacer \cite{droidracer} and CAFA \cite{cafa} extend the analysis to event-driven Android applications, using domain-specific heuristics (orthogonal to our method) to reduce false positives. % cut for space?
% No, IDGAF about pure happens before.
%FastTrack \cite{fasttrack} optimizes the performance of pure happens-

% TODO: Figure out why they claim "Happens before produces NO false positives, only benign races".
% It seems impossible.
% But if true, it means either (a) they can somehow identify FRM DRs on the 1st pass, not needing to replay,
% or (b) reuse of memory by malloc is somehow outside their concurrency model.
Closer to our work, replay analysis \cite{recordreplaydrs} also suppresses false positives by testing multiple thread interleavings.
%after finding data race candidates.
This work compares the immediately resulting program states for differences,
preferring to err on the side of false positives.
RaceFuzzer \cite{racefuzzer} avoids false positives by requiring an actual failure be exhibited,
although it uses random schedule fuzzing rather than stateless model checking.
Note while these techniques can also classify our malloc-recycle candidates as false positives (\sect{\ref{sec:recycle}}),
they require replaying the threads in a new interleaving.
Moreover, \cite{portend} argues that accurate classification may require many re-executions,
%according to many pre- and post-race sequences,
which is tantamount to adding a new state space in \quicksand.
Our proof in \sect{\ref{sec:recycle}} allows eliminating this special case with no additional replay beyond what DPOR already requires.

Portend \cite{portend} is the most closely related work we have found.
% FIXME: "limited"?
Based on single-pass data race reports, it explores a limited state space to classify candidates in a taxonomy of likely severity.
Compared to us, Portend additionally finds non-failing races which nevertheless cause
%suspiciously
different program output, while we depend on directly detecting failures.
It uses symbolic execution to test input nondeterminism as well as schedule nondeterminism,
while we explore the latter only.
However, Portend does not test alternate interleavings {\em in advance} of knowing data races,
which is necessary to expose some bugs (\sect{\ref{sec:eval-falseneg}}).
% XXX: Is this true??
It also assumes the POSIX synchronization API, so cannot verify arbitrary synchronization algorithms such as we do with \mxtest.
Future work could combine the two approaches, using MC to produce new data-race traces for Portend to classify, or using Portend's analysis to inform \quicksand's state space priorities.

% \subsection{Other Concurrency Testing Approaches} % TODO: Well?
%
% blah blah pldi'15 symbiosis DSP

% TODO: talk about data race detectors???
% eg Scalable Race Detection for Android Applications -- uses domain specific heuristics to filter out false positives

% Note that BPOR paper claims that ICB(3+) repeats LOADS of work, and that makes it ok for landslide-ID to repeat work.

% IDK if i should mention it, but OOPSLA 2015, protocol based verification of MPI concurrency paper. Different verification approach entirely; doesn't suffer exponential explosion but limited to programs with no shared state and MPI communication only

% Probably NOT worth a mention: OOPSLA 2015, stateless model checking of event driven applications. Turning timer-driven model on its head and checking single-threaded, but asynch-event-driven programs (i.e. device-like signal handlers)

% TODO: Read OOPSLA 2015 "SATcheck, sat-directed SMC for SC/TSO"


%%%%%%%%%%%%%%%%%%%%%%%%%%%%%%%%%%%%%%%%%%%%%%%%%%%%%%%%%%%%%%%%%%%%%%%%%%%%%%%%

\section{Conclusion}
\label{sec:conclusion}

%We are great. Accept our paper.

We have presented Iterative Deepening and \quicksand, a new technique and tool for automating the choice of preemption points (PPs) during stateless model checking.
% , and \quicksand, a tool which implements this technique tailored to the \landslide~model checker.
\quicksand~incorporates data-race analysis to create new PPs tailored specifically to the program under test,
and automatically finds state spaces that are appropriately sized to complete in a given CPU budget.
%and manages multiple model checker instances to test many different PP subsets in a given CPU budget, even when the full state space of all PPs would be computationally intractable.

We achieve better bug-finding results than either single-pass data-race detection or single-state-space model checking alone,
%We dramatically reduce false-positive data race reports,
%%by verifying the absence of bugs in their associated state spaces
%and find bugs caused by nondeterministic data-races in alternate thread interleavings missed entirely by single-pass analysis.
%Compared to prior work,
%, we find bugs faster in smaller subset state spaces, %% not true, or at least, not by very much
%when the maximal state space is too large to explore
finding new bugs with data-race PPs that could not be exposed by preempting only on synchronization APIs.
By using \numstudence~student projects as our test suite, we also show the potential benefit as a debugging platform in educational settings.
Moreover, when all data-race PPs can be fully tested within the CPU budget, we provide a verification as strong as if every single instruction had been used as a PP.

%Although \landslide~itself is not publicly available due to its dependency on the non-free simulator \simics, \quicksand~is open-source and its interface can be adapted to fit any similar tool.
\quicksand~is open-source and its interface can be adapted to fit any tool similar to \landslide.
We have also posted our evaluation's data set. %and logs.
They are available at:

% TODO CAMREADY: Make data set, at least, public.

{\em github links removed for double-blind review}
%\url{https://github.com/bblum/landslide} % POST-SUBMISSION TODO: put QS in its own repository
%
%\url{https://github.com/bblum/landslide} % POST-SUBMISSION TODO: put final logs somewhere
% TODO CAMREADY: Don't forget to anonymize the names of all your grupos.

%\section{Acknowledgements}

% TODO CAMREADY
%{\em removed for double-blind review}
%Many thanks to Vaishaal Shankar, Haryadi Gunawi, and David A. Eckhardt for generously providing student implementations from Berkeley's, U. of chicago's, and CMU's OS classes respectively.
%Thanks to Wind River for the use of their simulator \simics.
%Thanks to
%Ji\v{r}\'{i} \v{S}im\v{s}a, Joshua Wise, Jean Yang, Brandon Lucia, Haryadi Gunawi, John Wilkes, and
%the anonymous PLDI reviewers for their helpful comments.
%This work was supported in part by
%the U.S. Army Research Office under grant number W911NF0910273.


% TODO: proofs->appeendix
\appendix
\section{Appendix Title}

This is the text of the appendix, if you need one.

\acks

Acknowledgments, if needed.

% We recommend abbrvnat bibliography style.

\bibliographystyle{abbrvnat}
\bibliography{citations}{}

\end{document}
