\section{Implementation}
\label{sec:implementation}

\subsection{Landslide}
\label{sec:landslide}

We chose \landslide~\cite{landslide} as our MC tool due to its ability to trace execution at the granularity of individual instructions and memory accesses, which dynamic data-race detection requires.
\landslide~implements DPOR \cite{dpor},
state space estimation \cite{estimation},
\revisionx{the DJIT+ HB data-race analysis \cite{djit,fasttrack},}
and a hybrid
lockset/Limited HB
data-race analysis \cite{hybriddatarace}.
\revision{It can optionally replace DPOR with ICB \cite{chess-icb},
for which it uses Bounded Partial Order Reduction \cite{bpor} for a similar reduction,
%although we leave incorporating ICB with Iterative Deepening to future work (\sect{\ref{sec:future}})}.
though \quicksand~does not employ this feature (\sect{\ref{sec:future}})}.
It avoids state space cycles
%(e.g. ad-hoc synchronization with {\tt yield} or {\tt xchg} loops)
\revision{(e.g. spin
%or {\tt yield}
loops)}~with
a heuristic similar to Fair-Bounded Search \cite{bpor}.
% joshua wants "segfault" to be "memory access error (i.e., segmentation fault, or bus error)"
Its bug-detection metrics are assertion failure, deadlock, segfault, use-after-free \cite{valgrind}, and (heuristically) infinite loops.
%a heuristic infinite loop check.

\revision{\landslide}~can test both userspace and kernel code (although it is limited to timer nondeterminism),
and runs programs in a full-system hardware simulator \cite{simics}.
\revision{The simulator allows \landslide~to track memory accesses and check for heap errors on uninstrumented binaries,
although for performance, a similar MC under \quicksand~could use compiler instrumentation instead.
It also provides a convenient backtracking mechanism to avoid the need to re-simulate common execution prefixes among many interleavings.}

{\bf Restricting PPs with stack trace predicates.}
When testing a particular module in a large codebase,
the user is likely to be uninterested in PPs arising from other modules.
Rather than preempting indiscriminately on any synchronization call, regardless of the call-site,
prior work introduced Preemption Sealing \cite{sealing} for identifying which call-sites matter.
\landslide~provides this feature through a configuration command, {\tt within\_function}.
%Most MC tools preempt indiscriminately on any synchronization call, regardless of the call-site.
%However, when testing a particular module in a large codebase,
%the user is likely to be uninterested in PPs arising from other modules.
%\landslide~provides the {\tt within\_function} configuration command for a user to identify which call-sites matter most.
Before inserting a PP, \landslide~requires at least one argument to {\tt within\_function} to appear in the current thread's stack trace.
%Similarly, the {\tt without\_function} directive indicates a blacklist,
%serving as the dual of {\tt within\_function}.
The {\tt without\_\allowbreak{}function} directive is the dual of {\tt within\_function}, indicating a blacklist.
We use these to restrict the scope of some tests in our evaluation (\sect{\ref{sec:testsuite}}).
%\revision{similarly to Preemption Sealing from prior work \cite{sealing}}.
%Multiple invocations can be used; later ones take precedence.
%\cite{landslide} provides further detail on this feature.

{\bf Data races in lock implementations.}
Data race tools in prior work \cite{tsan,portend} recognize the implementations of synchronization primitives to avoid spuriously flagging memory accesses %therein. %resulting from the lock implementation itself.
that implement them.
Assuming that the lock implementation is already correct enables more productive data-race analysis on the rest of the codebase,
while the locks themselves can be verified separately \cite{dbug-thesis}.
%Otherwise, %with testing limited to one execution,
%%even if one wishes to test for lock bugs,
%\revision{a single-pass}~data-race analysis would flag every access pair in the lock implementation. %, requiring human attention to verify.
%However, Iterative Deepening removes the need for human attention to verify them. %can automatically verify a large quantity of data-race candidates as benign.
%While most tests in our evaluation assume correct locks,
We included a mutex test in our evaluation
to showcase \quicksand's ability to verify synchronization primitives with data-race PPs
(\sect{\ref{sec:testsuite}}).
To support this test, we extended \landslide~to
%with an option
optionally make its data race analysis consider accesses from {\tt mutex\_lock()} and {\tt mutex\_unlock()}.
%which we use to test mutexes (\sect{\ref{sec:testsuite}}).
%(Accesses from other synchronization functions, such as {\tt cond\_wait()}, would either be included already, or be protected by an internal mutex.)

\subsection{Quicksand}

\quicksand~is an independent program that wraps the execution of several \landslide~MC instances.
The implementation is roughly 3000 lines of C.
%It uses a thread pool to schedule the available state spaces,
%sorting such jobs according to their status among a running queue, pending queue, and suspended queue.
%Jobs are further prioritized by number of PPs, ETA, and whether they include data-race PPs.
The interface with the MC has two parts. %, which any similar MC could implement, has two parts.
First, when starting each job, \quicksand~creates a configuration file declaring which PPs to use,
% can lose this line due to space
plus other MC-specific options such as our modifications to \landslide~for testing mutexes. %mutex-testing mode,
%passed as an argument to \landslide.
Then, a dedicated \quicksand~thread communicates with the MC process via message-passing. %on a FIFO pipe.
%\landslide~messages \quicksand~
The MC messages after testing each interleaving to report updated progress and ETA
and whenever a new data-race candidate or bug is found.
\quicksand~in turn replies whether to resume/suspend (due to too high ETA), or quit (due to timeout).
We suspend jobs simply by making the MC wait on a message-passing reply.
Should \quicksand~later re-schedule a suspended job, it sends a message to continue,
resuming the job where it left off.
%otherwise, we resume it only after time runs out, causing it to exit.
