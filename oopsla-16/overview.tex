\section{Background}
\label{sec:overview}

We review the background in stateless model checking and data-race analysis using the example program in Figure~\ref{fig:example}.

\subsection{Stateless Model Checking}
\label{sec:overview-sssmc}

Stateless model checking \cite{verisoft} is a testing technique for systematically exploring the possible thread interleavings of a concurrent program.
A stateless model checker executes the program repeatedly, each time according to a new thread interleaving, until the state space (or the CPU budget) is exhausted.
During each execution, it forces threads to execute serially, thereby confining the program's nondeterminism to controlled thread switches.

Rather than identifying suspicious conditions which may include false alarms, the approach of many static analyses \cite{racerx,coverity},
stateless model checkers focus on concrete observed failures such as assertions, deadlocks, segfaults, and use-after-free.
For example, data races \revision{do not always lead to failures}, but represent suspicious violations of common locking discipline.
Although C++ defines all data races as undefined behaviour \cite{boehm-memorymodels},
this work focuses on reporting races to the user only when accompanied by direct evidence of a failure.

\begin{figure}[t]
	\small
\begin{tabular}{rll}
	& \multicolumn{2}{c}{\texttt{int x = 0; mutex\_t mx;}} \\
	& {\bf Thread 1} & {\bf Thread 2} \\
	1 & \texttt{\hilight{orange}{mutex\_lock(\&mx);}} & \\
	2 & \texttt{int tmp = x;} &\\
	3 & & \texttt{atomic\_xadd(\&x, 1);} \\
	4 & & \texttt{\hilight{olivegreen}{yield();}} \\
	5 & & \texttt{atomic\_xadd(\&x, 1);} \\
	6 & \texttt{x = tmp + 1;} & \\
	7 & \texttt{\hilight{commentblue}{mutex\_unlock(\&mx);}} & \\
	8 & & \texttt{assert(x >= 2);} \\
	%1 & \texttt{\hilight{brickred}{x->foo = ...;}} & \\
	%2 & \texttt{\hilight{olivegreen}{free}(x);} \\
	%3 & & \texttt{\hilight{commentblue}{// x's memory recycled}} \\
	%4 & & \texttt{y~=~\hilight{olivegreen}{malloc}(sizeof *y);} \\
	%5 & & \texttt{\hilight{commentblue}{// ...initialize...}}\\
	%6 & & \texttt{publish(y);} \\
	%7 & & \texttt{\hilight{brickred}{y->bar = ...;}} \\
\end{tabular}
\caption{Example program with a data-race bug. In this interleaving, the assertion on line 8 will fail. Two data-race preemptions are required to expose the bug: one just before thread 1's line 6, and one just before thread 2's line 8.}
\label{fig:example}
\end{figure}

%\begin{figure}[t]
%	\begin{tabular}{c}
%	\includegraphics[width=0.42\textwidth]{tree-maximal-only.pdf}
%	\end{tabular}
%	\caption{\revisionz{The state space of possible interleavings is represented as a tree,
%	which stateless model checkers test exhaustively.
%	This state space arises when preempting on synchronization APIs used by the program in Figure~\ref{fig:example};
%	other state spaces are possible with more, or fewer, preemption points.}}
%	\label{fig:tree-maximal}
%\end{figure}

\begin{figure}[t]
	\includegraphics[width=0.48\textwidth]{trees.pdf}
	\caption{Iterative Deepening example.
		The minimal state space (a) includes only voluntary thread switches, such as {\tt yield()}. %or {\tt cond\_wait()}.
		Multiple further tests can be run: preempting on calls to {\tt mutex\_lock} alone (b), {\tt mutex\_unlock} alone (c), or both together (d).
Each option increases the state space size unpredictably, so multiple state spaces should be tested in parallel.
Estimation techniques~\cite{estimation} inform which state spaces to prioritize.
}
	\label{fig:id}
\end{figure}

The checker defines the granularity of thread interleavings by the {\em preemption points} it uses to switch threads.
Most model checkers \cite{chess,dbug-ssv} choose synchronization and thread library API boundaries for these points;
in our example program, these would be lines 1, 4, and 7.
Figure~\ref{fig:id} shows several possible resulting state spaces.
The approach of prior work is to enable all preemption points simultaneously, i.e., to test only the state space denoted (d).

To mitigate the exponential explosion,
Dynamic Partial Order Reduction \cite{dpor} identifies equivalent execution sequences according to Mazurkiewicz trace theory \cite{mazurkiewicz},
and tests at least one execution from each equivalence class.
Intuitively, if two thread transitions between preemption points do not conflict on any shared resource access, reordering them produces an equivalent interleaving, i.e., the same program behaviour.
Iterative Context Bounding \cite{chess-icb}, another popular technique, heuristically reorders the search to prioritize interleavings with fewer preemptions first, according to the insight that most bugs require few preemptions to uncover.
Nevertheless, state spaces are still exponentially-sized in the number of conflicting transitions.

This motivates {\em Iterative Deepening}, our new technique for heuristically adjusting the preemption points at runtime.
Rather than committing to one state space with every available preemption point enabled,
we will search among different {\em subsets} of the points.
%for example, ``preempt on all calls to {\tt mutex\_lock} but not on {\tt mutex\_unlock}''.
Hence, we will test all the state spaces shown in Figure~\ref{fig:id} in parallel,
and decide on-the-fly whether to pursue each test, or to defer it in favour of others.

\revisionx{
{\bf Recent advances.}
Like many prior checkers \cite{chess,dbug-ssv,inspect,portend}, ours implements DPOR for sequentially-consistent hardware.
In the worst case, these tools may all suffer false negatives as they miss weak-memory-only bugs.
Recently, Zhang et al. \cite{tsopso} introduced a new formalism with which DPOR can control weak memory nondeterminism,
such as reordering store buffers.
Iterative Deepening could use this new technique,
%We expect this technique would apply directly with Iterative Deepening,
provided a model checker which implements such reorderings (not yet available to us).
%We believe this assumption is reasonable for the scope of this paper because the main property we require of DPOR is that
%A recent paper extended DPOR to support weaker memory models \cite{tsopso}, so our proofs could likewise be extended with that version of DPOR in future work.

Maximal Causality Reduction (MCR) \cite{mcr}, a reduction algorithm which may replace DPOR,
has also recently shown promising performance improvements over prior model checkers.
We expect Iterative Deepening to be compatible with MCR,
and look forward to evaluating the combination when an MCR implementation becomes widely available.
}

\subsection{Data Race Detection}
\label{sec:overview-dr}

\begin{figure}[t]
	\small
\begin{tabular}{c}
	\revision{
\begin{tabular}{rll}
	& \multicolumn{2}{c}{\texttt{int x = 0; bool y = false; mutex\_t mx;}} \\
	& {\bf Thread 1} & {\bf Thread 2} \\
	1 & \texttt{\hilight{brickred}{x++;}~// A1} & \\
	2 & \texttt{mutex\_lock(\&mx);} & \\
	3 & \texttt{mutex\_unlock(\&mx);} & \\
	4 & & \texttt{mutex\_lock(\&mx);} \\
	5 & & \texttt{mutex\_unlock(\&mx);} \\
	6 & & \texttt{\hilight{brickred}{x++;}~// A2} \\
\end{tabular}
}
\\
\revision{\normalsize (a) True potential data race.}
\\
\\
\revision{
\begin{tabular}{rll}
	%& \multicolumn{2}{c}{\texttt{int x = 0; bool y = false; mutex\_t mx;}} \\
	%& {\bf Thread 1} & {\bf Thread 2} \\
	1 & \texttt{\hilight{brickred}{x++;}~// B1} & \\
	2 & \texttt{mutex\_lock(\&mx);} & \\
	3 & \texttt{y = true;} & \\
	4 & \texttt{mutex\_unlock(\&mx);} & \\
	5 & & \texttt{mutex\_lock(\&mx);} \\
	6 & & \texttt{bool tmp = y;} \\
	7 & & \texttt{mutex\_unlock(\&mx);} \\
	8 & & \texttt{if (tmp) \hilight{brickred}{x++;}~// B2} \\
	%8 & & \texttt{if (tmp)} \\
	%9 & & \texttt{~~~~\hilight{brickred}{x++;}~// B2} \\
\end{tabular}
}
\\
\revision{\normalsize (b) No data race in any interleaving.}
\end{tabular}
\caption{\revision{Data-race analyses may be prone to either {\em false negatives} or {\em false positives}.
Applying HB to program (a) will miss the potential race possible between A1/A2 in an alternate interleaving,
while using Limited HB on (b) will produce a false alarm on B1/B2.}}
\label{fig:hb-example}
\end{figure}

Data race analysis \cite{eraser} identifies pairs of unsynchronized memory accesses between threads.
Two instructions are said to race if
they both access the same memory address,
at least one is a write,
the threads do not hold the same lock,
and no synchronization enforces an order on the thread transitions \revision{(the {\em Happens-Before} relation)}.
In Figure~\ref{fig:example}, lines 3 and 5 each race with 2 and 6, and line 6 races with 8.

\revision{
A data race analysis may be either {\em static} (inspecting source code) \cite{racerx} or {\em dynamic} (tracking individual accesses arising at run-time) \cite{tsan}.
This paper focuses exclusively on dynamic analysis,
so although our example refers to numbered source lines for ease of explanation,
in practice we are actually classifying the individual memory access events corresponding to those lines during execution.
}

%\revision{Among these, only the races involving lines 3 and 5 are relevant to the possible assertion failure;
%permuting the order of line 8 alone
%% relative to 2 and 6
%will not change the program's behaviour.}

Though state-of-the-art model checkers preempt only on synchronization events,
many serious concurrency bugs are caused by data races leading to corrupted shared state.
Figure~\ref{fig:example}'s buggy interleaving is possible only with {\em data-race preemption points}:
preempting just before an instruction identified as part of a data race.
None of the state spaces in Figure~\ref{fig:id} contain this interleaving,
as none of the mutex/yield preemptions split lines 2 and 6 across different transitions.

	{\bf Variants of Happens-Before.}
	Most prior work focuses on {\em Happens-Before} (HB) \cite{lamport-clocks,djit,fasttrack} as the order relation between accesses.
\cite{predictive-dr} and \cite{hybriddatarace} identify a problem with this approach:
it cannot identify access pairs separated by an unrelated lock operation which could race in an alternate interleaving.
%as shown in the example program in Figure~\ref{fig:hb-example}(a).
\revisionx{Figure~\ref{fig:hb-example}(a) shows a contrived example program in which HB masks the potential race.}
We call such unreported access pairs {\em false negatives}.
However, consider the similar program in Figure~\ref{fig:hb-example}(b),
in which the access pair ceases to exist in the alternate interleaving.
O'Callahan et al. \cite{hybriddatarace} introduced the {\em Limited HB} relation,
which will report such potential races
by considering only blocking operations like {\tt cond\_wait} to enforce the order.
Limited HB will report all potential races, avoiding \revisionx{many} false negatives \cite{tsan},
but at the cost of necessarily reporting some such {\em false positives}.

Finally, the {\em Causally-Precedes} relation \cite{predictive-dr} %strikes a middle ground,
extends HB to additionally report a subset of potential races while soundly avoiding false positives.
It tracks conflicting accesses in intervening
critical sections to determine whether lock events are unrelated to a potential race.
Causally-Precedes will identify the potential race in Figure~\ref{fig:hb-example}(a), as the two critical sections do not conflict,
although it can still miss true potential races in other cases.

\revisionx{Being dynamic analyses, both HB and Limited HB may suffer false negatives when a racy access pair is not executed at all in a specific interleaving.
Limited HB offers the advantage of identifying a potential race as long as the access pair is observed under any interleaving,
rather than requiring the accesses to be adjacent in time, as HB would.}
While stand-alone data-race analyses must avoid inundating the user with false alarms \cite{racerx},
%However,
we incorporate data-race analysis in an internal feedback loop,
using model checking to automatically test each potential race
and report only directly observed failures to the user.
%Our convergence theorem requires identifying all potential data-races to provide total verification (\sect{\ref{sec:totalverif}})
%Hence, we employ Limited HB in this paper},
%accepting some overhead from false positives for the sake of total verification (\sect{\ref{sec:totalverif}}).
\revisionx{Hence, we can accept some overhead from Limited HB's false positives for the sake of finding data-race candidates more quickly.
In \sect{\ref{sec:eval}} we will evaluate how HB and Limited HB each influence \quicksand's bug-finding and verification speed.
}

% Old explanation.
%Prior work \cite{hybriddatarace} distinguishes between {\em pure happens-before} \cite{lamport-clocks},
%in which accesses with a lock release/acquire in between are not considered concurrent,
%and {\em limited happens-before},
%in which only blocking operations like {\tt cond\_wait} or {\tt join} enforce the order.
%Compared to the former, the latter may report false positives, but avoids false negatives \cite{tsan},
%in which a potential racing access pair is not reported, even though it could lead to a bug.
%We use the limited happens-before analysis, accepting some overhead from false positives for the sake of total verification (\sect{\ref{sec:totalverif}}).

\revisionx{
{\bf Philosophy of bugs.}
% two camps: all races are bugs; citations:
% and, races must be classified; citations: replay analysis, portend,
While there is a vast body of work on how to detect data races to begin with,
judging data races once found is a matter of philosophical debate unto itself.
Some recent tools classify races depending on how they impact program behaviour \cite{recordreplaydrs,portend}, overlooking {\em benign} races in search of more dangerous ones.
%In particular,
\cite{racerx} acknowledges that a program may have too many races for a user to worry about,
so bug reports must be prioritized by severity of effects. %how likely they may cause failures or security vulnerabilities.
%
However, other prior work argues that data races are always bugs \cite{miscompile-benign,data-races-are-evil},
largely due to the possibility of compiler or hardware reordering of racy accesses.
%
We take the former camp:
we consider a data race a bug only when it results in a visible failure state (e.g., crash or deadlock)\footnote{
C++ declares any race between two non-atomic locations to be undefined behaviour \cite{cpp-foundation}.
From a C++ perspective, we assume all concurrent accesses are implemented by {\tt std::atomic} loads and stores.}.
We bypass concerns of compiler reordering by checking programs at the executable level;
for a study of hardware reordering in the context of DPOR, we refer the reader to \cite{tsopso}.
}

\subsection{Terminology}

For the rest of the paper, we will abbreviate {\em preemption point} (PP),
\revision{{\em happens-before} (HB)},
{\em model checking} (MC),
{\em single-state-space model checking} (SSS-MC), % (i.e., the approach of prior work),
{\em Dynamic Partial Order Reduction} (DPOR),
{\em Iterative Context Bounding} (ICB),
and {\em state space estimate} (ETA).

SSS-MC indicates the approach of prior tools:
the set of PPs is fixed in advance, and the tool commits to testing every interleaving available with those PPs.
Many techniques can skip equivalent interleavings or order the search to uncover bugs faster \cite{dpor,demeter,chess-icb,gambit,smc-empirical-study},
but new PPs cannot be added, nor ineffective ones removed, by any dynamic analysis.

We distinguish between data-race {\em candidates} and data-race {\em bugs}.
\revision{We refer to racing (or potentially-racing) memory access pairs as}~{\em data race candidates}.
Should %a future interleaving,
preempting during such accesses
lead to \revision{an observable}~failure, then we report a {\em data-race bug}.
Otherwise, if the access pair can be reordered,
%\revision{to be simultaneous},
but does not produce a failure under any interleaving, it is a {\em benign data race} (with respect to the test input).
If they cannot be reordered at all, due to some other communication \revision{such as in Figure~\ref{fig:hb-example}(b)}, it is a {\em false positive}.

We also identify the {\em minimal} and {\em maximal state space} for each test.
The {\em minimal state space} includes only thread switches arising from no preemptions (Figure~\ref{fig:id}(a)).
The {\em maximal state space} is the one tested by SSS-MC: all statically-available PPs are enabled (Figure~\ref{fig:id}(d)).
%However, should new data-race PPs be added during a test, the new maximal state space will be the one including those as well.
%% don't say this ^ -- because when you add dr pps, they add in pairs and there would be multiple maximals, at least until they each get explored and subseuqnetly add a pp of the other of the pair.
