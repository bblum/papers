\section{Related Work}
\label{sec:related}

\subsection{Stateless Model Checking}

We build upon many established model-checking techniques, dating back
% of course
to Verisoft, the original C model checker \cite{verisoft}.
%, and Eraser, the original data race detector \cite{eraser}.
%Our model checker
%\landslide~\cite{landslide} itself implements many techniques from prior work (\sect{\ref{sec:landslide}}).
%itself implements DPOR \cite{dpor},
%state space estimation \cite{estimation},
%and data-race detection \cite{eraser}.
We compare related tools by their treatment of shared-memory thread communication.

{\bf Synchronization events only.} CHESS \cite{chess} and dBug \cite{dbug-ssv} instrument the thread library API, and can preempt programs only during calls to this API.
Hence, they will miss any bugs that require interleaving threads at instruction granularity during a data race.
%CHESS provides a data-race analysis to report any such violations of its concurrency model to the user, but does not incorporate data-race candidates as PPs in future tests.
\revision{\cite{chess} mentions the ability of CHESS to add PPs using a single-pass data-race analysis,
% FIXME... idk, "discusses (without explicit evaluation)..., but, as our evaluation shows..."?
but does not discuss either the increase in bug-finding capacity or the soundness properties.
Our convergence theorem (\sect{\ref{sec:soundness}}) is a natural extension of \cite{chess-icb}'s Theorem 3,
which alone provides soundness only for race-free programs.}

{\bf Message-passing.} Other stateless model checkers, such as SAMC \cite{samc}, MaceMC \cite{macemc}, MoDist \cite{modist}, ETA \cite{dbug-retreat}, and Concuerror \cite{optimal-dpor},
limit thread communication to a message-passing API to more effectively test distributed systems.
This eliminates the need for data-race analysis, but restricts the class of programs that can be tested.
Nevertheless, Iterative Deepening is applicable to these tools.

{\bf Preempting at instruction granularity} is a prerequisite for using data-race PPs.
However, the resulting state space explosion demands that any such tool either
choose a small subset of instructions to consider as PPs
or be limited to very small test programs.
%However, every such prior tool we know of has serious drawbacks.
SKI \cite{ski} approaches kernel code by statically choosing a random set of instructions in advance, %offsets from the start of the test,
which is perhaps more similar to
%stress testing or
schedule fuzzing \cite{randomized-scheduler} than to exhaustive state space exploration.
%
SPIN \cite{spin} specializes in verifying synchronization primitive implementations such as RCU \cite{rcu}, which is similar to our \mxtest~experiment,
although it requires code to be written in the PROMELA language.
%However, SPIN is stateful rather than stateless, and explicitly storing visited program states rather than using DPOR limits the size of programs that can be practically tested.
%
Inspect \cite{inspect} instruments source code by inserting wrapper calls around all accesses to potentially-shared data.
It identifies such instructions in advance with an over-approximating static alias analysis,
while \landslide~\cite{landslide} traces the memory locations of accesses at runtime.
Both SPIN and Inspect fix their set of PPs in advance, so could be
%combined with \quicksand~
extended with Iterative Deepening in future work.
%so cannot check implementations directly.

{\bf Other techniques.} Various improvements or alternatives to DPOR have been developed, such as Dynamic Interface Reduction \cite{demeter}, Maximal Causality Reduction \cite{mcr},
%DPOR for relaxed memory models \cite{tsopso},
SAT-directed MC \cite{satcheck},
\revision{and R4 for event-driven applications \cite{r4}}.
These are all compatible with our technique.
Recent work \cite{tsopso} has extended DPOR for relaxed memory models \cite{memory-consistency-models},
which we do not yet account for in our proofs (\sect{\ref{sec:soundness}}).
Parrot \cite{parrot} combines MC with a partially-determinizing runtime for further reduction, but still, fewer than half the non-trivial state spaces in their evaluation could be completed.
%providing a strong argument for \quicksand.
Finally, Iterative Context Bounding (ICB) \cite{chess-icb} is most similar to our work,
as both approaches provide a partial verification
%on some subset of interleavings
when full completion is intractable (\sect{\ref{sec:future}}).
However, ICB is limited to a fixed set of PPs, and to our knowledge no algorithm has been proposed to dynamically add data-race PPs during a test with ICB.

\subsection{Data Race Detection}
\label{sec:related-dr}

%Too many related projects to list have made contributions to the
Many advances have been made on the false-positive data race problem since it was first introduced in \cite{eraser}.
\cite{hybriddatarace} and \cite{tsan} combine the lockset and happens-before analyses into a hybrid technique, which we employ.
\revision{Several tools \cite{droidracer,cafa,unnamed-android} have recently emerged to target Android applications},
%DroidRacer \cite{droidracer} and CAFA \cite{cafa} target
%%event-driven
%Android applications,
using domain-specific heuristics (orthogonal to our method) to reduce false positives. % cut for space?
Like \landslide, DataCollider \cite{datacollider} offers data-race techniques for kernel code.
IFRit \cite{ifrit}
improves performance using an interference analysis,
which could allow future work to avoid tracing every memory access.
%although in our context, we cannot admit false negatives (\sect{\ref{sec:totalverif}}).
% No, IDGAF about pure happens before.
%FastTrack \cite{fasttrack} optimizes the performance of pure happens-

Closer to our work, replay analysis \cite{recordreplaydrs} also suppresses false positives by testing multiple thread interleavings.
%after finding data race candidates.
This work compares the program states immediately after the access pair for differences,
preferring to err on the side of false positives.
RaceFuzzer \cite{racefuzzer} avoids false positives by requiring an actual failure be exhibited, as we do,
although it uses random schedule fuzzing rather than stateless MC.
While this technique can classify malloc-recycle candidates as false positives (\sect{\ref{sec:recycle}}),
they require replaying the threads in a new interleaving.
Moreover, \cite{portend} argues that accurate classification may require many re-executions,
%according to many pre- and post-race sequences,
which is tantamount to adding a new state space in \quicksand.
Our proof in \sect{\ref{sec:recycle}} allows us to eliminate this special case with no additional replay beyond what DPOR already requires.

Portend \cite{portend} is the most closely related work we have found.
Based on reports from single-pass data-race analysis, it tests alternate executions to classify candidates in a taxonomy of likely severity.
It uses symbolic execution to test input nondeterminism as well as schedule nondeterminism,
%while we explore the latter only.
and additionally reports non-failing races which nevertheless cause
%suspiciously
different program output. %, which is orthogonal to our technique.
However, Portend does not test alternate interleavings {\em in advance} of knowing any specific data races,
which is necessary to find certain bugs (\sect{\ref{sec:eval-falseneg}}) or to provide full verification (\sect{\ref{sec:totalverif}}).
% Not really true.
%It also assumes the POSIX synchronization API, so cannot verify arbitrary synchronization algorithms such as we do with \mxtest.
Future work could combine the two approaches, using MC to produce new data-race traces for Portend to classify, or using Portend's analysis to inform \quicksand's heuristic priorities.

% \subsection{Other Concurrency Testing Approaches}
%
% blah blah pldi'15 symbiosis DSP

% Note that BPOR paper claims that ICB(3+) repeats LOADS of work, and that makes it ok for landslide-ID to repeat work.

% IDK if i should mention it, but OOPSLA 2015, protocol based verification of MPI concurrency paper. Different verification approach entirely; doesn't suffer exponential explosion but limited to programs with no shared state and MPI communication only

% Probably NOT worth a mention: OOPSLA 2015, stateless model checking of event driven applications. Turning timer-driven model on its head and checking single-threaded, but asynch-event-driven programs (i.e. device-like signal handlers)
