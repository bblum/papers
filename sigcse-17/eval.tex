\section{Evaluation}
\label{sec:eval}

We made \landslide available to students for the thread library project
during the Spring '15, Fall '15, and Spring '16 semesters of \fourten.
We introduced stateless model checking in class, %and gave a demo during a special lecture,
and required students to pass the thread library hurdle (\sect{\ref{sec:grading}}) to use \landslide.
Of 90 two-student groups in those semesters, 47 tested their thread libraries with \landslide.

Whenever a student ran \landslide,
it recorded which options were used (test program and time limit),
a snapshot of the student's code,
and the result of the test (completed, timed out, bug found, or ctrl-C'ed).
%
We manually analyzed these snapshots to count how many bugs each group found
(not double-counting the same bug found repeatedly)
and how many of those were fixed before submission
(i.e., the students changed their code and re-ran the same test successfully).
%
We also distinguished {\em deterministic} bugs from concurrency bugs
by identifying whether \landslide needed to test multiple interleavings before finding the bug\footnote{
	For example, many students found deterministic use-after-free bugs,
	previously overlooked because the unit tests do not provide heap checking like \landslide does.
}.

\begin{table}[t]
	\begin{center}
	\begin{tabular}{r|cc|cc}
	%det bugs found	det bugs fixed	races fixed	races found
	& \multicolumn{4}{c}{Number of groups} \\
	Number & \multicolumn{2}{c|}{Deterministic bugs} & \multicolumn{2}{c}{Concurrency bugs} \\
	of bugs	& found & fixed & found & fixed \\
	\hline
	0	& 27	& 32	& 15	& 23	\\
	1	& 6	& 5	& 7	& 11	\\
	2	& 9	& 6	& 12	& 8	\\
	3	& 2	& 1	& 7	& 2	\\
	4	& 1	& 1	& 3	& 1	\\
	5	& 2	& 2	& 2	& 1	\\
	11	& 	& 	& 1	& 1	\\
	\hline
	%Total groups
	%	& 47	& 47	& 47	& 47	\\
	Total bugs
		& 44	& 34	& 85	& 53	\\
	\end{tabular}
	\end{center}
	\caption{Summary of how many bugs were found and/or fixed by how many groups.
	Each column counts how many groups found/fixed the number of bugs in each row;
	for instance, two groups found 5 concurrency bugs, one of which fixed all 5.}
	\label{tab:this-table-sucks-but-it's-the-best-i-got}
\end{table}

Table \ref{tab:this-table-sucks-but-it's-the-best-i-got} shows our results.
In total, \landslide found 44 deterministic bugs and 85 concurrency bugs.
\landslide found at least one concurrency bug for 32 groups (68\%),
24 of whom (51\%) were able to fix at least one,
verifying their update with a succesful re-run of the same test.

\begin{figure}[t]
	\includegraphics[width=0.48\textwidth]{p2-distribution-logos.pdf}
	\caption{Distribution of thread library grades.}
	\label{fig:p2-distribution}
\end{figure}
We also collected the overall thread library grades to evaluate whether \landslide helps students achieve more during the project.
Figure~\ref{fig:p2-distribution} shows the distribution of these grades.
Compared to students from the same semesters who opted not to use \landslide,
those who did scored on average 3\% higher. % that's rough, buddy.
%
%However, to account for the possible bias for opting-out students being more at the bottom of the class
%on account of not enough free time to volunteer for \landslide,
However, the latter distribution is potentially biased towards struggling students
who lack enough free time to volunteer for \landslide.
%To account for this,
Thus, we also checked grades from prior semesters when \landslide was not available.
%This distribution is indistiguishable from the experimental group,
Although the minimum grade among the experimental group is 17\% higher than the minimum from prior semesters,
the middle 50\% of the distributions are quite similar,
suggesting that \landslide's measurable benefit is largely offset by the time investment required.
%
%In future work, we will streamline the user experience to help students fix as many concurrency bugs with a smaller time investment.
