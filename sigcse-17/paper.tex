\documentclass{sig-alternate-05-2015}

\usepackage{xspace}
\usepackage{color}
\usepackage{graphicx}
\usepackage{listings}

\lstset{language=C,
        frame=none,%single,
        basicstyle=\ttfamily,
        keywordstyle=\bfseries,
        commentstyle=\em,
        tabsize=2,
        showstringspaces=false,
        flexiblecolumns=false,
        mathescape=true,
        escapeinside={\#}{\#},
        % numbers=left, numberstyle=\tiny, numbersep=5pt %, stepnumber=2,
        numbersep=5pt,
        % literate={->}{{$\rightarrow$}}1 {<->}{{$\leftarrow$}}1
}
\newcommand\x[1]{\lstinline{#1}}


\begin{document}

\setcopyright{acmcopyright}
%\setcopyright{acmlicensed}
%\setcopyright{rightsretained}
%\setcopyright{usgov}
%\setcopyright{usgovmixed}
%\setcopyright{cagov}
%\setcopyright{cagovmixed}


% TODO: ???
%\doi{10.475/123_4}
%\isbn{123-4567-24-567/08/06}
%\acmPrice{\$15.00}
%\crdata{0-12345-67-8/90/01}  % Allows default copyright data (0-89791-88-6/97/05) to be over-ridden - IF NEED BE.

\conferenceinfo{SIGCSE '17}{March 8th - 11th, Seattle, Washington, USA}
\CopyrightYear{2017}

% TODO camready
\newcommand\carnegiemellon{{\em Double-Blind University}\xspace}
\newcommand\cmu{DBU\xspace}
\newcommand\pebbles{\cmu-OS\xspace}
\newcommand\landslide{Landslide\xspace}
\newcommand\fourten{CS999\xspace}
%\newcommand\carnegiemellon{{\em Carnegie Mellon University}\xspace}
%\newcommand\cmu{CMU\xspace}
%\newcommand\pebbles{Pebbles\xspace}
%\newcommand\landslide{Landslide\xspace}
%\newcommand\fourten{15-410}

\title{Teaching Low-Level Concurrency with \pebbles and \landslide}

\numberofauthors{3}

% TODO camready
\author{
	\em double-blind submission
%	\alignauthor
%	Ben Blum \\
%	\affaddr{Carnegie Mellon University} \\
%	\affaddr{5000 Forbes Avenue} \\
%	\affaddr{Pittsburgh, PA, USA} \\
%	\email{bblum@cs.cmu.edu}
%	\alignauthor
%	David A. Eckhardt \\
%	\affaddr{Carnegie Mellon University} \\
%	\affaddr{5000 Forbes Avenue} \\
%	\affaddr{Pittsburgh, PA, USA} \\
%	\email{davide@cs.cmu.edu}
%	\alignauthor
%	Garth Gibson \\
%	\affaddr{Carnegie Mellon University} \\
%	\affaddr{5000 Forbes Avenue} \\
%	\affaddr{Pittsburgh, PA, USA} \\
%	\email{garth@cs.cmu.edu}
}


\maketitle
\begin{abstract}

In \carnegiemellon's undergraduate operating systems course
students learn how to write correct and principled concurrent code through
a series of low-level programming projects.
The curriculum includes implementing hardware device drivers and a user-level
threading library, and culminates in a six-week long project in which students
build from the ground up a fully-preemptible, UNIX-like kernel that can run
on real hardware.

We aspire to expand the scope of these projects in response to
typical hardware platforms'
increasing complexity,
especially massive multi-core.
To that end, we wish to reduce the time students spend
debugging concurrency bugs.
We present Landslide, a
%systematic testing
stateless model checker which
offers a more principled approach than stress testing.
% edupar version
%We report on our preliminary experience working with students using Landslide to find previously-overlooked bugs in their own code, and discuss future methods for making Landslide more accessible to struggling students.
% sigcse version
We report on our experience giving students Landslide during the thread library
%and kernel projects.
project,
%Although kernel-mode Landslide requires manual instrumentation, testing thread libraries is fully automatic.
% TODO: verify these, obv. ;)
showing that it allows them to find and fix previously-overlooked bugs.
Further, %even struggling students can take advantage of it,
Landslide's users achieve higher concurrency scores on average in subsequent projects,
% FIXME awkward phrasing
suggesting that debugging with a model checker produces better student retention of concurrency principles.

\end{abstract}


% TODO, do dis
% The code below should be generated by the tool at
% http://dl.acm.org/ccs.cfm
% Please copy and paste the code instead of the example below.
\begin{CCSXML}
<ccs2012>
 <concept>
  <concept_id>10010520.10010553.10010562</concept_id>
  <concept_desc>Computer systems organization~Embedded systems</concept_desc>
  <concept_significance>500</concept_significance>
 </concept>
 <concept>
  <concept_id>10010520.10010575.10010755</concept_id>
  <concept_desc>Computer systems organization~Redundancy</concept_desc>
  <concept_significance>300</concept_significance>
 </concept>
 <concept>
  <concept_id>10010520.10010553.10010554</concept_id>
  <concept_desc>Computer systems organization~Robotics</concept_desc>
  <concept_significance>100</concept_significance>
 </concept>
 <concept>
  <concept_id>10003033.10003083.10003095</concept_id>
  <concept_desc>Networks~Network reliability</concept_desc>
  <concept_significance>100</concept_significance>
 </concept>
</ccs2012>
\end{CCSXML}

\ccsdesc[500]{Computer systems organization~Embedded systems}
\ccsdesc[300]{Computer systems organization~Redundancy}
\ccsdesc{Computer systems organization~Robotics}
\ccsdesc[100]{Networks~Network reliability}

% End generated code

\printccsdesc

\keywords{concurrency, debugging, model checking, computer science education}

\section{Introduction}

% blah blah trite opening sentence
%As parallelism becomes ever more important for achieving high performance in modern-day programs,
%so too do advanced concurrency testing techniques become important for verifying the correctness of those programs.
Concurrency bugs are notoriously hard to find and reproduce because they only appear in specific thread interleavings, which arise at random during normal program execution.
{\em Stateless model checking} \cite{verisoft} offers a method for finding such bugs,
or verifying their absence,
%by systematically executing a program along as many distinct interleavings as possible,
by forcing a program to execute each distinct interleaving,
capturing and controlling this nondeterminism using a finite state space.
Unfortunately, these state spaces explode exponentially in the size of the input program.
Reduction techniques such as Dynamic Partial Order Reduction \cite{dpor} and Maximal Causality Reduction \cite{mcr} expand the limits of feasible test completion,
and search ordering strategies such as Iterative Context Bounding \cite{chess-icb} allow bugs to be found sooner in a given exploration should they exist.

% Can I even make a claim this broad to begin with?
However, all stateless model checkers to date are bound by a fixed set of {\em preemption points}: code locations that define the granularity at which threads interleave.
% TODO: You repeat the following 1.5 sentence in the related work section; maybe you can eliminate this redundancy to save space?
For example, \textsc{CHESS} \cite{chess} preempts only on synchronization operations and library calls, which can miss lock-free shared memory races.
It provides an additional data-race analysis to report any violations of this model;
however, data-race analyses are prone to report false positives
%and benign races
which require annotations or imprecise heuristics to suppress \cite{racerx,tsan,datacollider}.
%
On the other hand, SPIN \cite{spin}
is able to preempt threads around any shared memory access. Such fine granularity would automatically check if each data race is a real bug, but makes full state space completion intractable for even modestly-sized tests.
%
Configuring a model checker is a tradeoff between schedule coverage and feasibility of completion.
This work shows how to avoid making that tradeoff decision in advance.

% TODO: ttuttle says this is too much of a jump -- that exactly what "subsets" means is not well defined (i.e, lock vs unlock, while NOT addressing the within/without_function problem)

We present \quicksand, a framework for {\em Iterative Deepening} of preemption points during stateless model checking.
Named after the analogous technique in chess AI \cite{iterative-deepening-chess-ai}, our approach likewise makes progressively deeper searches of the state space until a given CPU budget is exhausted.
Rather than attempting to search a single state space with every available preemption point enabled (e.g., preempting on every pthread API call),
\quicksand~tests many different state spaces corresponding to subsets of those points, managing a model checker instance to explore each one.
It estimates the size of each state space to decide when long-running instances should be suspended, and dynamically generates new state spaces based on data race analysis.
%In fact, if given enough time to fully test all discovered data-race preemption points,
%Iterative Deepening provides a full verification of all thread schedules that could arise from preempting anywhere.
In fact, Iterative Deepening is fully general:
we prove that if it completes all state spaces resulting from data-race preemption points,
that serves as a total verification of all possible thread interleavings of the given test program.

We evaluate \quicksand~by testing \numstudence~student thread libraries and kernels from the undergraduate OS classes at Carnegie Mellon, Berkeley, and the University of Chicago.
\quicksand~finds more bugs than the conventional stateless model checking approach given the same CPU budget,
% joshua wants me to say "conventional approachES" here
and furthermore, adding data-race preemption points quickly exposes bugs missed by even the ``maximal'' state space of the conventional approach.

This paper's contributions are as follows:
\begin{enumerate}
	\item Iterative Deepening, a new technique for combining data-race analysis with stateless model checking, and \quicksand, an open-source implementation of the technique;
	\item A proof of convergence, which shows that should it be possible in the given CPU budget,
		fully testing every discovered data-race preemption point is equivalent to testing all possible thread schedules;
	\item A new tactic for eliminating one class of false-positive data races,
		which cannot soundly be used in a single-pass analysis,
		but which we prove correct when used with Iterative Deepening;
		%, unsound in single-pass analysis but which we prove sound when used with Iterative Deepening;
	%\item Techniques for detecting and flattening cyclic state spaces resulting from ad-hoc while-loop synchronization % TODO: is there actually room for this in the paper?
	\item A large evaluation in which \quicksand~compares favorably to stand-alone data-race detection and stateless model checking approaches.
\end{enumerate}

The remainder of the paper is organized as follows.
\sect{\ref{sec:design}} discusses the background and design of Iterative Deepening, including our proof of convergence,
\sect{\ref{sec:implementation}} explains \quicksand's approach to implementing it, including our new false-positive data-race tactic,
\sect{\ref{sec:eval}} presents our evaluation,
\sect{\ref{sec:future}} discusses limitations and future work,
\sect{\ref{sec:related}} surveys the related work,
and \sect{\ref{sec:conclusion}} concludes.


\section{The {\secit Body} of The Paper}
% TODO
TODO body

\section{Conclusions}
% TODO
TODO conclusion

% TODO camready
% \section{Acknowledgments}
% Many people have contributed to the development of
% 15-410 over the past decade.
% Stephen Muckle, currently at Qualcomm,
% led the switch to x86 projects running in Simics.
% Nathaniel Wesley Filardo,
% currently a Ph.D.\ candidate at Johns Hopkins,
% % increased the sanity of the build infrastructure,
% designed the \x{swexn()} system call and helped critique this paper.
% The current reference kernel, called ``pathos,''
% was written by Michael J. Sullivan,
% currently a CMU CS Ph.D.\ candidate,
% and Elly Fong-Jones, currently at Google.
% The SMP infrastructure code was contributed by
% Ryan Pearl, currently at Dropbox.
% Joshua Wise, currently at NVIDIA,
% led several kernel-extension projects.
% % including the hypervisor project.
% The course has benefitted
% greatly from free educational Simics licenses
% donated by Virtutech/Wind River.
% Code-size numbers were generated using David A.\ Wheeler's
% ``SLOCCount.''
% Work on Landslide was sponsored by the U.S. Army Research Office under grant number W911NF0910273.

\bibliographystyle{abbrv}
\bibliography{citations}

%\balancecolumns
% That's all folks!
\end{document}
