\section{Introduction}
\label{sec:intro}

% TODO: lol.
Building real-world multithreaded concurrent programming systems is well-known to be notoriously difficult to design, implement, test, debug, hard to get right, challenging, and error-prone arguably because of the difficulties of finding, reproducing, and fixing concurrency errors, attributable to the astronomically large number of thread interleavings, which arise at random during normal program execution.

In this paper we describe our experience teaching concurrent programming at the operating-system level to advanced undergraduate students at \carnegiemellon (\cmu).
In \fourten, the Operating Systems Design and Implementation class at \cmu,
students complete aggressively low-level thread library and kernel projects,
which run on real hardware,
in which they implement threads and processes.
Presented with a kernel specification called \pebbles%
% TODO camready
%\cite{kspec}
, students build a threading and synchronization library that runs in user space on top of a \pebbles-compliant kernel in a two-week project, and subsequently, in a six-week project, write their own multi-threaded kernel implementation.
During these projects students face demanding synchronization problems, including how to implement locking primitives and how to coordinate the possibly-simultaneous exiting of parent and child processes.
% Finally, in another two-week project, students implement an extension to their Pebbles kernel. This project changes every semester, and recently has involved supporting symmetric multiprocessing (SMP).

In comparison with other educational kernel projects,
\pebbles focuses more on ground-up design than on teaching a broad feature set.
At Stanford, students build kernels following the Pintos architecture \cite{pintos}, and at MIT, using Xv6/JOS \cite{xv6}.
In both courses, student kernels support multiprocessor execution;
however, students are also supplied with starter code that already implements thread creation/destruction, context switching, and locking primitives.
In \fourten, we provide starter code for the machine boot sequence and for a minimal C library (\x{printf()}, \x{malloc()}), leaving students to design threading and scheduling primitives from scratch.

Incorporating SMP into the \fourten curriculum would expand the challenges students face in a single semester.
However, we wish to avoid compromising the complexity
%, hardware fidelity,
and learning experience of the project set.
One solution would be to ease the burden of debugging concurrency errors.
Hence, we have built \landslide%
% TODO camready
%\cite{landslide}
, a tool that uses stateless model checking \cite{verisoft,dpor,chess-icb}
as an alternative to the conventional stress tests.
%to force kernel threads to take execution paths that expose concurrency bugs.
\landslide can test student thread libraries fully automatically, and can test kernels provided some manual instrumentation.
We present our experience sharing \landslide with volunteering students who used it to find bugs in their own code,
and discuss the possibility for integrating it as a main component of operating systems curricula.
