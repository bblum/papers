\section{Outcomes}

In this section we briefly describe selected
outcomes of the Spring 2012 offering of 15-410.
We begin by discussing the results of the
pre-Landslide educational experience and then
report on the results of applying Landslide to
both kernels previously written by members of the course staff
and kernels written that semester by students.

\subsection{Teaching with Pebbles}

At approximately the mid-point of the semester,
thirty-nine thread-library projects were submitted
on behalf of thirty-eight two-person teams and one
auditing student who was working solo.
%
Submitted thread libraries were roughly 800 lines of C code
and 50 lines of x86 assembly-language code,
though some were half that size and some were triple.
%
At this point, three teams dropped and the auditing
student's requirements were complete;
also, four teams fragmented---some students dropped and
two new teams emerged.
Thus, thirty-three teams attempted the kernel project.
%
Twenty-four teams submitted kernels on time and nine
teams were forced to take extra time to complete the
kernel project.
Most kernels ranged in size from 2,500 to 3,500 lines
of C code and 100 to 400 lines of assembly code,
though some submissions were roughly twice as large
(late kernels and on-time kernels were essentially the
same size).

%%%%%%%%%%%%%%%%%%%%%%%%%%%%%%%%%%%%%%%%
% \pull \p4 \text \from \410.tex?
%%%%%%%%%%%%%%%%%%%%%%%%%%%%%%%%%%%%%%%%

The SMP kernel-extension project
was a mixed success.
Of the twenty-four teams that attempted it,
one (solo) team was unable to submit a solution.
Nineteen of the remaining twenty-three groups
completed part of the project---at least enough to
boot multiple cores
into their scheduler.
Four of those nineteen groups were able to pass
almost all of our test suite,
which consisted of simple multi-threaded applications
and machines with various core counts\shortversion
{.}
{(one, two, or eight depending on the simulator configuration).}
Seven more groups were able to pass roughly half of the test suite.
Finally, it was the sense of the grading team that
most groups would have been able to substantially complete
the project given a further week of development time.
We view this outcome as an indication that feasible adjustments
to the course design would bring this material within reach
of most students.

%%%%%%%%%%%%%%%%%%%%%%%%%%%%%%%%%%%%%%%%%%%%%%%%%%%%%%%%%%%%%%%%%%%%%%
% \newcommand{\mathcols}{\multicolumn{1}{c}{$\downarrow$} & \multicolumn{1}{c}{$M$} & \multicolumn{1}{c}{$\mu$} & \multicolumn{1}{c}{$\sigma$} & \multicolumn{1}{c|}{$\uparrow$} }
%
% \begin{table*}[htb]
% \typeout{Manually banging a table* to 9 points...}
% \fontsize{9}{12}\selectfont
% \centering
% \begin{tabular}{|l|r||r r r r r|r r r r r|} 
% \hline
% Project   & Number    & \multicolumn{5}{c|}{C SLOC} & \multicolumn{5}{c|}{Assembly SLOC} \\
% Type      & Submitted & \mathcols              & \mathcols                     \\
% \hline
% \hline
% Thread libraries & 39 &  491 &  805 &  837 & (225) & 1693   &  29 &  43 & 52.5 & (25.0) & 142 \\
% \hline
% All kernels      & 31 & 2545 & 3408 & 3620 & (824) & 6697   & 107 & 365 &  391 & (198) & 829 \\
% \hline
% On-time kernels  & 24 & 2545 & 3459 & 3703 & (881) & 6697   & 107 & 373 &  391 & (208) & 829 \\
% \hline
% Late kernels     &  7 & 2576 & 3288 & 3336 & (490) & 4124   & 189 & 354 &  391 & (155) & 716 \\
% \hline
% \end{tabular}
% \caption{Code-size statistics}
% \label{table:sizes}
% \end{table*}
%%%%%%%%%%%%%%%%%%%%%%%%%%%%%%%%%%%%%%%%%%%%%%%%%%%%%%%%%%%%%%%%%%%%%%

\subsection{Evaluating Landslide}

We evaluated Landslide in two ways: first, by instrumenting two prior-semester student kernels to measure the exploration time needed to find different races, and second, by meeting with current-semester students, before they submitted their kernel for grading, to see if they could find bugs on their own with Landslide.

In the first phase, we instrumented one kernel written by a teaching assistant in a previous year, and also one student kernel later graded by that TA.
We configured Landslide to search for five complicated well-known race conditions.
In addition to finding all five races, Landslide also found a sixth previously unknown race in the TA's own kernel.
Using additional decision points only on calls to \x{mutex_lock}, Landslide found each of the six bugs in 11 to 57 seconds on a 2.6 GHz Xeon server, executing between 1 and 377 distinct interleavings per bug.

In the user-study phase, we found that students spent on average 119 minutes (60 to 158) on the required instrumentation, and a further 36 minutes (10 to 60) refining Landslide's search.
Of the four groups who finished the required part, all four found previously unknown bugs in their kernels: two races and two deterministic errors.
These bugs manifested as infinite loops, a kernel panic, and a use-after-free.
Despite wishing the instrumentation were easier, the students reported that they found working with Landslide rewarding.
