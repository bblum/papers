\section{Introduction}

Managing parallel computation, whether on thousand-node clusters or on a single-core machine, is often taught via powerful abstractions that simplify the hardest problems.
Programming patterns such as SPMD, divide-and-conquer, and fork/join parallelism enable developers to reason about parallel algorithms at a high, abstract level.
The user-level libraries and distributed systems that provide these abstractions, in turn, rely on operating system kernels to simplify the hardware and communication mechanisms on which they are built. % garth's

With the advent of massively multicore hardware,
kernels themselves have increasing hardware concurrency and complexity. % garth's
% there is also a need for reasoning about parallelism at lower levels: those of operating system kernels and concurrent systems applications, which are built on (or even below) the abstractions of threads and processes. % ben's old text
Low-level systems programmers must reason about concurrent code without the convenience of high-level simplifications, and hence often encounter notoriously difficult concurrency errors.

In this paper we describe our experience teaching concurrent programming at the operating-system level to advanced undergraduate students at Carnegie Mellon University (CMU).
In 15-410, the Operating Systems Design and Implementation class at CMU, students complete an aggressively low-level kernel project, which runs on real hardware, in which they implement threads and processes.
Presented with a kernel specification called Pebbles~\cite{kspec}, students build a threading and synchronization library that runs in user space on top of a Pebbles-compliant kernel in a two-week project, and subsequently, in a six-week project, write their own multi-threaded kernel implementation.
During these projects students face demanding synchronization problems, including how to implement locking primitives and how to coordinate the possibly-simultaneous exiting of parent and child processes.
Finally, in another two-week project, students implement an extension to their Pebbles kernel. This project changes every semester, and recently has involved supporting symmetric multiprocessing (SMP).
% in the past has included extending their kernel to serve as a hypervisor, or to support symmetric multiprocessing.

In comparison with other educational kernel projects,
% such as Pintos~\cite{pintos} and Xv6~\cite{xv6},
Pebbles focuses more on ground-up design than on teaching a broad feature set.
% Pebbles allows students to design and implement their systems ``from the ground up.''
At Stanford, students build kernels following the Pintos architecture~\cite{pintos}, and at MIT, following the Xv6/JOS architecture~\cite{xv6}.
In both courses, student kernels support multiprocessor execution;
however, students are also supplied with starter code that already implements thread creation/destruction, context switching, and locking primitives.
In 15-410, we provide starter code for the machine boot sequence and for a minimal C library (\x{printf()}, \xmalloc()}), leaving students to design threading and scheduling primitives from scratch.

%% As we ask our students to write more and more difficult concurrent code in one semester,
%As we plan to incorporate SMP into 15-410's curriculum, expanding the challenges our students face in one semester,
%we wish to ease their burden of debugging concurrency errors, which often consumes a significant fraction of their time.
Incorporating SMP into the 15-410 curriculum would expand the challenges students face within a single semester.
However, we wish to avoid compromising the complexity, hardware fidelity, and learning experience of the project set.
One solution would be easing the burden of debugging concurrency errors.
Hence, we have built Landslide~\cite{landslide}, a tool that uses systematic exploration~\cite{verisoft} to force kernel threads to take execution paths that expose concurrency bugs.
We present our experience sharing a prototype of Landslide with volunteering students who used it to find bugs in their own code, and discuss the possibility for integrating it as a main component of the 15-410 curriculum.

% should we do a "The contributions of this paper are as follows:"?

\shortversion{}{
The rest of this paper is organized as follows.
Section~\ref{sec:curriculum} gives an overview of 15-410's five programming projects.
Section~\ref{sec:pebbles} discusses the Pebbles kernel project and how it is graded.
Section~\ref{sec:landslide} describes Landslide's testing mechanisms and how we worked with students to evaluate Landslide.
Section~\ref{sec:future} concludes, discussing strategies for making Landslide more accessible to struggling students, and how Landslide might be applied to related operating systems projects at other universities.
}
