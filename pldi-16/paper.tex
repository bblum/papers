%
% LaTeX template for prepartion of submissions to PLDI'15
%
% Requires temporary version of sigplanconf style file provided on
% PLDI'15 web site.
% 
\documentclass[pldi]{sigplanconf-pldi15}

%
% the following standard packages may be helpful, but are not required
%
\usepackage{SIunits}            % typset units correctly
\usepackage{courier}            % standard fixed width font
\usepackage[scaled]{helvet} % see www.ctan.org/get/macros/latex/required/psnfss/psnfss2e.pdf
\usepackage{url}                  % format URLs
\usepackage{listings}          % format code
\usepackage{enumitem}      % adjust spacing in enums
\usepackage[colorlinks=true,allcolors=blue,breaklinks,draft=false]{hyperref}   % hyperlinks, including DOIs and URLs in bibliography
% known bug: http://tex.stackexchange.com/questions/1522/pdfendlink-ended-up-in-different-nesting-level-than-pdfstartlink
\newcommand{\doi}[1]{doi:~\href{http://dx.doi.org/#1}{\Hurl{#1}}}   % print a hyperlinked DOI



\begin{document}

%
% any author declaration will be ignored  when using 'plid' option (for double blind review)
%

\title{Systematic Testing with Data-Race Preemption Points}
\authorinfo{Ben Blum}{Carnegie Mellon University}{bblum@cs.cmu.edu}
\authorinfo{Garth Gibson}{Carnegie Mellon University}{garth@cs.cmu.edu}

\maketitle
\begin{abstract}
Stateless model checking is a promising technique for testing concurrent programs,
but is vulnerable to exponential explosion of the state space when the test input parameters are too large.
Several partial-order reduction techniques exist for mitigating this explosion,
but even after pruning equivalent interleavings, the state space size for a fixed set of preemption points is unpredictable and often intractable.
%
Data race detection, another concurrency testing approach, focuses on identifying suspicious memory access pairs during a single test execution.
It avoids concerns of intractable state space size, but suffers from a high rate of false positives.

We present an {\em Iterative Deepening} framework for stateless model checking,
which manages the exploration of many state spaces using different subsets of preemption points.
It uses state space estimation to prioritize jobs most likely to complete in a fixed CPU budget,
and it incorporates a data-race analysis to dynamically add new preemption points to verify each data race as buggy or benign.
%
Our evaluation shows this technique is
more effective than single-state-space model checking, both at finding more bugs and at completing more state spaces when no bug exists.

\end{abstract}

\section{Introduction}

% blah blah trite opening sentence
%As parallelism becomes ever more important for achieving high performance in modern-day programs,
%so too do advanced concurrency testing techniques become important for verifying the correctness of those programs.
Concurrency bugs are notoriously hard to find and reproduce because they only appear in specific thread interleavings, which arise at random during normal program execution.
{\em Stateless model checking} \cite{verisoft} offers a method for finding such bugs,
or verifying their absence,
%by systematically executing a program along as many distinct interleavings as possible,
by forcing a program to execute each distinct interleaving,
capturing and controlling this nondeterminism in a finite state space.
Unfortunately, these state spaces explode exponentially in the size of the input program.
Reduction techniques such as Dynamic Partial Order Reduction (DPOR) \cite{dpor} and Maximal Causality Reduction (MCR) \cite{mcr} expand the limits of feasible test completion,
and search ordering strategies such as Iterative Context Bounding (ICB) \cite{chess} allow bugs to be found sooner in a given exploration should the exist.

% Can I even make a claim this broad to begin with?
However, all stateless model checkers to date are bound by a fixed set of {\em preemption points} (PPs): code locations that define the granularity at which threads interleave.
For example, CHESS \cite{chess} preemptions only on synchronization operations and library calls, which can miss lock-free shared memory races.
It provides an additional data-race analysis to report any violations of this model;
% TODO: Make sure that there are enough citations for this claim.
however, data-race analyses are prone to report false positives and benign races which require annotations or imprecise heuristics to reduce \cite{racerx,tsan,datacollider}.
%
On the other hand,
% TODO: Find a systematic tester that does this!
SKI \cite{ski}
is able to preempt threads around any shared resource access. Such fine granularity would automatically check if each data race is a real bug, but makes state space completion intractable for all but the most rudimentary test inputs.
%
This work shows how to avoid making this tradeoff decision in advance.

\newcommand\landslideid{\textsc{Landslide-ID}}
We present \landslideid...

\section{Implementation}

Landslide provides estimated completion time using the Recursive Estimator algorithm in \cite{estimation}.

% TODO
TODO

% TODO: make sure to talk about avoiding thrashing

\section{Evaluation}

Evaluation questions:
\begin{enumerate}
	\item Is ID more effective than 
\end{enumerate}


We tested 87 pintos kernels and found races in 45-47 of them, among which 34-47 of those bugs required data-race preemption points to expose. That means we rock.

% TODO: Future work.
% Future work: Add a way to configure even smaller subsets (eg "only mutex_locks called from site X") for cases where mx_lock and mx_unlock alone are still too big. Count the number of kernels for which this was the case.

\section{Related Work}

None. We are the first ever to explore this area, and of course expect the Turing Award for our groundbreaking insights.

% dbug-like systematic testers: dbug, chess, modist

% SKI-like systematic testers: SKI

% other: macemc(???), ETA (message passing only)

% TODO: Mention dbug+PARROT as a motivating argument for our work. The number of tests with INF state space, EVEN WITH ONLY PTHREAD API PPs, shows that fixed-in-advance-pp-set needs our work to improve.

\section{Conclusion}

We are
%overwhelmingly
great. Accept our paper.

% TODO: Publish source of ID wrapper alone, if can't publish landslide itself.
% TODO: Publish log files.

\section{Acknowledgements}

Thanks.

\bibliographystyle{abbrvnat}
\bibliography{citations}{}

\end{document}
