%
% LaTeX template for prepartion of submissions to PLDI'15
%
% Requires temporary version of sigplanconf style file provided on
% PLDI'15 web site.
%
\documentclass[pldi]{sigplanconf-pldi15}

%
% the following standard packages may be helpful, but are not required
%
\usepackage{SIunits}            % typset units correctly
\usepackage{courier}            % standard fixed width font
\usepackage[scaled]{helvet} % see www.ctan.org/get/macros/latex/required/psnfss/psnfss2e.pdf
\usepackage{url}                  % format URLs
\usepackage{listings}          % format code
\usepackage{enumitem}      % adjust spacing in enums
\usepackage[colorlinks=true,allcolors=blue,breaklinks,draft=false]{hyperref}   % hyperlinks, including DOIs and URLs in bibliography
% known bug: http://tex.stackexchange.com/questions/1522/pdfendlink-ended-up-in-different-nesting-level-than-pdfstartlink
\newcommand{\doi}[1]{doi:~\href{http://dx.doi.org/#1}{\Hurl{#1}}}   % print a hyperlinked DOI



\begin{document}

%
% any author declaration will be ignored  when using 'plid' option (for double blind review)
%

% TODO: Don't forget to run e.g. s/\\quicksand /\\quicksand~/
\newcommand\landslide{\textsc{Landslide}}
\newcommand\quicksand{\textsc{Quicksand}}

\title{Stateless Model Checking with Data-Race Preemption Points}
\authorinfo{Ben Blum}{Carnegie Mellon University}{bblum@cs.cmu.edu}
\authorinfo{Garth Gibson}{Carnegie Mellon University}{garth@cs.cmu.edu}

\maketitle
\begin{abstract}
Stateless model checking is a promising technique for testing concurrent programs,
but is vulnerable to exponential explosion of the state space when the test input parameters are too large.
Several partial-order reduction techniques exist for mitigating this explosion,
but even after pruning equivalent interleavings, the state space size for a fixed set of preemption points is unpredictable and often intractable.
%
Data race detection, another concurrency testing approach, focuses on identifying suspicious memory access pairs during a single test execution.
It avoids concerns of intractable state space size, but suffers from a high rate of false positives.

We present an {\em Iterative Deepening} framework for stateless model checking,
which manages the exploration of many state spaces using different subsets of preemption points.
It uses state space estimation to prioritize jobs most likely to complete in a fixed CPU budget,
and it incorporates a data-race analysis to dynamically add new preemption points to verify each data race as buggy or benign.
%
Our evaluation shows this technique is
more effective than single-state-space model checking, both at finding more bugs and at completing more state spaces when no bug exists.

\end{abstract}

%%%%%%%%%%%%%%%%%%%%%%%%%%%%%%%%%%%%%%%%%%%%%%%%%%%%%%%%%%%%%%%%%%%%%%%%%%%%%%%%

\section{Introduction}

% blah blah trite opening sentence
%As parallelism becomes ever more important for achieving high performance in modern-day programs,
%so too do advanced concurrency testing techniques become important for verifying the correctness of those programs.
Concurrency bugs are notoriously hard to find and reproduce because they only appear in specific thread interleavings, which arise at random during normal program execution.
{\em Stateless model checking} \cite{verisoft} offers a method for finding such bugs,
or verifying their absence,
%by systematically executing a program along as many distinct interleavings as possible,
by forcing a program to execute each distinct interleaving,
capturing
%and controlling
this nondeterminism in a finite state space.
Unfortunately, these state spaces explode exponentially in the size of the input program.
%Reduction
Techniques such as Dynamic Partial Order Reduction \cite{dpor} and Maximal Causality Reduction \cite{mcr} expand the limits of feasible test completion,
and search ordering strategies such as Iterative Context Bounding \cite{chess-icb} \revision{encourage}~bugs to be found sooner in a given space should they exist.

However, all stateless model checkers to date are bound by a fixed set of {\em preemption points}: code locations that define the granularity at which threads interleave.
For example, \textsc{CHESS} \cite{chess} \revisionx{by default} preempts only on synchronization operations and library calls, which can miss lock-free shared memory races.
%
On the other hand, SPIN \cite{spin} %and Inspect \cite{inspect}
%are able to preempt
preempts threads before any shared memory access.
Such fine granularity would automatically check each data race for the possibility of failure, but risks timing out before the state space can be completed. %makes full state space completion intractable for even modestly-sized tests.
\revisionx{Some tools, such as CHESS and Inspect \cite{inspect} can strike a middle ground using compiler instrumentation to statically add preemption points on memory accesses.
Nevertheless,}
choosing preemption points is a tradeoff between schedule coverage and feasibility of completion:
%but
even with state-of-the-art reduction techniques,
\revision{fixing the degree of coverage}~in advance necessarily leaves some tests unaffordably large \cite{parrot,mcr}.
%This work shows how to avoid making that tradeoff decision in advance.

We present \quicksand,
a model checking framework for deciding at runtime which preemption points to test,
according to which resulting state spaces are most likely to fit a prescribed CPU budget.
It uses data-race analysis \cite{eraser} to dynamically find new preemption points which expose bugs not reachable by preempting on API calls alone.
When prior \revision{approaches}~would time out on large tests by trying several preemption points simultaneously,
\quicksand~\revision{identifies this pitfall in advance using state space size estimation \cite{estimation},
and instead}~tests smaller state spaces based on subsets of those preemption points.
\revision{Often, testing these}~smaller state spaces can even find the same bugs sooner.

%\revision{Hence, recent approaches favor
%heuristic
%%search ordering
%strategies \cite{chess-icb,randomized-scheduler}
%to prioritize fast bug-finding over ultimate completion time,
%%assuming state spaces will always be too big and
%%at the expense of ultimate completion time,
%which in turn sacrifices the potential for a full safety guarantee when no bugs exist.}
%%search ordering strategies such as Iterative Context Bounding \cite{chess-icb}
%%assume all state spaces will be too big,
%%and optimize the search to uncover bugs more quickly at the expense of ultimate completion time.}
%%/ by sacrificing the ability to complete a total verification quickly when possible.

On the other hand, when the CPU budget is large enough to fully test all data-race preemption points,
we prove that this constitutes a total verification of all possible thread schedules.
%Short of deciding in advance to preempt on every single instruction \cite{spin}, this was not previously possible without data-race preemption points.
\revision{To achieve the same level of verification, prior model checkers must decide}~in advance to preempt on every single memory access \cite{spin}, \revision{which is computationally prohibitive for
even moderately-sized
%larger
tests.
Our approach provides the best of both worlds:
by estimating the size of the test on-the-fly, \quicksand~can find bugs quickly in large tests {\em and} provide fast total verification for small ones.}


We evaluate \quicksand~by testing \numstudence~student thread libraries and kernels from the undergraduate operating systems classes at Carnegie Mellon University, University of California at Berkeley, and University of Chicago.
%We show many advantages over both conventional stateless model checking and single-pass data-race analysis.
%We find many of the same bugs found by the conventional approach faster using subsets of preemption points,
We find that data-race preemption points quickly expose many new bugs that prior model checkers could not find at all,
and that they enable full verification of many more tests than before.
%and that they enable full verification of many racy, but non-failing, tests.

Our contributions are as follows:
\begin{enumerate}
	\item {\em Iterative Deepening}, a new \revision{algorithm}~for combining data-race analysis with stateless model checking, and \quicksand, an open-source implementation;
	\item A proof of convergence \revisionx{for sequentially-consistent memory models}, showing that should it be possible in the given CPU budget,
		fully testing every discovered data-race preemption point is equivalent to testing all possible thread schedules;
	\item A new tactic for eliminating one class of false-positive data race candidates,
		which cannot soundly be used in a single-pass analysis,
		but which we prove correct when used with Iterative Deepening;
	\item A large evaluation in which \quicksand~compares favourably to stand-alone data-race detection and stateless model checking approaches, finding new bugs that would be missed by either alone.
\end{enumerate}

% Comment this out for space?
The rest of the paper is organized as follows.
\sect{\ref{sec:overview}} reviews the background material, %on stateless model checking and data-race analysis,
\sect{\ref{sec:design}} and \sect{\ref{sec:implementation}} discuss our design and implementation, %of Iterative Deepening,
\sect{\ref{sec:soundness}} provides our proofs of soundness, %convergence and of the soundness of our new false-positive data-race tactic,
\sect{\ref{sec:eval}} presents our evaluation,
\sect{\ref{sec:future}} discusses limitations and future work,
\sect{\ref{sec:related}} surveys the related work,
and \sect{\ref{sec:conclusion}} concludes.

\section{Definitions}

\subsection{System Model}

To avoid relying on any particular programming language features,
we leave the program syntax and execution semantics opaque,
reasoning instead about execution traces.
We require that a program's evaluation produce a trace of instructions of the following form:
\begin{eqnarray*}
	\mathcal{A} &::=& v \leftarrow \mathsf{read}(a) \quad | \quad \mathsf{write}(a,v) \quad | \quad \mathsf{xchg}(a,v) \\
		&|& a \leftarrow \mathsf{malloc}(n) \quad | \quad \mathsf{free}(a) \\
		&|& \mathcal{A}_{local} \quad | \quad \mathcal{A}_{sync}
\end{eqnarray*}
The execution steps take the following forms:

{\bf Memory.}
$\mathsf{read}$, $\mathsf{write}$, and $\mathsf{xchg}$ access global or heap memory shared by all threads, indicated by some address $a$.
(Other atomic swap operations are omitted for brevity.)
$\mathsf{malloc}$ and $\mathsf{free}$ provide access to fresh memory accessible by all threads. % free is irrelevant to the system model in terms of expressive power

{\bf Local state.}
$\mathcal{A}_{local}$ represents any thread-local instruction, such as modifying local variables, flow control, function calls, and assertions. %, and side effects.
We omit a detailed list for brevity, as we do not need to reason about them in these proofs.
%For brevity, we omit a detailed list, although note that $\mathsf{call}$ would be implemented by modifying the instruction stream $[\mathcal{I}]$ and creating a new frame on the stack of local variables.

{\bf Threads.}
$\mathcal{A}_{sync}$ denotes the subclass of evaluation steps which implement inter-thread synchronization; i.e., the {\em synchronization API}:
\begin{eqnarray*}
	\mathcal{A}_{sync} &::=& \mathsf{mutex\_lock}(m) \quad | \quad \mathsf{mutex\_unlock}(m) \\
		&|& \mathsf{deschedule} \quad | \quad \mathsf{make\_runnable}(t) \\
		&|& t \leftarrow \mathsf{thread\_fork} \quad | \quad \mathsf{thread\_exit} \quad | \quad \mathsf{yield}
\end{eqnarray*}
$\mathsf{mutex\_lock}$ and
$\mathsf{mutex\_unlock}$ provide mutual exclusion:
a thread which evaluates $\mathsf{mutex\_lock}$ on some lock $m$ becomes blocked until no other thread holds $m$.
% the second formulation here excludes the possibility of m_r waking a thread blocked on mutex_lock (which would be horrible)
%$\mathsf{deschedule}$ and $\mathsf{make\_runnable}$ allow threads to manipulate their own or another's runnability, respectively,
$\mathsf{deschedule}$ alows a thread to block itself until another thread wakes it with $\mathsf{make\_runnable}$,
%
and $\mathsf{thread\_fork}$ and $\mathsf{thread\_exit}$ allow creation and destruction of new threads.
$\mathsf{thread\_fork}$ is defined in the Pebbles manner \cite{kspec};
we omit higher-level abstractions such as $\mathsf{cond\_wait}$, $\mathsf{create}$, or $\mathsf{join}$, which can be implemented using these primitives \cite{thrlib}.
%$\mathsf{yield}$ has no effect under these execution semantics, but is included for the sake of synchronization API preemption points (see below).
$\mathsf{yield}$
allows the execution semantics to switch threads %without blocking the invoker,
while respecting the preemption-point-switching invariant discussed below.
%would have no effect under an execution semantics which schedules threads arbitrarily
%but is included for the sake of synchronization API preemption points (see below).

\begin{definition}[Interleaving]
An interleaving (or execution trace) is a list of these instructions, annotated to indicate the currently-running thread as well as the runnability of all existing threads:
\begin{eqnarray*}
	\mathcal{I} &::=& [\mathcal{A}, t, (t \rightarrow \mathsf{bool})]
\end{eqnarray*}
\end{definition}

We say a {\em thread switch} occurs when adjacent elements in $\mathcal{I}$ have different thread IDs $t$.
%
We say a thread is {\em blocked} when its value in the runnability map is false.

{\bf Interleaving invariants.}
We require the evaluation semantics to produce interleavings which fulfil several invariants,
apart from the obvious ones ensuring correct synchronization and $\mathsf{malloc}$ discipline.

Model checkers often include heuristics to identify blocking via open-coded $\mathsf{yield}$ loops,
but we assume here that such patterns are implemented more tastefully with a condition-variable-like primitive built upon $\mathsf{deschedule}$.

Most importantly, we assume that threads switch only at instructions identified by the set of {\em preemption-point predicates}, defined below.

\subsection{Stateless model checking terms}

\begin{definition}[Preemption point (PP) predicate]
	A predicate on the execution state which identifies a class of instruction pairs between which we may force threads to switch.
\end{definition}

We use {\em synchronization API PPs} to denote the set of predicates
%statically-available
which occur immediately before or after any of $\mathcal{A}_{sync}$.
%$\mathsf{mutex\_lock}$,
%$\mathsf{mutex\_unlock}$,
%$\mathsf{deschedule}$,
%$\mathsf{make\_runnable}$,
%$\mathsf{thread\_fork}$,
%$\mathsf{thread\_exit}$,
%$\mathsf{yield}$.
Because no other instruction affects a thread's runnability, it is always possible to execute a program by switching the currently-executing thread only at synchronization API PPs.

All data-race PP predicates will occur immediately before a $\mathsf{read}$, $\mathsf{write}$, or atomic swap.
Other predicates are possible, though we will show they are irrelevant.

When generating execution traces, the evaluation semantics is parameterized by a set of active PP predicates.
As long as the set contains the synchronization API PPs,
%the execution shall switch threads only when a PP predicate holds.
the thread-switch invariant discussed above will hold.

\begin{definition}[Preemption point (PP)]
	%A PP is a code location between two instructions at which we may force threads to switch.
	%A PP is a predicate on the execution state which identifies a class of instruction pairs between which we may force threads to switch.
	A PP is any site in an interleaving at which a PP predicate evaluated to true.
\end{definition}

As discussed above, all thread switches occur at PPs.
However, an interleaving may also contain PPs at which the same thread continued running.

In the main paper, ``PP'' referred to what we now call ``PP predicates''.
Hence, the ``same'' PP could occur multiple times during an execution.
%for example, {\tt mutex\_lock()} may be called from {\tt foo()} and later again from {\tt bar()}.
In these proofs, we separate such cases into multiple unique PPs:
each PP is simply a label denoting the boundary between two transitions.

\begin{definition}[Transition]
A sequence of execution steps from a program's evaluation between two PPs.
\label{def:transition}

\end{definition}
The thread-switch invariant guarantees that each transition's instructions are associated with exactly one thread.
%The set of synchronization API PPs provides this invariant, and all other PP sets in these proofs will be supersets of those.
%We also assume a trace of all memory accesses is available in the representation of transitions. %% It is, now.

\begin{definition}[State space]
	A state space $\mathcal{S}$ is a set of interleavings representing all execution sequences legal under a given set of PP predicates.
\end{definition}

\begin{definition}[Must-happen-before (MHB)]
%Given two transitions $A$ and $B$, we say $A$ MHB $B$ if $B$ cannot be reordered to occur before $A$.
Let $t_1$ and $t_2$ be two transitions of an interleaving, and $T1$ and $T2$ be the corresponding thread IDs,
and let $t_1$ occur before $t_2$.
Then $t_1$ MHB $t_2$ if
\begin{enumerate}
	\item $T2$ is blocked immediately preceding $t_1$ and not blocked immediately afterward,
		and there does not occur another $t_2'$ by $T2$ between $t_1$ and $t_2$; or
	\item there occurs some $t_3$ by thread $T3$ such that $t_1$ MHB $t_3$, $t_3$ MHB $t_2$, and $T3 \ne T2$; or
	\item $T1 = T2$.
\end{enumerate}
\end{definition}

Intuitively, MHB expresses when two transitions cannot be reordered
(the ``enables'' relation in DPOR terminology \cite{dpor}).
Two transitions $A$ and $B$ of different threads MHB if some synchronization event in $A$ causes $B$ to become runnable while it was previously blocked.
Such synchronization events include $\mathsf{thread\_fork}$, $\mathsf{make\_runnable}$,
and sometimes but not always $\mathsf{mutex\_unlock}$.

Note how our {\em must}-happen-before relation differs from the conventional definition of happens-before (``observed to happen before'') \cite{lamport-clocks}.
Our use of MHB matches the ``limited happens-before'' used in \cite{hybriddatarace} and \cite{tsan};
the advantage of this over pure-happens-before detectors in producing fewer false negatives is well-argued in those prior works\footnote{
Because pure-HB data race detectors avoid false positives altogether, they would have no trouble avoiding our malloc-recycle false positives.
However, as prior work has shown, they miss many other bugs involving unprotected variables accessed alternately before and after mutex-protected critical sections.
%Indeed, because most concurrent malloc implementations are protected by a lock,
%our malloc-recycle false positives are indistinguishable from such false negatives under pure-HB.
}.
We illustrate the difference in Figure~\ref{fig:mhb}.

\begin{figure}[h]
	\small
\begin{tabular}{rll}
	& {\bf Thread 1} & {\bf Thread 2} \\
	1 & \texttt{\hilight{brickred}{my\_x->foo = ...;}} & \\
	2 & \texttt{\hilight{olivegreen}{mutex\_lock(...);}} &\\
	3 & \texttt{global\_x = my\_x;} & \\
	%4 & \texttt{\hilight{olivegreen}{yield();}} & \\
	4 & \texttt{\hilight{olivegreen}{mutex\_unlock(...);}} & \\
	5 & & \texttt{\hilight{olivegreen}{mutex\_lock(...);}} \\
	6 & & \texttt{my\_x = global\_x;} \\
	7 & & \texttt{\hilight{olivegreen}{mutex\_unlock(...);}} \\
	8 & & \texttt{if (my\_x != NULL)} \\
	9 & & \texttt{\hilight{brickred}{~~~~my\_x->foo = ...;}} \\
\end{tabular}
	\caption{Example program to illustrate the difference between {\em pure happens-before} and {\em must-happen-before}.
	Under pure happens-before (which does not identify false positives), lines 1 and 9 are not a data race candidate.
	Under MHB, they are; although after trying to reorder them, it will be classified as a false positive.}
	\label{fig:mhb}
\end{figure}

Note also that although transitions of the same thread are related by MHB,
MHB is transitive only when the latter two transitions are not by the same thread (condition 2).
While lock-protected critical sections can be reordered around each other (i.e., line 1 not MHB lines 8-9),
one cannot be reordered to be in the middle of the other (i.e, lines 3-4 MHB line 6).
%Hence, MHB is not necessarily transitive.
In the latter case, the MHB relation is established by the mutex's blocking mechanism used during contention.

Our main paper refers to this relation (in conjunction with a lock-set analysis) as Limited HB.

\begin{definition}[Shared memory conflict]
A pair of memory accesses between two threads to the same address where at least one of them is a write.
\end{definition}

% Outdated. See above.
%\begin{definition}[Interleaving]
%	An ordered list of transitions and preemption points between them.
%\end{definition}

\begin{definition}[Independent transitions]
Two transitions between different threads are independent if the intersection of their shared memory accesses contains no conflicts.
\end{definition}

\begin{definition}[Equivalent interleaving]
Two interleavings are equivalent if one can be transformed into the other by permuting only independent transitions.
\end{definition}

Intuitively, the behaviour of a program could change by reordering two transitions only if they contain a memory conflict.
All possible interleavings of a program can be partitioned into equivalence classes,
so only one interleaving from each equivalence class need be tested to ensure total schedule coverage \cite{mazurkiewicz}.
Equivalence is, of course, transitive.

% Outdated. See above.
%\begin{definition}[State space]
%	A set of interleavings subject to the constraint that, given the preemption points used, all equivalence classes of possible interleavings are represented by at least one member.
%\end{definition}

\begin{definition}[Dynamic Partial Order Reduction (DPOR)]
	A state-space search algorithm for stateless model checkers;
	given a state space $\mathcal{S}$, it will test at least one interleaving from each equivalence class in $\mathcal{S}$.
	%guaranteed to reorder transitions of two threads
	%iff they are not independent and are not related by MHB \cite{dpor}.
	\label{def:dpor}
\end{definition}

Considering an interleaving $\mathcal{I}$ in $\mathcal{S}$, if two transitions $t_1$ and $t_2$ by different threads are not independent and not related by MHB, let $\mathcal{J}$ be the interleaving which reorders $t_1$ with $t_2$. DPOR is then guaranteed to test some interleaving in $\mathcal{S}$ equivalent to $\mathcal{J}$ \cite{dpor}.

Because equivalent interleavings produce identical execution states,
DPOR guarantees to expose all reachable execution states by testing its subset of interleavings.
We refer to this property as {\em the soundness of DPOR}.

%The soundness of DPOR guarantees that if a program behaviour can possibly be exposed by any thread interleaving around the given transitions/PPs,
%that interleaving will eventually be tested by reordering only such conflicting transitions.
%In other words, reordering memory-independent thread transitions cannot possibly affect program behaviour.

\subsection{Data race and other memory terms}

\begin{definition}[Data race]
A shared memory conflict where furthermore:
\begin{itemize}
	\item The intersection of both threads' locksets is empty (i.e., the same lock does not protect both accesses), and
	\item The containing transitions are not related by MHB.
\end{itemize}
\end{definition}

The same as in the paper, we distinguish between data-race {\em candidates} (or {\em potential} data races) and data-race {\em bugs}.
For brevity, we now use ``data race'' to refer both to true races and to potential data-race candidates identified by MHB.
%In this proof we are concerned solely with candidates, and whether they can be observed to race or are false positives.
%It is up to the MC to decide whether true data races are benign or buggy.

\begin{definition}[False positive data race]
	An apparent data race that cannot be executed in the opposite order from what was observed.
\end{definition}

False positives are caused when some data dependency based on some other shared state %, invisible to the data-race analysis,
changes some variable values when the threads are reordered, such that the memory addresses no longer collide.

\begin{definition}[Malloc-recycle data race]
	A data race where the address is contained in some heap-allocated memory, and between the two accesses, that memory was passed to free() and returned again by a subsequent malloc().
\end{definition}

Figures~\ref{fig:recycle} and \ref{fig:recycle-bug} show an example.
In the case of malloc-recycle false positives, the allocation heap is the ``other shared state'' mentioned in the previous definition, and malloc's return value is the variable value that changed.

Recent work \cite{sparc-ssm} proposed hardware techniques for detecting many classes of stale heap pointer accesses, including the one shown in Figure \ref{fig:recycle-bug}.
Future work could combine this approach with MC to identify such bugs immediately,
rather than requiring Iterative Deepening to explore new state spaces corresponding to the data race.
However, if the {\tt malloc} call were in thread 1 instead of thread 2, the bug would still be nondeterministic, requiring MC to expose.

\begin{definition}[Use after free]
	Any read or write to heap memory which was once allocated, but no longer is.
\end{definition}

These can immediately be identified as failures by a MC which tracks allocation state.
%Most commonly this refers to accesses to a region already freed, but for brevity we also include

%%%%%%%%%%%%%%%%%%%%%%%%%%%%%%%%%%%%%%%%%%%%%%%%%%%%%%%%%%%%%%%%%%%%%%%%%%%%%%%%

\section{Intuition}

This section provides (hopefully) intuitive summaries of our proof goals.
%for readers not interested in double-checking the proofs' internal structure.

{\bf Intuition for Iterative Deepening convergence.}
%In summary, we are proving
We will prove
that when Iterative Deepening saturates the set of data-race PPs,
that set represents every instruction where a preemption could possibly affect the program's behaviour
Hence, completing the associated state spaces is as strong a verification as testing all possible thread interleavings by preempting anywhere.
A data-race may be hidden in control-flow paths reachable only after preempting on a different data-race,
but the technique's iterative nature will eventually find it.
On the other hand, relying on the soundness of DPOR, preempting on an instruction which is neither a data-race or sync API boundary cannot affect the program's behaviour,
so any such PPs are irrelevant.

\begin{figure}[t]
	\small
\begin{tabular}{rll}
	& \multicolumn{2}{c}{\texttt{struct x \{ int foo; int baz; \} *x;}} \\
	& \multicolumn{2}{c}{\texttt{struct y \{ int bar; \} *y;~~~~~~~~~~}} \\
	\\
	& {\bf Thread 1} & {\bf Thread 2} \\
	1 & \texttt{\hilight{brickred}{x1->foo = ...;}} & \\
	2 & \texttt{\hilight{olivegreen}{free}(x1);} \\
	3 & & \texttt{\hilight{commentblue}{// x's memory recycled}} \\
	4 & & \texttt{y~=~\hilight{olivegreen}{malloc}(sizeof *y);} \\
	5 & & \texttt{\hilight{commentblue}{// ...initialize...}}\\
	6 & & \texttt{publish(y);} \\
	7 & & \texttt{\hilight{brickred}{y->bar = ...;}} \\
\end{tabular}
\caption{False-positive malloc-recycle pattern. This is the common case for which we avoid creating new state spaces.}
\label{fig:recycle}
\end{figure}

\begin{figure}[t]
	\small
\begin{tabular}{rll}
	& {\bf Thread 1} & {\bf Thread 2} \\
	1 & \texttt{publish(x1);} & \\
	2 & & \texttt{x2 = get\_published\_x();} \\
	3 & \texttt{\hilight{brickred}{x1->foo = ...;}} & \\
	4 & \texttt{\hilight{olivegreen}{free}(x1);} \\
	5 & & \texttt{\hilight{commentblue}{// x's memory recycled}} \\
	6 & & \texttt{y~=~\hilight{olivegreen}{malloc}(sizeof *y);} \\
	7 & & \texttt{\hilight{brickred}{x2->foo = ...;}} \\
\end{tabular}
\caption{Adversarial program which fits the malloc-recycle pattern, but nevertheless contains a true race.}
\label{fig:recycle-bug}
\end{figure}

\begin{figure}[t]
	\small
\begin{tabular}{rll}
	& {\bf Thread 1} & {\bf Thread 2} \\
	1 & \texttt{publish(x1);} & \\
	2 & & \texttt{x2 = get\_published\_x();} \\
	3 & & \texttt{\hilight{commentblue}{// x not free, so malloc's}} \\
	4 & & \texttt{\hilight{commentblue}{// return value changes!}} \\
	5 & & \texttt{y~=~\hilight{olivegreen}{malloc}(sizeof *y);} \\
	6 & & \texttt{\hilight{brickred}{x2->foo = ...;}} \\
	7 & \texttt{\hilight{brickred}{x1->foo = ...;}} & \\
	8 & \texttt{\hilight{olivegreen}{free}(x1);} \\
\end{tabular}
\caption{Goal interleaving, reordering the adversarial threads away from the pattern, while the data race remains.}
\label{fig:recycle-goal}
\end{figure}

Note that while we defined synchronization API PPs to occur both before and after any $\mathcal{A}_{sync}$ instruction,
we will only add data-race PPs {\em before} their associated $\mathsf{read}$ or $\mathsf{write}$.
Our proof requires preempting only before each racing instruction, not after,
in order to fully saturate the PP set with all possible data races.
Then, once saturated, preempting after each racing instruction would be extraneous,
because all subsequent instructions before the next PP must be local operations or non-communicating reads/writes%.
% Adding a PP both before and after could potentially expose some bugs sooner,
% if the 'after' PP were redundant with another data-race PP but that other race would take forever to find.
% So, I think it's an optimization to do it this way (although I haven't done a comparison experiment).
%
\footnote{Our implementation also optimizes the synchronization API PPs,
generally preempting only {\em after} each synchronization instruction.
$\mathsf{mutex\_lock}$ is the exception, as it can cause other threads to become blocked.
All the others can only make blocked threads runnable, establishing MHB,
which also provides MHB for the preceding transition of the invoking thread.
Hence, coalescing those transitions does not exclude any possible interleavings.
% If a third thread could "query" whether the 2nd thread is still blocked (or nonexistent),
% that would break the soundness of this optimization, and you'd need to preempt before mx_unlock and thr_create after all.
% XXX: In the implementation, such queries are actually possible with m_r or yield;
% while they do not interact with mutex_unlock, I think an adversary could use them to interleave with thr_create/exit.
}.

{\bf Intuition for malloc-recycle soundness.}
%In summary, we are proving
We prove that if a malloc-recycle-pattern data race is not a false positive, %a true race, rather than a false positive,
then DPOR %is guaranteed to ``reorder away the free and re-malloc''.
%In other words, DPOR's exploration
will eventually interleave threads in such a way that the malloc-recycle pattern will disappear,
while the access pair remains for the data-race detector to find, as shown in Figure~\ref{fig:recycle-goal}.
Hence, in the same state space where the malloc-recycle data race was found, if it's a true race, the same race will also appear without the recycle pattern.
So if that race can lead to a failure, Iterative Deepening will still be led to the necessary preemption point to find it.

% TODO: Abbreviate SMC, and fixed-PP approach.

\section{Evaluation}

Although \quicksand~presents Iterative Deepening and data-race PPs as interconnected techniques, they each could theoretically be employed alone in other model checkers.
For example, a single-state-space tool could use data-race candidates during immediately subsequent interleavings, essentially changing the state space on the fly.
Likewise, a message-passing-only tool could employ Iterative Deepening despite data races being absent from its concurrency model.
Hence, though many of our experiments compare \quicksand~to the state-of-the-art as a whole,
we also sought to evaluate each technique individually.
Our evaluation answers the following questions:
\begin{enumerate}

	\item Does \quicksand~improve upon state-of-the-art MC?
		\begin{enumerate}
			\item Does Iterative Deepening find bugs faster
				%than SSS-MC
				in subset state spaces, even without data-race PPs?
			% Probably not... ICB is state of the art here.
			% In large tests, can Iterative Deepening provide partial verification by completing smaller state spaces
			\item Do data-race PPs expose new bugs that couldn't be found with SSS-MC's fixed-PP-set approach?
				% Elaborate later:
				% Among those, how many were missed in a {\em completed} execution of the otherwise ``maximal'' state space?
		\end{enumerate}
	\item Does MC improve the accuracy of data-race detection?
		\begin{enumerate}
			\item Do we avoid false positives compared to a single-execution data race analysis?
				% Explain later as:
				% How many data-race candidates were verified as benign
				% But to be fair, you have to count how many DRs are reported as "couldn't test these, check yourself" at the end.
				% Also Include:
				% How many false positives does the free-re-malloc technique suppress?
				% TODO: If you have time, re-run all of the dr-only bug tests, with DR_FALSE_NEG enabled, and see how much fewer bugs get found (how many bugs get pushed past the time limit?)
			% TODO: This one's optional. You can give up on it.
			\item Do we find data-race bugs that would be false-negatives during a single-execution analysis?%Do we avoid false negatives compared to single-pass?
				%TODO: for this experiment, set EXPLORE_BACKWARDS=0
				% TODO: And disable false-neg malloc-free technique
				% TODO: And also disable the confirmed/suspected thing
				% (where mem.c waits for reorder observed before
				% messaging the latter half of the DR to QS)
		\end{enumerate}
\end{enumerate}

%%%%%%%%%%%%%%%%%%%%%%%%%%%%%%%%%%%%%%%%%%%%%%%%%%%%%%%%%%%%%%%%%%%%%%%%%%%%%%%%

\subsection{Test Suite}
% TODO: Maybe say how many lines of code total? How many lines on average per P2/pintos? (careful with pintos; lots of basecode)
Our test suite consists of \numthrlibs~``P2'' student thread libraries, from CMU's 15-410 OS class,
%across the Spring and Fall 2014 and Spring 2015 semesters;
and \numpintoses~``Pintos'' student kernels, from Berkeley's CS162 and U. Chicago's CS230 OS classes.
%
The P2 thread library comprises \texttt{thr\_create()}, \texttt{thr\_exit()}, \texttt{thr\_join()}, mutexes, condvars, semaphores, and r/w locks;
all implemented from scratch in userspace with a UNIX-like system call interface \cite{kspec,thrlib}.
%
The Pintos kernel project
involves implementing priority scheduling, \texttt{sleep()}, and user-space process management (\texttt{wait()} and \texttt{exit()})
using provided bare-bones mutex, context-switch, and virtal memory implementations
\cite{pintos}.
% P2 SLOC stats: 1807 avg; 1723 median; range 1181-4114.
% All numbers, obtained with:
% cd p2s; for i in */*; do wc -l $i/user/libthread/*.{c,h,S} $i/user/libthread/*/*.{c,h,S} ; done | grep total
% 1181 1192 1221 1230 1238 1240 1243 1261 1275 1307 1310 1318 1325 1334 1336 1345 1366
% 1388 1388 1403 1415 1416 1430 1451 1478 1498 1527 1589 1618 1635 1638 1654 1675 1676
% 1716 1719 1720 1723 1723 1727 1737 1743 1744 1751 1769 1777 1782 1789 1789 1812 1918
% 1926 1946 1994 2022 2043 2066 2077 2088 2099 2131 2164 2172 2190 2215 2227 2277 2282
% 2384 2387 2483 2486 2503 2514 2551 2597 2610 2665 4114
Though not ``real world'' programs, both projects are quite large: % maybe "complex"?
the P2s average 1807 lines of C and x86 assembly (stddev 489.5),
% Pintos SLOC stats: TODO
and the Pintoses average {\bf 9999999} % TODO

\newcommand\mxtest{\texttt{mx\_test}}
\newcommand\tej{\texttt{thr\_exit\_join}}
\newcommand\bct{\texttt{broadcast}}
\newcommand\paraguay{\texttt{paraguay}}
\newcommand\paradise{\texttt{paradise\_lost}}
\newcommand\rwl{\texttt{rwl\_test}}
We tested P2s with 6 multithreaded programs:
% from the 410 test suite % XXX: I would like to say this but this is a lie; figure out what else i can say instead
% each tailored to exercise a different part of the P2 project
\mxtest, for locking algorithm correctness, \tej, a test of thread lifecycle, \bct~and \paraguay, for condvars, \paradise~for semaphores, and \rwl~for r/w locks.
For \mxtest, \paradise, and \paraguay, we used {\tt without\_function} to blacklist {\tt thr\_create}, {\tt thr\_exit}, and {\tt thr\_join},
and for \mxtest~we enabled \landslide's mutex-testing option
(see \sect{\ref{sec:landslide}}).
\newcommand\prisema{\texttt{priority\_sema}}
\newcommand\waitsimple{\texttt{wait\_simple}}
% TODO: Add more test cases
We tested the Pintoses with 2 programs from the class test suite: \prisema, a test of the kernel scheduling algorithm, and \waitsimple, a test of process lifecycle system calls.
For all tests, we used {\tt without\_function} to blacklist PPs on the {\tt malloc} mutex.
% XXX: Some pintoses can't run all the tests. So this number is too high.
In total, this test suite comprises 632 unique state spaces.
All tests were run on 12-core 3.2 GHz Xeon machines with 12GB of RAM.

\begin{table}[t]
	\begin{tabular}{l|l|l}
			& QS bugs & SSS-MC bugs \\
		\hline
		\mxtest & eg 1000 & eg 0 \\
		\bct & & \\
		etc... & & \\
		\hline
		Total & & \\
	\end{tabular}
	\caption{Comparison of all bugs found, broken down by test case, among all P2s (top 6) and Pintoses (bottom 2)}
	\label{tab:allbugs}
\end{table}

\begin{table}[t]
	\small
	\begin{tabular}{l|l|l||l|l}
	& QS bug & \begin{tabular}{c} SSS-MC \\ completed\end{tabular}
	& QS bug & \begin{tabular}{c}SSS-MC \\ timeout \end{tabular} \\
		\hline
		\mxtest & e.g. 5 & 10 & 0 & 0 \\
		\bct & & & & \\
		etc... & & & & \\
		\hline
		Total & & & & \\
	\end{tabular}
	\caption{Bugs requiring data-race PPs to expose, found by \quicksand~but missed by the single-state-space approach.}
	\label{tab:drbugs}
\end{table}

%%%%%%%%%%%%%%%%%%%%%%%%%%%%%%%%%%%%%%%%%%%%%%%%%%%%%%%%%%%%%%%%%%%%%%%%%%%%%%%%

\subsection{Comparing Iterative Deepening to SSS-MC}
\label{sec:eval-sssmc}

% TODO: If you have time, rerun all the quicksand experiments JUST running the 4 base state spaces. Give it 10/4 hours of cpu budget on 4 cores. John's expt.

% mention exactly which state spaces we are comparing here
% mention partial verification in terms of state space completion, when SSSMC times out.

%In this section we show that using data-race PPs with \landslide~is more effective than either SSS-MC or single-pass data-race detection alone.

To compare to SSS-MC, we ran a control experiment for each test, running \landslide~on a single state space with all PPs on sync primitives enabled in advance (and no data-race PPs).
We gave each \quicksand~test 10 CPUs for 1 hour each. % XXX
% TODO figure out somewhere to mention what landslide's pps are: mx lock/unlock (aka sema up/down)
Though \landslide~does not implement parallel DPOR \cite{parallel-dpor}, we compensated by giving each control test 1 CPU for 10 hours,
%then dividing all associated times by 10 (simulating perfect parallelism).
and instrumenting \quicksand~to report total CPU-hours rather than wall-clock time.
%\quicksand's times by 10 to convert from wall-clock time to CPU-hours (even though it sometimes falls short of 100\% parallelism).
Figure~\ref{fig:dowefindbugsfaster} plots the bug-finding speed of SSS-MC against that of two different \quicksand~experiments:
%which we explain presently.

\begin{figure}[t]
	\includegraphics[width=0.48\textwidth]{dowefindbugsfaster.pdf}
	\caption{Comparison of bug-finding performance
	by several configurations of \quicksand~and the SSS-MC control.
	\quicksand~finds 169\% as many bugs with data-race PPs.}
	\label{fig:dowefindbugsfaster}
\end{figure}

{\bf Finding the same bugs faster.}
To show that Iterative Deepening is effective even for MC domains without data races, such as message-passing distributed systems,
we ran the test suite with \quicksand~configured to explore only subsets of the hard-coded mutex PPs (i.e., ignoring all data-race reports)\footnote{
Because \quicksand~is not yet instrumented to subset hard-coded PPs beyond the 4 ways shown in Figure~\ref{fig:id},
we ran these tests for 2.5 hours on 4 CPUs each.
Future work could parallelize QS-no-DR-PPs further; see \sect{\ref{sec:future}}.}.
%
The line QS-no-DR-PPs represents this experiment;
% TODO: rephrase based on result of expt
we see that even though SSS-MC mostly catches up to it by the end of the 10-hour budget,
QS-no-DR-PPs finds many more of the bugs much sooner.
From this we conclude that for smaller arbitrary CPU budgets,
Iterative Deepening is likely to find bugs SSS-MC will miss,
and programmers can be more confident in the verification provided when \quicksand~times out with no bug found.

{\bf Finding new data-race bugs.}
Though state-of-the-art MCs preempt only on synchronization events, many serious concurrency bugs are caused by data races leading to corrupted shared state.
The line QS-DRs represents our ``no holds barred'' \quicksand~tests:
we quickly pull ahead of SSS-MC, and ultimately conclude with 69\% more bugs in total.
The break-even point is at a negligible 45 seconds.

Furthermore, we plotted another line from this dataset, QS-no-DR-bugs,
which represents only the bugs found in state spaces without data-race PPs (like QS-no-DR-PPs, but even when data-race PPs are enabled).
Intuitively, this line shows that for programs with only benign data races,
\quicksand~can afford the extra overhead of verifying them while still slightly edging out SSS-MC\footnote{
The initial perfect overlap between QS-DRs and QS-no-DR-bugs indicates how long it takes before the first data-race bug is found.}.
%even after the extra overhead of verifying them, \quicksand~still slightly edges out SSS-MC

{\bf Partial verification guarantees.}
TODO % TODO

%Hence, in Table~\ref{tab:drbugs} we count how often \quicksand~uncovered a bug only in state spaces which included data-race PPs, while
%
%In Table~\ref{tab:allbugs} ....

%%%%%%%%%%%%%%%%%%%%%%%%%%%%%%%%%%%%%%%%%%%%%%%%%%%%%%%%%%%%%%%%%%%%%%%%%%%%%%%%

\subsection{Avoiding false positive data-race candidates}
\label{sec:eval-falsepos}
% Though we mechanically verify whether each data race candidate leads to a bug, each new PP can increase combinatorially..... obviously wish to avoid...

We also counted the number of malloc-recycle false positives that \landslide~suppressed.

%%%%%%%%%%%%%%%%%%%%%%%%%%%%%%%%%%%%%%%%%%%%%%%%%%%%%%%%%%%%%%%%%%%%%%%%%%%%%%%%

\subsection{Finding nondeterministic data-race candidates}
\label{sec:eval-falseneg}
Some memory accesses may be hidden in a control flow path that requires a nondeterministic preemption to be executed.
In such cases, a single-pass dynamic data-race detector
%could not achieve the coverage necessary
would fail
to identify a racing access pair as a candidate to begin with.
%
We counted how many such data-races, used as PPs, led to \quicksand~finding new bugs,
thereby making them {\em false negatives} of the single-pass approach.
% TODO: Put a figure here giving an example of where e.g. a data race only shows up during the contention path of a mutex.
We classified each data-race candidate according to whether \landslide~had reported them during the first interleaving, before any backtracking or preempting: if so, they were {\em deterministic data races} (hence could be found by single-pass).

To ensure a fair comparison, we disabled \landslide's {\em false-positive}-avoidance techniques during this experiment.
For example, we reported malloc-recycle data races during the first interleaving as {\em deterministic}, as a single-pass analysis must,
rather than waiting until future interleavings to confirm them (as explained in \sect{\ref{sec:recycle}}).

				% TODO: Argue:
				% It is fair to compare multiple pass DR-analysis under Landslide against just a single execution because prior work DR detectors, being not integrated with a MC, are not intended to uncover different results under subsequent runs.
				% Define a "stress tester" as a class of bug detectors where they are intended to [...]
				% Maybe say: Do there exist any stress-tester DR detectors where they are intended to produce new results under reruns?

% TODO: restructure this paragraph to account for RaceFuzzer/Portend. Be like, "Yes they explore multiple schedules, but only AFTER finding a race." And/or make this argument in prior work
One might also wonder: Why is it fair to compare the data race bugs \quicksand~finds (10 CPU-hours)
against the candidates found by a prior work's single execution (less than a minute)?
We argue this comparison is meaningful because prior work data-race tools, being not integrated with a MC,
are not intended to uncover different results under subsequent runs.
One could run a data-race tool repeatedly for 10 CPU-hours, but the advantage of stateless MC over stress testing is already well-understood.
% TODO TODO TODO get a big citation for teh above sentence.


% Figure out concretely what the data race tricks are that we do, so we can claim them as contributions in the paper. Then ACTUALLY EVALUATE THEM.
%         - Speculative DR PPs.
%                 Not a heuristic, rather how to make it work at all to begin with.
%                 (Cite MS thesis, claim on backwards explorating finding bugs faster)
%         - Free/re-malloc to eliminate some false positives. See #193.
%                 Measure how many false positives are eliminated.
%                 Check, ofc, to make ABSOLUTE SURE, that no bugs missed w/ this trick.
%                         If there are, it could be because of the implementation
%                         bug described in #193.
%         - Using tid/last_call filtering because whole stack traces are too expensive.
%                 Moderately optional, 1st priority since theoretically interesting:
%                 Turn on/off and measure how resulting DR bug #s change.
%         - Optional: Reprioritizing DRs based on "confirmed" / "suspected"
%                 Shouldn't be hard just make ID wrapper print "s" or "c"!
%                 Is it helpful for ID to put priorities on DR PPs?
%                         Test by inverting the priority and see if fewer buges are found.
%         // Super optional to talk about. Probably not worth the time.
%         // - "Too suspicious" (during init/destroy)
%         //      (Cite eraser, section 2.2)

%%%%%%%%%%%%%%%%%%%%%%%%%%%%%%%%%%%%%%%%%%%%%%%%%%%%%%%%%%%%%%%%%%%%%%%%%%%%%%%%

\subsection{Discussion and future work}
\label{sec:future}

%TODO: Run a mutex expt where "all atomic instrs" are PPs. See how many bugs are missed anyway.
% Wwe re-ran the \mxtest control experiment with \landslide~hard-coded to preempt on any atomic instruction
% (as well as on the mutex API boundaries).
% Still, this smarter configuration for SSS-MC found only 99999999999 bugs of \quicksand's 13.

{\bf Testing lock implementations.}
In \sect{\ref{sec:eval-sssmc}}, using data-race PPs compares favourably across the board to SSS-MC.
Observe in particular that in \mxtest, the control experiment found dramatically fewer bugs (just 1),
even compared to the other test cases\footnote{
% FIXME: "Aren't all lock impls assembly?" "Yes, but this one was ALL assembly."
	The one bug SSS-MC found was in a fully-assembly lock implementation. {\tt yield()}'s return value clobbered a value stored in {\tt \%eax}, which could lead to a failure after two repeated contentions. Preempting only on {\tt yield()} (in the contention loop) was sufficient.}.
%Intuitively, this is due to our control experiment being able to preempt only on the boundaries of the API which
%Though for many applications of MC, assuming a correct lock implementation is sufficient,
Though it suffices for many applications of MC to assume correctly-implemented locks,
we consider this strong evidence that any verification of low-level synchronization code must incorporate data-race PPs.

% TODO: get these numbers
{\bf Finer-grained PP subsets.}
\quicksand~was able to partially guarantee safety in {\large \bf 99999\%} of tests with large maximal state spaces.
However, in {\large \bf 1337} tests, no more than the minimal state space could be verified,
and in {\large \bf 42} tests, not even that much.
Larger state spaces often result from finer-grained locking,
which can indicate a more complicated concurrent algorithm requiring more rigorous verification than a program with a single global lock.
%Hence these corner cases are important to consider for future work.
While this work used {\tt within\_function} (\sect{\ref{sec:landslide}}) {\em statically} to restrict where PPs could arise in advance of the test,
we envision future Iterative Deepening implementations could incorporate this method to {\em dynamically} subset PPs further,
making partial verification of such large tests possible.
%enabling partial verification of such large tests. % if need space

{\bf Partial verification.}
We are not the first to provide a partial verification guarantee when timing out on too-large state spaces (\sect{\ref{sec:eval-sssmc}}).
While we guarantee safety when preempted on certain combinations of PPs,
CHESS
guarantees safety under no more than a certain number of preemptions \cite{chess-icb}.
%according to the maximum bound reached in the time limit.
We imagine these two guarantees could be each be useful to developers in different scenarios,
and are presently working to combine the two approaches to provide both at once.
One benefit of our technique is that {\tt within\_function}-based Iterative Deepening (discussed above)
would enable expert developers to configure custom subsets of PPs they are most interested in verifying,
according to which modules of a codebase they wish to test.

% Future work: Add parallel DPOR so you can fill your spare CPUs when there are fewer than the max number of jobs.

\section{Related Work}
\label{sec:related}

% TODO: ", and related data race analyses based on ????"

\subsection{Model Checking}

\landslide~uses many established model-checking techniques, dating back
% of course
to Verisoft, the original C model checker \cite{verisoft}.
%, and Eraser, the original data race detector \cite{eraser}.
%Our model checker
%\landslide~\cite{landslide} itself implements many techniques from prior work (\sect{\ref{sec:landslide}}).
%itself implements DPOR \cite{dpor},
%state space estimation \cite{estimation},
%and data-race detection \cite{eraser}.
We compare related tools by their treatment of shared-memory thread communication.

{\bf Synchronization events only.} CHESS \cite{chess} and dBug \cite{dbug-ssv} instrument the thread library API, and can preempt programs only during calls to this API.
Hence they will miss any bugs that require interleaving threads at instruction granularity during a data race. CHESS provides a data-race analysis to report any such violations of its concurrency model to the user, but does not incorporate data-race candidates as PPs in future tests.

{\bf Message-passing.} Other stateless model checkers, such as SAMC \cite{samc}, MaceMC \cite{macemc}, MoDist \cite{modist}, and ETA \cite{dbug-retreat}, limit thread communication to a message-passing API to more effectively test distributed systems.
This eliminates the need for data-race analysis, but restricts the class of programs that can be tested.
Nevertheless, Iterative Deepening is still applicable to these tools.

{\bf Preempting at instruction granularity} is a prerequisite for using data-race PPs.
However, the resulting state space explosion demands that any such tool either
choose a small subset of instructions to consider as PPs
or be limited to very small test inputs.
%However, every such prior tool we know of has serious drawbacks.
SKI \cite{ski} approaches kernel code by choosing in advance a random set of instruction offsets from the start of the test,
which is more similar to stress testing or fuzzing than to exhaustive state space exploration.
SPIN \cite{spin} specializes in verifying synchronization primitive implementations such as RCU, which is very similar to our \mxtest~experiment.
However, SPIN is stateful rather than stateless, and explicitly storing visited program states rather than using DPOR limits the size of programs that can be practically tested.
SPIN also requires code to be written in the PROMELA DSL.
%so cannot check implementations directly.

{\bf Other techniques.} Various improvements to DPOR have been proposed, such as Dynamic Interface Reduction \cite{demeter}, Maximal Causality Reduction \cite{mcr}, and DPOR for TSO/PSO \cite{tsopso}.
These are all orthogonal to our technique.
Parrot \cite{parrot} combines MC with a partially-determinizing runtime, but still, fewer than half the non-trivial state spaces in their evaluation could be completed.
%providing a strong argument for \quicksand.
Finally, Iterative Context Bounding (ICB) \cite{chess-icb} is most similar to Iterative Deepening,
as both approaches provide a partial verification on some subset of interleavings when full completion is intractable (\sect{\ref{sec:future}}).
However, ICB is limited to a fixed set of PPs, and so far no algorithm has been proposed to dynamically add data-race PPs during a test with ICB.

\subsection{Data Race Detection}

%Too many related projects to list have made contributions to the
Many advances have been made on the false-positive data race problem since it was first introduced in \cite{eraser}.
\cite{hybriddatarace} and \cite{tsan} combine the lockset and happens-before analyses into a hybrid technique, which we employ.
% TODO: any more?
DroidRacer \cite{droidracer} and CAFA \cite{cafa} extend the analysis to event-driven Android applications, using domain-specific heuristics (orthogonal to our method) to reduce false positives. % cut for space?
% No, IDGAF about pure happens before.
%FastTrack \cite{fasttrack} optimizes the performance of pure happens-

% TODO: Figure out why they claim "Happens before produces NO false positives, only benign races".
% It seems impossible.
% But if true, it means either (a) they can somehow identify FRM DRs on the 1st pass, not needing to replay,
% or (b) reuse of memory by malloc is somehow outside their concurrency model.
Closer to our work, replay analysis \cite{recordreplaydrs} also suppresses false positives by testing multiple thread interleavings.
%after finding data race candidates.
This work compares the immediately resulting program states for differences,
preferring to err on the side of false positives.
RaceFuzzer \cite{racefuzzer} avoids false positives by requiring an actual failure be exhibited,
although it uses random schedule fuzzing rather than stateless model checking.
Note while these techniques can also classify our malloc-recycle candidates as false positives (\sect{\ref{sec:recycle}}),
they require replaying the threads in a new interleaving.
Moreover, \cite{portend} argues that accurate classification may require many re-executions,
%according to many pre- and post-race sequences,
which is tantamount to adding a new state space in \quicksand.
Our proof in \sect{\ref{sec:recycle}} allows eliminating this special case with no additional replay beyond what DPOR already requires.

Portend \cite{portend} is the most closely related work we have found.
% FIXME: "limited"?
Based on single-pass data race reports, it explores a limited state space to classify candidates in a taxonomy of likely severity.
Compared to us, Portend additionally finds non-failing races which nevertheless cause
%suspiciously
different program output, while we depend on directly detecting failures.
It uses symbolic execution to test input nondeterminism as well as schedule nondeterminism,
while we explore the latter only.
However, Portend does not test alternate interleavings {\em in advance} of knowing data races,
which is necessary to expose some bugs (\sect{\ref{sec:eval-falseneg}}).
% XXX: Is this true??
It also assumes the POSIX synchronization API, so cannot verify arbitrary synchronization algorithms such as we do with \mxtest.
Future work could combine the two approaches, using MC to produce new data-race traces for Portend to classify, or using Portend's analysis to inform \quicksand's state space priorities.

% \subsection{Other Concurrency Testing Approaches} % TODO: Well?
%
% blah blah pldi'15 symbiosis DSP

% TODO: talk about data race detectors???
% eg Scalable Race Detection for Android Applications -- uses domain specific heuristics to filter out false positives

% Note that BPOR paper claims that ICB(3+) repeats LOADS of work, and that makes it ok for landslide-ID to repeat work.

% IDK if i should mention it, but OOPSLA 2015, protocol based verification of MPI concurrency paper. Different verification approach entirely; doesn't suffer exponential explosion but limited to programs with no shared state and MPI communication only

% Probably NOT worth a mention: OOPSLA 2015, stateless model checking of event driven applications. Turning timer-driven model on its head and checking single-threaded, but asynch-event-driven programs (i.e. device-like signal handlers)

% TODO: Read OOPSLA 2015 "SATcheck, sat-directed SMC for SC/TSO"


%%%%%%%%%%%%%%%%%%%%%%%%%%%%%%%%%%%%%%%%%%%%%%%%%%%%%%%%%%%%%%%%%%%%%%%%%%%%%%%%

\section{Conclusion}

We are
%overwhelmingly
great. Accept our paper.

% TODO: Publish source of ID wrapper alone, if can't publish landslide itself.
% TODO: Publish log files.

\section{Acknowledgements}

Many thanks to David A. Eckhardt, Vaishaal Shankar, and Haryadi Gunawi for generously providing student implementations from CMU's, Berkeley's, and U. of Chicago's OS classes respectively.
Thanks to Ji\v{r}\'{i} \v{S}im\v{s}a and the anonymous PLDI reviewers for their helpful comments.
This work was supported in part by % TODO: funding sources


\bibliographystyle{abbrvnat}
\bibliography{citations}{}

\end{document}
