\section{Related Work}

% TODO: ", and related data race analyses based on ????"

\subsection{Model Checking}

\landslide~uses many established model-checking techniques, dating back
% of course
to Verisoft, the original C model checker \cite{verisoft}.
%, and Eraser, the original data race detector \cite{eraser}.
%Our model checker
%\landslide~\cite{landslide} itself implements many techniques from prior work (\sect{\ref{sec:landslide}}).
%itself implements DPOR \cite{dpor},
%state space estimation \cite{estimation},
%and data-race detection \cite{eraser}.
We compare related tools by their treatment of shared-memory thread communication.

{\bf Synchronization events only.} CHESS \cite{chess} and dBug \cite{dbug-ssv} instrument the thread library API, and can preempt programs only during calls to this API.
Hence they will miss any crashes that require interleaving threads at instruction granularity during a data race. CHESS provides a data-race analysis to report any such violations of its concurrency model to the user, but does not incorporate data-race candidates as PPs in future tests.

{\bf Message-passing.} Other stateless model checkers, such as SAMC \cite{samc}, MaceMC \cite{macemc}, MoDist \cite{modist}, and ETA \cite{dbug-retreat}, limit thread communication to a message-passing API to more effectively test distributed systems.
This eliminates the need for data-race analysis, but restricts the class of programs that can be tested.
Nevertheless, Iterative Deepening is still applicable to these tools.

{\bf Preempting at instruction granularity} is a prerequisite for using data-race PPs.
However, the resulting state space explosion demands that any such tool either
choose a small subset of instructions to consider as PPs.
or be limited to very small test inputs.
%However, every such prior tool we know of has serious drawbacks.
SKI \cite{ski} approaches kernel code by choosing in advance a random set of instruction offsets from the start of the test,
which is more similar to stress testing or fuzzing than to exhaustive state space exploration.
SPIN \cite{spin} specializes in verifying synchronization primitive implementations such as RCU, which is very similar to our \mxtest~experiment.
However, SPIN explicitly stores visited program states rather than using DPOR, limiting the size of programs that can be practically tested,
and requires code to be written in the PROMELA DSL.
%so cannot check implementations directly.

{\bf Other techniques.} Various improvements to DPOR have been proposed, such as Dynamic Interface Reduction \cite{demeter}, Maximal Causality Reduction \cite{mcr}, and DPOR for TSO/PSO \cite{tsopso}.
These are all orthogonal to our technique.
Parrot \cite{parrot} combines MC with a partially-determinizing runtime, but still, fewer than half the non-trivial state spaces in their evaluation could be completed.
%providing a strong argument for \quicksand.
Finally, Iterative Context Bounding (ICB) \cite{chess-icb} is most similar to Iterative Deepening,
as both approaches provide a partial verification on some subset of interleavings when full completion is intractable (\sect{\ref{sec:future}}).
However, ICB is limited to a fixed set of PPs, and so far no algorithm has been proposed to dynamically add data-race PPs during a test with ICB.

\subsection{Data Race Detection}

Too many related projects to list have made contributions to the false-positive problem first introduced in \cite{eraser}.
\cite{hybriddatarace} and \cite{tsan} combine the lockset and happens-before analyses into a hybrid technique, which we employ.
% TODO: any more?
DroidRacer \cite{droidracer} and CAFA \cite{cafa} extend the analysis to event-driven Android applications, using domain-specific heuristics (orthogonal to our method) to reduce false positives. % cut for space?

% TODO: Figure out why they claim "Happens before produces NO false positives, only benign races".
% It seems impossible.
% But if true, it means either (a) they can somehow identify FRM DRs on the 1st pass, not needing to replay,
% or (b) reuse of memory by malloc is somehow outside their concurrency model.
Closer to our work, replay analysis \cite{recordreplaydrs} also suppresses false positives by testing multiple thread interleavings.
%after finding data race candidates.
This work compares the immediately resulting program states for differences,
preferring instead still to err on the side of false positives.
RaceFuzzer \cite{racefuzzer} avoids false positives by requiring an actual failure be exhibited,
although it uses random schedule fuzzing rather than stateless model checking.
Note while these techniques can also classify our malloc-recycle candidates as false positives (\sect{\ref{sec:recycle}}),
they require replaying the threads in a new interleaving.
Moreover, \cite{portend} argues that accurate classification may require many re-executions,
%according to many pre- and post-race sequences,
which is tantamount to adding a new state space in \quicksand.
% XXX: you *ARE* replaying threads! you're just not requiring any *additional* replay with *new* PPs beyond what you already do
% XXX need rephrase!
Our proof in \sect{\ref{sec:recycle}} allows eliminating this special case with no replay required.

Portend \cite{portend} is the most closely related work we have found.
% FIXME: "limited"?
Based on single-pass data race reports, it explores a limited state space to classify candidates in a taxonomy of likely severity.
Compared to us, Portend additionally finds non-failing races which nevertheless cause
%suspiciously
different program output, while we depend on directly detecting failures.
It uses symbolic execution to test input nondeterminism as well as schedule nondeterminism,
while we explore the latter only.
However, Portend does not test alternate interleavings {\em in advance} of knowing data races,
which is necessary to expose some bugs (\sect{\ref{sec:eval-falseneg}}).
% XXX: Is this true??
It also assumes the POSIX synchronization API, so cannot verify lock implementations such as we do with \mxtest.
Future work could combine the two approaches, using \landslide~to produce new traces for Portend to classify, or using Portend's analysis to inform \quicksand's state space priorities.

% \subsection{Other Concurrency Testing Approaches} % TODO: Well?
%
% blah blah pldi'15 symbiosis DSP

% TODO: talk about data race detectors???
% eg Scalable Race Detection for Android Applications -- uses domain specific heuristics to filter out false positives

% Note that BPOR paper claims that ICB(3+) repeats LOADS of work, and that makes it ok for landslide-ID to repeat work.

% IDK if i should mention it, but OOPSLA 2015, protocol based verification of MPI concurrency paper. Different verification approach entirely; doesn't suffer exponential explosion but limited to programs with no shared state and MPI communication only

% Probably NOT worth a mention: OOPSLA 2015, stateless model checking of event driven applications. Turning timer-driven model on its head and checking single-threaded, but asynch-event-driven programs (i.e. device-like signal handlers)

% TODO: Read OOPSLA 2015 "SATcheck, sat-directed SMC for SC/TSO"
