%
% LaTeX template for prepartion of submissions to PLDI'15
%
% Requires temporary version of sigplanconf style file provided on
% PLDI'15 web site.
%
\documentclass[pldi]{../sigplanconf-pldi15}

%
% the following standard packages may be helpful, but are not required
%
\usepackage{SIunits}            % typset units correctly
\usepackage{courier}            % standard fixed width font
\usepackage[scaled]{helvet} % see www.ctan.org/get/macros/latex/required/psnfss/psnfss2e.pdf
\usepackage{url}                  % format URLs
\usepackage{listings}          % format code
\usepackage{amsthm}          % format code
\usepackage{enumitem}      % adjust spacing in enums
\usepackage[linesnumbered,ruled]{algorithm2e}
\usepackage{algpseudocode}
\usepackage{graphicx}
%\usepackage{setspace}
\usepackage[colorlinks=true,allcolors=blue,breaklinks,draft=false]{hyperref}   % hyperlinks, including DOIs and URLs in bibliography
% known bug: http://tex.stackexchange.com/questions/1522/pdfendlink-ended-up-in-different-nesting-level-than-pdfstartlink
\newcommand{\doi}[1]{doi:~\href{http://dx.doi.org/#1}{\Hurl{#1}}}   % print a hyperlinked DOI
%\doublespacing


\newtheorem{lemma}{Lemma}
\newtheorem{theorem}{Theorem}
\newtheorem{definition}{Definition}

\begin{document}

%
% any author declaration will be ignored  when using 'plid' option (for double blind review)
%

%\newcommand\landslide{\textsc{Landslide}} % Don't mention landslide in the proof.
\newcommand\quicksand{\textsc{Quicksand}}
\newcommand\simics{\textsc{Simics}}
\newcommand{\sect}[1]{\S #1}
\newcommand\hilight[2]{\color{#1}#2\color{black}}
\definecolor{olivegreen}{RGB}{0,127,0}
\definecolor{brickred}{RGB}{192,0,0}
\definecolor{commentblue}{RGB}{0,0,255}

%\title{Soundness Proof for Eliminating Malloc-Recycle Data Races in Stateless Model Checking}
\title{Source Code of Test Cases for Iterative Deepening}
\authorinfo{Ben Blum}{Carnegie Mellon University}{bblum@cs.cmu.edu}

\maketitle
\begin{abstract}
In our paper we show the effectiveness of combining dynamic data-race detection \cite{eraser,hybriddatarace} with stateless model checking \cite{verisoft,dpor,chess}.
Our approach involves adding new state spaces to explore each time a new data-race candidate is found.
This document presents the source code of the 9 test cases we used, as supplemental material to our main paper.
\end{abstract}

\section{``P2'' thread library tests}

The following tests were used to test the user-space POSIX-like thread libraries implemented by students in CMU's undergrad OS class, 15-410.
They use the thread library API defined in \cite{thrlib} and the kernel system-call API defined in \cite{kspec}.

\begin{figure*}
\begin{verbatim}
#include <syscall.h>
#include <stdlib.h>
#include <thread.h>
#include <mutex.h>
#include <cond.h>
#include "410_tests.h"
#include <report.h>
#include <test.h>

#define STACK_SIZE 4096
#define ERR REPORT_FAILOUT_ON_ERR

mutex_t lock;
cond_t cvar;
static int ready = 0;

void *waiter(void *dummy)
{
        mutex_lock(&lock);
        while (ready == 0) {
                cond_wait(&cvar, &lock);
        }
        mutex_unlock(&lock);
        return NULL;
}

int main(void)
{
        report_start(START_CMPLT);

        ERR(thr_init(STACK_SIZE));
        ERR(mutex_init(&lock));
        ERR(cond_init(&cvar));

        ERR(thr_create(waiter, NULL));

        mutex_lock(&lock);
        ready = 1;
        cond_broadcast(&cvar);
        mutex_unlock(&lock);

        return 0;
}
\end{verbatim}
	\caption{{\tt broadcast\_test}, a test of {\tt cond\_signal}/{\tt cond\_broadcast} interaction.}
	\label{fig:bct}
\end{figure*}
\begin{figure*}
\begin{verbatim}
#include <syscall.h>
#include <stdlib.h>
#include <thread.h>
#include <mutex.h>
#include <cond.h>
#include <assert.h>
#include "410_tests.h"
#include <report.h>
#include <test.h>

#define STACK_SIZE 4096
#define ERR REPORT_FAILOUT_ON_ERR

mutex_t lock;
static int num_in_section = 0;

void critical_section()
{
        mutex_lock(&lock);
        num_in_section++;
        yield(-1);
        assert(num_in_section == 1);
        num_in_section--;
        mutex_unlock(&lock);
}

void *contender(void *dummy)
{
        ERR(swexn(NULL, NULL, NULL, NULL));
        critical_section();
        // Do it again, in case lock is correct but unlock corrupts something.
        critical_section();
        vanish();
        return NULL;
}

int main(void)
{
        ERR(thr_init(STACK_SIZE));
        ERR(mutex_init(&lock));
        ERR(thr_create(contender, NULL));
        ERR(swexn(NULL, NULL, NULL, NULL));
        critical_section();
        vanish(); // bypass thr_exit
        return 0;
}
\end{verbatim}
	\caption{{\tt mutex\_test}, a test of {\tt mutex\_lock}/{\tt mutex\_unlock} correctness.
	Used with the special mutex-testing mode of data-race detection, described in the main paper (Section 3.1).}
	\label{fig:mx}
\end{figure*}
\begin{figure*}
\begin{verbatim}
#include <syscall.h>
#include <stdlib.h>
#include <thread.h>
#include <mutex.h>
#include <cond.h>
#include <sem.h>
#include <assert.h>
#include "410_tests.h"
#include <report.h>
#include <test.h>

#define STACK_SIZE 4096
#define ERR REPORT_FAILOUT_ON_ERR

mutex_t little_lock; /* used to avoid spurious data race reports */
sem_t lock;
int num_in_section = 0;

// note: the mutex accesses in this function are ignored as PPs
// by use of without_function; they are here to prevent
// num_in_section acesses from being identified as data races.
void __attribute__((noinline)) critical_section() {
        mutex_lock(&little_lock);
        num_in_section++;
        mutex_unlock(&little_lock);
        yield(-1);
        mutex_lock(&little_lock);
        assert(num_in_section == 1);
        num_in_section--;
        mutex_unlock(&little_lock);
}
void *consumer(void *dummy) {
        ERR(swexn(NULL, NULL, NULL, NULL));
        sem_wait(&lock);
        critical_section();
        sem_signal(&lock);
        vanish();
        return NULL;
}
void producer() {
        ERR(mutex_init(&little_lock));
        ERR(sem_init(&lock, 0));
        ERR(thr_create(consumer, NULL));
        ERR(thr_create(consumer, NULL));
        sem_signal(&lock);
}
int main(void) {
        ERR(thr_init(STACK_SIZE));
        ERR(swexn(NULL, NULL, NULL, NULL));
        producer();
        vanish(); // bypass thr_exit
        return 0;
}
\end{verbatim}
	\caption{{\tt sem\_test},
	%known to 15-410 students as {\tt paradise\_lost}, % TODO CAMREADY
	a test of {\tt sem\_wait}/{\tt sem\_signal} interaction.}
	\label{fig:paradise}
\end{figure*}
\begin{figure*}
\begin{verbatim}
#include <syscall.h>
#include <stdlib.h>
#include <thread.h>
#include <mutex.h>
#include <cond.h>
#include "410_tests.h"
#include <report.h>
#include <test.h>

#define STACK_SIZE 4096
#define ERR REPORT_FAILOUT_ON_ERR

mutex_t lock1, lock2;
cond_t cvar1, cvar2;
int slept1 = 0; /* Whether thread1 has gotten to sleep on cvar1 */
int signaled1 = 0; /* Set right before main thread signals cvar1 */
int slept2 = 0; /* Whether thread1 has gotten to sleep on cvar2 */
int signaled2 = 0; /* Set right before main thread signals cvar2 */

void *thread1(void *dummy) {
        /* go to sleep on cvar1 */
        mutex_lock(&lock1);
        slept1 = 1;
        cond_wait(&cvar1, &lock1);
        if (!signaled1) {
                report_end(END_FAIL);
        }   
        mutex_unlock(&lock1);

        /* go to sleep on cvar1 */
        mutex_lock(&lock2);
        slept2 = 1;
        cond_wait(&cvar2, &lock2);
        if (!signaled2) {
                report_end(END_FAIL);
        }   
        mutex_unlock(&lock2);
        return NULL;
}
\end{verbatim}
	\caption{{\tt signal\_test},
	%known to 15-410 students as {\tt paraguay}, % TODO CAMREADY
	a test for a deep bug in {\tt cond\_signal} (Part 1; see part 2 in Figure~\ref{fig:paraguay-pt2}).}
	\label{fig:paraguay}
\end{figure*}
\begin{figure*}
\begin{verbatim}
int main(void) {
        ERR(thr_init(STACK_SIZE));
        ERR(mutex_init(&lock1));
        ERR(mutex_init(&lock2));
        ERR(cond_init(&cvar1));
        ERR(cond_init(&cvar2));
        ERR(thr_create(thread1, NULL));

        /* Wait for thread1 to get to sleep on cvar1. */
        mutex_lock(&lock1);
        while (!slept1) {
                mutex_unlock(&lock1);
                thr_yield(-1);
                mutex_lock(&lock1);
        }
        /* Indicate that we are about to signal */
        signaled1 = 1;
        mutex_unlock(&lock1);

        /* Signal. Note that we know for sure that thread1 is asleep on
         * cvar1 (assuming correct cond vars and mutexes...) */
        cond_signal(&cvar1);

        /* Now do it all again for the second set of things. */
        /* Wait for thread1 to get to sleep on cvar2. */
        mutex_lock(&lock2);
        while (!slept2) {
                mutex_unlock(&lock2);
                thr_yield(-1);
                mutex_lock(&lock2);
        }
        /* Indicate that we are about to signal */
        signaled2 = 1;
        mutex_unlock(&lock2);

        /* Signal. Note that we know for sure that thread1 is asleep on
         * cvar1 (assuming correct cond vars and mutexes...) */
        cond_signal(&cvar2);

        return 0;
}
\end{verbatim}
	\caption{{\tt signal\_test},
	%known to 15-410 students as {\tt paraguay}, % TODO CAMREADY
	a test for a deep bug in {\tt cond\_signal}
	(Part 2; see part 1 in Figure~\ref{fig:paraguay}).}
	\label{fig:paraguay-pt2}
\end{figure*}
\begin{figure*}
\begin{verbatim}
#include <thread.h>
#include <mutex.h>
#include <cond.h>
#include <rwlock.h>
#include <syscall.h>
#include <stdlib.h>
#include <stdio.h>
#include "410_tests.h"
#include <test.h>
#define STACK_SIZE 4096

int read_count = 0;
mutex_t read_count_lock;
rwlock_t lock;
int pass = -1; 

void g() {
        mutex_lock(&read_count_lock);
        read_count++;
        if (read_count == 2) {
                pass = 0;
        }   
        mutex_unlock(&read_count_lock);
}
void *f(void *arg) {
        rwlock_lock(&lock, RWLOCK_READ);
        g();
        return((void *)0);
}
int main() {
        int tid1;
        thr_init(STACK_SIZE);

        REPORT_ON_ERR(rwlock_init(&lock));
        REPORT_ON_ERR(mutex_init(&read_count_lock));
        rwlock_lock(&lock, RWLOCK_WRITE);

        if ((tid1 = thr_create(f, NULL)) < 0) {
                REPORT_END_FAIL;
                return -1; 
        }   
        rwlock_downgrade(&lock);
        g();
        if (thr_join(tid1, NULL) < 0) {
                REPORT_END_FAIL;
                return -1;
        }
        if (pass != 0)
                REPORT_END_FAIL;
        return pass;
}
\end{verbatim}
	\caption{{\tt rwlock\_test},
	%known to 15-410 students as {\tt rwlock\_downgrade\_read\_test}, % TODO CAMREADY
	a test of all {\tt rwlock} functions.}
	\label{fig:rwldgr}
\end{figure*}
\begin{figure*}
\begin{verbatim}
#include <stdio.h>
#include <thread.h>
#include <syscall.h> /* for PAGE_SIZE */

void *waiter(void *p) {
  int status;
  thr_join((int)p, (void **)&status);
  thr_exit((void *) 0); 
}

int main() {
  thr_init(16 * PAGE_SIZE);
  (void) thr_create(waiter, (void *) thr_getid());
  thr_exit((void *)'!');
}
\end{verbatim}
	\caption{{\tt join\_test},
	% known to 15-410 students as {\tt thr\_exit\_join}, % TODO CAMREADY
	a test of {\tt thr\_create}, {\tt thr\_exit}, and {\tt thr\_join}.}
	\label{fig:tej}
\end{figure*}

\begin{enumerate}
	\item {\tt broadcast\_test}: Figure \ref{fig:bct}
	\item {\tt mutex\_test}: Figure \ref{fig:mx}
	\item {\tt sem\_test}: Figure \ref{fig:paradise}
	\item {\tt signal\_test}: Figures \ref{fig:paraguay} and \ref{fig:paraguay-pt2}
	\item {\tt rwlock\_test}: Figure \ref{fig:rwldgr}
	\item {\tt join\_test}: Figure \ref{fig:tej}
\end{enumerate}

\section{``Pintos'' kernel tests}

The following tests were used to test the first two Pintos kernel projects, ``threads'' and ``userprog'',
implemented by students in Berkeley's and the University of Chicago's undergrad OS classes,
CS162 and CS230 respectively \cite{pintos}.
They use the internal kernel API provided by the basecode, available online at \cite{pintos-github}.

\begin{figure*}
\begin{verbatim}
#include <stdio.h>
#include "tests/threads/tests.h"
#include "threads/init.h"
#include "threads/malloc.h"
#include "threads/synch.h"
#include "threads/thread.h"
#include "devices/timer.h"

static struct semaphore sema;

static void priority_sema_thread (void *aux UNUSED) 
{
  sema_down (&sema);
}

void test_priority_sema (void) 
{
  int i;

  sema_init (&sema, 0); 
  thread_set_priority (PRI_DEFAULT);
  for (i = 0; i < 2; i++)
    {   
      int priority = PRI_DEFAULT;
      char name[16];
      snprintf (name, sizeof name, "priority %d", priority);
      thread_create (name, priority, priority_sema_thread, NULL);
    }   

  for (i = 0; i < 2; i++)
    {   
      sema_up (&sema);
    }   
}
\end{verbatim}
	\caption{{\tt sched\_test},
	% known to CS162 and CS230 students as {\tt priority\_sema}, % TODO CAMREADY
	a test of {\tt thread\_create}, {\tt sema\_up}, and {\tt sema\_down}.}
	\label{fig:prisema}
\end{figure*}
\begin{figure*}
\begin{verbatim}
#include <stdio.h>
#include "tests/threads/tests.h"
#include "threads/init.h"
#include "threads/synch.h"
#include "threads/thread.h"
#include "devices/timer.h"

struct semaphore done_sema;

// semaphore PPs here are ignored using without_function
static __attribute__((noinline)) void child_done() {
        sema_up(&done_sema);
}
// semaphore PPs here are ignored using without_function
static __attribute__((noinline)) void parent_done() {
        sema_down(&done_sema);
        sema_down(&done_sema);
}

static void sleeper (void *hax) {
        timer_sleep ((int)hax);
        child_done();
        thread_yield();
}
void test_alarm_simultaneous() {
        sema_init(&done_sema, 0); 
        thread_create ("Patrice Godefroid", PRI_DEFAULT, sleeper, (void *)3);
        thread_create ("Cormac Flanagan", PRI_DEFAULT, sleeper, (void *)4);
        thread_yield();
        timer_sleep (2);
        parent_done();
        thread_yield();
}
\end{verbatim}
	\caption{{\tt alarm\_test},
	% known to CS162 and CS230 students as {\tt alarm\_simultaneous}, % TODO CAMREADY
	a test of {\tt timer\_sleep}.}
	\label{fig:alarm}
\end{figure*}
\begin{figure*}
\begin{verbatim}
#include <syscall.h>
#include "tests/lib.h"
#include "tests/main.h"

void test_main (void) 
{
  msg ("wait(exec()) = %d", wait (exec ("child-simple")));
}
\end{verbatim}
	\caption{{\tt wait\_test},
	% known to CS162 and CS230 students as {\tt wait\_simple}, % TODO CAMREADY
	a test of {\tt exec} and of {\tt wait}/{\tt exit} interaction.}
	\label{fig:waitsimple}
\end{figure*}
\begin{figure*}
\begin{verbatim}
#include <stdio.h>
#include "tests/lib.h"

int main (void) 
{
  msg ("run");
  return 81; 
}
\end{verbatim}
	\caption{{\tt child\_simple}, a helper program used by {\tt wait\_test} (Figure~\ref{fig:waitsimple}).}
	\label{fig:childsimple}
\end{figure*}

\begin{enumerate}
	\item {\tt sched\_test}: Figure \ref{fig:prisema}
	\item {\tt alarm\_test}: Figure \ref{fig:alarm}
	\item {\tt wait\_test}: Figures \ref{fig:waitsimple} and \ref{fig:childsimple}
\end{enumerate}




\bibliographystyle{abbrvnat}
\bibliography{../citations}{}
\end{document}
