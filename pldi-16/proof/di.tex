\section{Definitions}

\subsection{Stateless model checking terms}

\begin{definition}[Transition]
A sequence of instructions executed by the program between two preemption points (PPs).
\label{def:transition}
\end{definition}
Our model checker provides the invariant that each transition's instructions are associated with exactly one thread. (That is, the set of PPs always includes all thread switches.)

\begin{definition}[Must-happen-before (MHB)]
	Given two transitions $A$ and $B$, we say $A$ MHB $B$ if $B$ cannot be reordered to occur before $A$.
\end{definition}
All transitions of the same thread are trivially MHB-related.
Two transitions $A$ and $B$ of different threads MHB if some synchronization event in $A$ causes $B$ to become runnable while it was previously blocked. Such synchronization events include {\tt thread\_create}, {\tt cond\_signal}, {\tt sem\_post}, but {\em not} {\tt mutex\_lock} or {\tt mutex\_unlock}.
% FIXME: Address concern about using blocking sync primitives, eg sem or rwlock, in a mutex like fashion.
% FIXME: Even though MHB(T1_0, T1_1) and MHB(T1_1, T2_0), then still not necessarily MHB(T1_0, T2_0).

Note how our {\em must}-happen-before relation differs from the conventional definition of happens-before (``observed to happen before'') \cite{lamport-clocks}.
Our use of MHB matches the ``limited happens-before'' used in \cite{hybriddatarace} and \cite{tsan};
the advantage of this over pure-happens-before detectors in producing fewer false negatives is well-argued in those prior works\footnote{
Because pure-HB data race detectors avoid false positives altogether, they would have no trouble avoiding our malloc-recycle false positives.
However, as prior work has shown, they miss many other bugs involving unprotected variables accessed alternately before and after mutex-protected critical sections.
Indeed, because most concurrent malloc implementations are protected by a lock,
our malloc-recycle false positives are indistinguishable from such false negatives under pure-HB.
}.

\begin{definition}[Shared memory conflict]
A pair of memory accesses between two threads to the same address where at least one of them is a write.
\end{definition}
Intuitively, the behaviour of a program could change by reordering two transitions only if they contain a memory conflict.

\begin{definition}[Dynamic Partial Order Reduction (DPOR)]
	A state-space search algorithm for stateless model checkers,
	guaranteed to reorder transitions of two threads
	iff they have a shared memory conflict and are not related by MHB \cite{dpor}.
	\label{def:dpor}
\end{definition}

\subsection{Data race and other memory terms}

\begin{definition}[Data race]
A shared memory conflict where furthermore:
\begin{itemize}
	\item The intersection of both threads' locksets is empty (i.e., the threads do not hold the same lock during each access), and
	\item The threads' transitions are not related by MHB.
\end{itemize}
\end{definition}

The same as in the paper, we distinguish between data-race {\em candidates} and data-race {\em bugs}.
In this proof we are concerned solely with candidates, and judging whether they are true data races or false positives.
It is up to \quicksand, outside the scope of this proof, to decide whether true data races are benign or buggy.

\begin{definition}[False positive data race]
	An apparent data race that cannot be observed in the opposite order from what was actually executed.
\end{definition}

False positives are caused when some data dependency based on some other shared state, invisible to the data-race analysis,
changes some variable values when the threads are reordered, such that the memory addresses no longer collide.

\begin{definition}[Malloc-recycle data race]
	A data race where the address is contained in some heap-allocated memory, and between the two accesses, that memory was passed to free() and returned again by a subsequent malloc().
\end{definition}

Figures~\ref{fig:recycle} and \ref{fig:recycle-bug} show an example.
In the case of malloc-recycle false positives, the allocation heap is the ``other shared state'' mentioned in the previous definition, and malloc's return value is the variable value that changed.

\begin{definition}[Use after free]
	Any read or write to heap memory which was once allocated, but no longer is.
\end{definition}

These can immediately be identified as failures by a MC which tracks allocation state.
%Most commonly this refers to accesses to a region already freed, but for brevity we also include
% TODO: need heap block overrun?

\begin{figure}[t]
	\small
\begin{tabular}{rll}
	& \multicolumn{2}{c}{\texttt{struct x \{ int foo; int baz; \} *x;}} \\
	& \multicolumn{2}{c}{\texttt{struct y \{ int bar; \} *y;~~~~~~~~~~}} \\
	\\
	& Thread 1 & Thread 2 \\
	1 & \texttt{\hilight{brickred}{x1->foo = ...;}} & \\
	2 & \texttt{\hilight{olivegreen}{free}(x1);} \\
	3 & & \texttt{// x's memory recycled} \\
	4 & & \texttt{y~=~\hilight{olivegreen}{malloc}(sizeof *y);} \\
	5 & & \texttt{// ...initialize...}\\
	6 & & \texttt{publish(y);} \\
	7 & & \texttt{\hilight{brickred}{y->bar = ...;}} \\
\end{tabular}
\caption{False-positive malloc-recycle pattern. This is the common case for which we avoid creating new state spaces.}
\label{fig:recycle}
\end{figure}

\begin{figure}[t]
	\small
\begin{tabular}{rll}
	& Thread 1 & Thread 2 \\
	1 & \texttt{publish(x1);} & \\
	2 & & \texttt{x2 = get\_published\_x();} \\
	3 & \texttt{\hilight{brickred}{x1->foo = ...;}} & \\
	4 & \texttt{\hilight{olivegreen}{free}(x1);} \\
	5 & & \texttt{// x's memory recycled} \\
	6 & & \texttt{y~=~\hilight{olivegreen}{malloc}(sizeof *y);} \\
	7 & & \texttt{\hilight{brickred}{x2->foo = ...;}} \\
\end{tabular}
\caption{Adversarial program which fits the malloc-recycle pattern, but nevertheless contains a true race.}
\label{fig:recycle-bug}
\end{figure}

\begin{figure}[t]
	\small
\begin{tabular}{rll}
	& Thread 1 & Thread 2 \\
	1 & \texttt{publish(x1);} & \\
	2 & & \texttt{x2 = get\_published\_x();} \\
	3 & & \texttt{// x not free, so malloc's} \\
	4 & & \texttt{// return value changes!} \\
	5 & & \texttt{y~=~\hilight{olivegreen}{malloc}(sizeof *y);} \\
	6 & & \texttt{\hilight{brickred}{x2->foo = ...;}} \\
	7 & \texttt{\hilight{brickred}{x1->foo = ...;}} & \\
	8 & \texttt{\hilight{olivegreen}{free}(x1);} \\
\end{tabular}
\caption{Goal interleaving, which reorders the adversarial program's threads away from the pattern, while the data race remains.}
\label{fig:recycle-goal}
\end{figure}

%%%%%%%%%%%%%%%%%%%%%%%%%%%%%%%%%%%%%%%%%%%%%%%%%%%%%%%%%%%%%%%%%%%%%%%%%%%%%%%%

\section{Intuition}

In this section we provide (hopefully) intuitive summaries of both our proof goals,
for readers who might be satisfied without necessarily double-checking the proofs' internal structure.

{\bf Intuition for Iterative Deepening convergence.}
In summary, we are proving that when Iterative Deepening finishes saturating the set of available data-race PPs,
that set represents every single code location where a preemption could possibly affect the program's behaviour,
and that completing the associated state spaces is as strong of a verification as testing all possible thread interleavings under any preemptions anywhere.
Some data-races may be hidden in control-flow paths which could only be executed after preempting on a different data-race,
but the technique's iterative nature will eventually find it.
On the other hand, relying on the soundness of DPOR, preempting on an instruction which is neither a data-race or sync API boundary cannot affect the program's behaviour,
so any PPs beyond the ones we consider are irrelevant.

{\bf Intuition for malloc-recycle soundness.}
In summary, we are proving that if a malloc-recycle-pattern data race is a true race, rather than a false positive,
then DPOR is guaranteed to ``reorder away the free and re-malloc''.
In other words, DPOR's exploration will eventually interleave threads in such a way that the malloc-recycle pattern will disappear,
while the access pair remains for the data-race detector to find, as shown in Figure~\ref{fig:recycle-goal}.
Hence, in the same state space where the malloc-recycle data race was found, if it's a true race, the same race will also appear without the recycle pattern.
So if that race hides a failure bug (assertion, segfault, etc.), Iterative Deepening will still be led to the necessary preemption point to find that bug.
