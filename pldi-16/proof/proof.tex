%
% LaTeX template for prepartion of submissions to PLDI'15
%
% Requires temporary version of sigplanconf style file provided on
% PLDI'15 web site.
%
\documentclass[pldi]{../sigplanconf-pldi15}

%
% the following standard packages may be helpful, but are not required
%
\usepackage{SIunits}            % typset units correctly
\usepackage{courier}            % standard fixed width font
\usepackage[scaled]{helvet} % see www.ctan.org/get/macros/latex/required/psnfss/psnfss2e.pdf
\usepackage{url}                  % format URLs
\usepackage{listings}          % format code
\usepackage{amsthm}          % format code
\usepackage{enumitem}      % adjust spacing in enums
\usepackage[linesnumbered,ruled]{algorithm2e}
\usepackage{algpseudocode}
\usepackage{graphicx}
%\usepackage{setspace}
\usepackage[colorlinks=true,allcolors=blue,breaklinks,draft=false]{hyperref}   % hyperlinks, including DOIs and URLs in bibliography
% known bug: http://tex.stackexchange.com/questions/1522/pdfendlink-ended-up-in-different-nesting-level-than-pdfstartlink
\newcommand{\doi}[1]{doi:~\href{http://dx.doi.org/#1}{\Hurl{#1}}}   % print a hyperlinked DOI
%\doublespacing


\newtheorem{lemma}{Lemma}
\newtheorem{theorem}{Theorem}
\newtheorem{definition}{Definition}

\begin{document}

%
% any author declaration will be ignored  when using 'plid' option (for double blind review)
%

%\newcommand\landslide{\textsc{Landslide}} % Don't mention landslide in the proof.
\newcommand\quicksand{\textsc{Quicksand}}
\newcommand\simics{\textsc{Simics}}
\newcommand{\sect}[1]{\S #1}
\newcommand\hilight[2]{\color{#1}#2\color{black}}
\definecolor{olivegreen}{RGB}{0,127,0}
\definecolor{brickred}{RGB}{192,0,0}
\definecolor{commentblue}{RGB}{0,0,255}

%\title{Soundness Proof for Eliminating Malloc-Recycle Data Races in Stateless Model Checking}
\title{Soundness Proofs for Iterative Deepening}
\authorinfo{Ben Blum}{Carnegie Mellon University}{bblum@cs.cmu.edu}

\maketitle
\begin{abstract}
In our paper we show the effectiveness of combining dynamic data-race detection \cite{eraser,hybriddatarace} with stateless model checking \cite{verisoft,dpor}.
Our approach involves adding new state spaces to explore each time a new data-race candidate is found.
This document presents two proofs concerning the technique.

First, we prove that given enough time, Iterative Deepening will converge to the same degree of schedule coverage that would be provided by a na\"{i}ve model checker (MC) preempting on every single instruction.
In other words, when Iterative Deepening finishes generating new data-race PPs, and completes all associated state spaces,
it serves as a full verification of {\em all possible} thread schedules of the given test case.

Second, we prove the soundness of the {\em malloc-recycle} false-positive-elimination tactic we discuss in the paper.
Despite many powerful reduction techniques, the state spaces are still exponentially sized,
so any way to avoid exploring some in advance is obviously beneficial.
We prove that for the special case of {\em malloc-recycle} false positives, it is safe to eliminate these immediately upon discovering them, without bothering to explore their associated state spaces.


This document is supplemental material to our main paper; we assume the reader is already familiar with our motivation and terminology.
\end{abstract}

\section{Definitions}

\subsection{System Model}

To reason about programs at the executable level, we will define them as a list of assembly-like instructions,
including certain concurrency primitives which we evaluate atomically, assuming the runtime provides correct implementations for them.
\begin{eqnarray*}
	\mathcal{P} &::=& [n, \mathcal{I}] \\
	\mathcal{I} &::=& v \leftarrow \mathsf{read}(a) \quad | \quad \mathsf{write}(a,v) \quad | \quad \mathsf{xchg}(a,v) \\
	&|& a \leftarrow \mathsf{malloc}(n) \quad | \quad \mathsf{free}(a) \\
	&|& \mathsf{mutex\_lock}(m) \quad | \quad \mathsf{mutex\_unlock}(m) \\
	&|& \mathsf{deschedule} \quad | \quad \mathsf{make\_runnable}(t) \\
	&|& t \leftarrow \mathsf{thread\_fork} \quad | \quad \mathsf{thread\_exit} \quad | \quad \mathsf{yield} \\
	&|& \textit{local}
\end{eqnarray*}

The execution state consists of any number of threads, each represented by a list of instructions to execute ($[\mathcal{I}]$),
a stack of variable maps for thread-local state ($v$),
and an indicator of whether the thread is runnable.
The state also includes a map of global memory, indexed by addresses denoted $a$,
and a set of mutex objects, denoted $m$. Thus:
\begin{eqnarray*}
	\mathcal{E} &::=& [\mathcal{T}] \times [a \rightarrow \mathsf{int}] \times [m \rightarrow \mathsf{bool}] \\
	\mathcal{T} &::=& \mathcal{P} \times [[v \rightarrow \mathsf{int}]] \times \mathsf{bool}
\end{eqnarray*}
For the sake of brevity, we will not explicitly define execution semantics, though they are straightforward:
To execute one step, arbitrarily choose one runnable thread, and evaluate the first element of its instruction list.
This gives rise to thread concurrency which can interleave at instruction granularity,
although it assumes a sequentially-consistent memory model. For a treatment of DPOR with relaxed memory, we refer the reader to \cite{tsopso}.

{\bf Memory.} $\mathsf{read}$, $\mathsf{write}$, and $\mathsf{xchg}$ access global or heap memory shared by all threads, indicated by some address $a$, and copying and/or modifying a thread-local variable $v$. % cmpxchg, xadd, etc omitted for brevity
$\mathsf{malloc}$ and $\mathsf{free}$ provide access to fresh memory accessible by all threads. % free is irrelevant to the system model in terms of expressive power

{\bf Threads.} $\mathsf{mutex\_lock}$ and
$\mathsf{mutex\_unlock}$ provide mutual exclusion:
a thread cannot evaluate $\mathsf{mutex\_lock}$ until that lock is released.
$\mathsf{deschedule}$ and $\mathsf{make\_runnable}$ allow threads to manipulate their own or another's runnability, respectively,
and $\mathsf{thread\_fork}$ and $\mathsf{thread\_exit}$ allow creation and destruction of new threads.
$\mathsf{thread\_fork}$ is defined in the Pebbles manner \cite{kspec};
we omit higher-level abstractions such as $\mathsf{create}$ or $\mathsf{join}$, which can be implemented using these primitives \cite{thrlib}.
$\mathsf{yield}$ has no effect under these execution semantics, but is included for the sake of synchronization API preemption points (see below).

{\bf Local state.}
$\textit{local}$ represents any thread-local instruction, such as modifying local variables, flow control, function definition/calling, assertions, and side effects.
For brevity, we omit a detailed list, although note that $\mathsf{call}$ would be implemented by modifying the instruction stream $[\mathcal{I}]$ and creating a new frame on the stack of local variables.

\begin{definition}[Blocked threads]
	A thread is blocked when either its runnable flag is false, or when its next instruction is a $\mathsf{mutex\_lock}$ on an unavailable mutex.
\end{definition}

Model checkers often also include heuristics to identify threads using open-coded $\mathsf{yield}$ loops to block,
but for the sake of our proofs, we assume such patterns are implemented more tastefully with a condition-variable-like primitive built upon $\mathsf{deschedule}$.

\begin{definition}[Interleaving]
	An interleaving is a sequence of execution steps, beginning with some initial program state and ending when all threads are blocked or have exited, and/or when a bug is observed.
\end{definition}

\subsection{Stateless model checking terms}

\begin{definition}[Preemption point (PP)]
	%A PP is a code location between two instructions at which we may force threads to switch.
	A PP is a predicate on the execution state which identifies a class of instruction pairs between which we may force threads to switch.
\end{definition}

In the main paper, we identified PPs by predicates on the instruction pointer and current thread ID. Hence, the ``same'' PP may occur multiple times during an execution; for example, {\tt mutex\_lock()} may be called from {\tt foo()} and later again from {\tt bar()}.
%For the sake of this proof, we separate such cases into multiple unique PPs: each PP is simply a label denoting the end of a single transition.
In our proofs, we will use the term ``PP'' to indicate specific instruction sites at which a PP predicate holds.

We also use ``synchronization API PPs'' to denote the class of
%statically-available
PPs which occur immediately {\em after} any of
$\mathsf{mutex\_lock}$,
$\mathsf{mutex\_unlock}$,
$\mathsf{deschedule}$,
$\mathsf{make\_runnable}$,
$\mathsf{thread\_fork}$,
$\mathsf{thread\_exit}$,
$\mathsf{yield}$.
Because no other instruction affects a thread's runnability, it is always possible to execute a program by switching the currently-executing thread only at synchronization API PPs.

All data-race PPs will occur immediately {\em before} a $\mathsf{read}$, $\mathsf{write}$, or $\mathsf{xchg}$.

\begin{definition}[Transition]
A sequence of execution steps from a program's evaluation between two preemption points (PPs).
\label{def:transition}

\end{definition}
We require the invariant that each transition's instructions are associated with exactly one thread. (That is, the set of PPs always includes all thread switches.)
The set of synchronization API PPs provides this invariant, and all other PP sets in these proofs will be supersets of those.
We also assume a trace of all memory accesses is available in the representation of transitions.

\begin{definition}[State space]
Given a set of PP predicates, a state space is
a set of interleavings representing all possible execution sequences which switch threads only on those PPs.
\end{definition}

\begin{definition}[Must-happen-before (MHB)]
%Given two transitions $A$ and $B$, we say $A$ MHB $B$ if $B$ cannot be reordered to occur before $A$.
Let $t_1$ and $t_2$ be two transitions of an interleaving, and $T1$ and $T2$ be the corresponding thread IDs,
and let $t_1$ occur before $t_2$.
Then $t_1$ MHB $t_2$ if
\begin{enumerate}
	\item $T2$ is blocked immediately preceding $t_1$ and not blocked immediately afterward,
		and there does not occur another $t_2'$ by $T2$ between $t_1$ and $t_2$; or
	\item there occurs some $t_3$ by thread $T3$ such that $t_1$ MHB $t_3$, $t_3$ MHB $t_2$, and $T3 \ne T2$; or
	\item $T1 = T2$.
\end{enumerate}
\end{definition}

Intuitively, MHB expresses the "cannot be reordered with" (or "enables") relation between transitions.
Two transitions $A$ and $B$ of different threads MHB if some synchronization event in $A$ causes $B$ to become runnable while it was previously blocked. Such synchronization events include {\tt thread\_create}, {\tt cond\_signal}, {\tt sem\_post}, but {\em not} {\tt mutex\_lock} or {\tt mutex\_unlock}.

Note how our {\em must}-happen-before relation differs from the conventional definition of happens-before (``observed to happen before'') \cite{lamport-clocks}.
Our use of MHB matches the ``limited happens-before'' used in \cite{hybriddatarace} and \cite{tsan};
the advantage of this over pure-happens-before detectors in producing fewer false negatives is well-argued in those prior works\footnote{
Because pure-HB data race detectors avoid false positives altogether, they would have no trouble avoiding our malloc-recycle false positives.
However, as prior work has shown, they miss many other bugs involving unprotected variables accessed alternately before and after mutex-protected critical sections.
%Indeed, because most concurrent malloc implementations are protected by a lock,
%our malloc-recycle false positives are indistinguishable from such false negatives under pure-HB.
}.
We illustrate the difference in Figure~\ref{fig:mhb}.

\begin{figure}[t]
	\small
\begin{tabular}{rll}
	& Thread 1 & Thread 2 \\
	1 & \texttt{\hilight{brickred}{my\_x->foo = ...;}} & \\
	2 & \texttt{\hilight{olivegreen}{mutex\_lock(...);}} &\\
	3 & \texttt{global\_x = my\_x;} & \\
	%4 & \texttt{\hilight{olivegreen}{yield();}} & \\
	4 & \texttt{\hilight{olivegreen}{mutex\_unlock(...);}} & \\
	5 & & \texttt{\hilight{olivegreen}{mutex\_lock(...);}} \\
	6 & & \texttt{my\_x = global\_x;} \\
	7 & & \texttt{\hilight{olivegreen}{mutex\_unlock(...);}} \\
	8 & & \texttt{if (my\_x != NULL)} \\
	9 & & \texttt{\hilight{brickred}{~~~~my\_x->foo = ...;}} \\
\end{tabular}
	\caption{Example program to illustrate the difference between {\em pure happens-before} and {\em must-happen-before}.
	Under pure happens-before (which does not identify false positives), lines 1 and 9 are not a data race candidate.
	Under MHB, they are; although after trying to reorder them, it will be classified as a false positive.}
	\label{fig:mhb}
\end{figure}

Note also that although transitions of the same thread are related by MHB,
MHB is transitive only when the latter two transitions are not by the same thread (condition 2).
While lock-protected critical sections can be reordered around each other (i.e., line 1 not MHB lines 8-9),
one cannot be reordered to be in the middle of the other (i.e, lines 3-4 MHB line 6).
%Hence, MHB is not necessarily transitive.
In the latter case, the MHB relation is established by the mutex's blocking mechanism used during contention.

In our main paper, we refer to this relation (in conjunction with a lock-set analysis) as Limited HB.

\begin{definition}[Shared memory conflict]
A pair of memory accesses between two threads to the same address where at least one of them is a write.
\end{definition}

% Outdated. See above.
%\begin{definition}[Interleaving]
%	An ordered list of transitions and preemption points between them.
%\end{definition}

\begin{definition}[Independent transitions]
Two transitions between different threads are independent if the intersection of their shared memory accesses contains no conflicts.
\end{definition}

\begin{definition}[Equivalent interleaving]
Two interleavings are equivalent if one can be transformed into the other by permuting only independent transitions.
\end{definition}

Intuitively, the behaviour of a program could change by reordering two transitions only if they contain a memory conflict.
All possible interleavings of a program can be partitioned into equivalence classes,
so only one interleaving from each equivalence class need be tested to ensure total schedule coverage \cite{mazurkiewicz}.

% Outdated. See above.
%\begin{definition}[State space]
%	A set of interleavings subject to the constraint that, given the preemption points used, all equivalence classes of possible interleavings are represented by at least one member.
%\end{definition}

\begin{definition}[Dynamic Partial Order Reduction (DPOR)]
	A state-space search algorithm for stateless model checkers;
	given a state space $\mathcal{S}$, it will test at least one interleaving from each equivalence class in $\mathcal{S}$.
	%guaranteed to reorder transitions of two threads
	%iff they are not independent and are not related by MHB \cite{dpor}.
	\label{def:dpor}
\end{definition}

Considering an interleaving $\mathcal{I}$ in $\mathcal{S}$, if two transitions $t_1$ and $t_2$ by different threads are not independent and not related by MHB, let $\mathcal{J}$ be the interleaving which reorders $t_1$ with $t_2$. DPOR is then guaranteed to test some interleaving in $\mathcal{S}$ equivalent to $\mathcal{J}$ \cite{dpor}.

Because equivalent interleavings produce identical execution states,
DPOR guarantees to expose all reachable execution states by testing its subset of interleavings.
We refer to this property as {\em the soundness of DPOR}.

%The soundness of DPOR guarantees that if a program behaviour can possibly be exposed by any thread interleaving around the given transitions/PPs,
%that interleaving will eventually be tested by reordering only such conflicting transitions.
%In other words, reordering memory-independent thread transitions cannot possibly affect program behaviour.

\subsection{Data race and other memory terms}

\begin{definition}[Data race]
A shared memory conflict where furthermore:
\begin{itemize}
	\item The intersection of both threads' locksets is empty (i.e., the threads do not hold the same lock during each access), and
	\item The threads' transitions are not related by MHB.
\end{itemize}
\end{definition}

The same as in the paper, we distinguish between data-race {\em candidates} (or {\em potential} data races) and data-race {\em bugs}.
For brevity, we now use ``data race'' to refer both to true races and to potential data-race candidates identified by MHB.
In this proof we are concerned solely with candidates, and whether they can be observed to race or are false positives.
It is up to the MC to decide whether true data races are benign or buggy.

\begin{definition}[False positive data race]
	An apparent data race that cannot be observed in the opposite order from what was actually executed.
\end{definition}

False positives are caused when some data dependency based on some other shared state, invisible to the data-race analysis,
changes some variable values when the threads are reordered, such that the memory addresses no longer collide.

\begin{definition}[Malloc-recycle data race]
	A data race where the address is contained in some heap-allocated memory, and between the two accesses, that memory was passed to free() and returned again by a subsequent malloc().
\end{definition}

Figures~\ref{fig:recycle} and \ref{fig:recycle-bug} show an example.
In the case of malloc-recycle false positives, the allocation heap is the ``other shared state'' mentioned in the previous definition, and malloc's return value is the variable value that changed.

Recent work \cite{sparc-ssm} has proposed hardware techniques for detecting many classes of stale heap pointer accesses, including the one shown in Figure \ref{fig:recycle-bug}.
This approach could be combined with stateless model checking in future work to identify such bugs immediately,
rather than requiring Iterative Deepening to explore new state spaces corresponding to the data race.
However, if the {\tt malloc} call were in thread 1 instead of thread 2, the bug would still be nondeterministic, requiring stateless model checking to expose.

\begin{definition}[Use after free]
	Any read or write to heap memory which was once allocated, but no longer is.
\end{definition}

These can immediately be identified as failures by a MC which tracks allocation state.
%Most commonly this refers to accesses to a region already freed, but for brevity we also include

\begin{figure}[t]
	\small
\begin{tabular}{rll}
	& \multicolumn{2}{c}{\texttt{struct x \{ int foo; int baz; \} *x;}} \\
	& \multicolumn{2}{c}{\texttt{struct y \{ int bar; \} *y;~~~~~~~~~~}} \\
	\\
	& Thread 1 & Thread 2 \\
	1 & \texttt{\hilight{brickred}{x1->foo = ...;}} & \\
	2 & \texttt{\hilight{olivegreen}{free}(x1);} \\
	3 & & \texttt{\hilight{commentblue}{// x's memory recycled}} \\
	4 & & \texttt{y~=~\hilight{olivegreen}{malloc}(sizeof *y);} \\
	5 & & \texttt{\hilight{commentblue}{// ...initialize...}}\\
	6 & & \texttt{publish(y);} \\
	7 & & \texttt{\hilight{brickred}{y->bar = ...;}} \\
\end{tabular}
\caption{False-positive malloc-recycle pattern. This is the common case for which we avoid creating new state spaces.}
\label{fig:recycle}
\end{figure}

\begin{figure}[t]
	\small
\begin{tabular}{rll}
	& Thread 1 & Thread 2 \\
	1 & \texttt{publish(x1);} & \\
	2 & & \texttt{x2 = get\_published\_x();} \\
	3 & \texttt{\hilight{brickred}{x1->foo = ...;}} & \\
	4 & \texttt{\hilight{olivegreen}{free}(x1);} \\
	5 & & \texttt{\hilight{commentblue}{// x's memory recycled}} \\
	6 & & \texttt{y~=~\hilight{olivegreen}{malloc}(sizeof *y);} \\
	7 & & \texttt{\hilight{brickred}{x2->foo = ...;}} \\
\end{tabular}
\caption{Adversarial program which fits the malloc-recycle pattern, but nevertheless contains a true race.}
\label{fig:recycle-bug}
\end{figure}

\begin{figure}[t]
	\small
\begin{tabular}{rll}
	& Thread 1 & Thread 2 \\
	1 & \texttt{publish(x1);} & \\
	2 & & \texttt{x2 = get\_published\_x();} \\
	3 & & \texttt{\hilight{commentblue}{// x not free, so malloc's}} \\
	4 & & \texttt{\hilight{commentblue}{// return value changes!}} \\
	5 & & \texttt{y~=~\hilight{olivegreen}{malloc}(sizeof *y);} \\
	6 & & \texttt{\hilight{brickred}{x2->foo = ...;}} \\
	7 & \texttt{\hilight{brickred}{x1->foo = ...;}} & \\
	8 & \texttt{\hilight{olivegreen}{free}(x1);} \\
\end{tabular}
\caption{Goal interleaving, which reorders the adversarial program's threads away from the pattern, while the data race remains.}
\label{fig:recycle-goal}
\end{figure}

%%%%%%%%%%%%%%%%%%%%%%%%%%%%%%%%%%%%%%%%%%%%%%%%%%%%%%%%%%%%%%%%%%%%%%%%%%%%%%%%

\section{Intuition}

In this section we provide (hopefully) intuitive summaries of both our proof goals.
%for readers not interested in double-checking the proofs' internal structure.

{\bf Intuition for Iterative Deepening convergence.}
In summary, we are proving that when Iterative Deepening finishes saturating the set of available data-race PPs,
that set represents every single code location where a preemption could possibly affect the program's behaviour,
and that completing the associated state spaces is as strong of a verification as testing all possible thread interleavings under any preemptions anywhere.
Some data-races may be hidden in control-flow paths which could only be executed after preempting on a different data-race,
but the technique's iterative nature will eventually find it.
On the other hand, relying on the soundness of DPOR, preempting on an instruction which is neither a data-race or sync API boundary cannot affect the program's behaviour,
so any PPs beyond the ones we consider are irrelevant.

{\bf Intuition for malloc-recycle soundness.}
In summary, we are proving that if a malloc-recycle-pattern data race is a true race, rather than a false positive,
then DPOR is guaranteed to ``reorder away the free and re-malloc''.
In other words, DPOR's exploration will eventually interleave threads in such a way that the malloc-recycle pattern will disappear,
while the access pair remains for the data-race detector to find, as shown in Figure~\ref{fig:recycle-goal}.
Hence, in the same state space where the malloc-recycle data race was found, if it's a true race, the same race will also appear without the recycle pattern.
So if that race hides a failure bug (assertion, segfault, etc.), Iterative Deepening will still be led to the necessary preemption point to find that bug.

\section{Assumptions}

This section documents the assumptions we make about the concurrency model, language model, and test environment,
and discusses some limitations that may arise from these assumptions.

\subsection{Assumptions for both proofs}

{\bf Maximal state space.}
We assume the model checker has all static/hard-coded PPs enabled during this state space exploration,
and that we are not limited by a pressing CPU budget.
Further, we assume that the static PPs include all lock/unlock/trylock operations on mutexes (or whatever other low-level locks are used) and also all higher-level sync primitives which can cause HB (either directly, or because they are built on top of mutexes).

Using only a subset of PPs or aborting early due to time-out could each ruin our ability to reach the goal interleaving or goal state space.
However, Iterative Deepening aims to test the most important interleavings with the time available,
so in the case of not enough time, our point here is that continuing the current state space fits that goal best.
%An alternate approach would be to accept the malloc-recycle PPs, giving them the lowest possible priority, just in case any CPUs would otherwise be idle with not enough jobs to run.
%In that case, our point is to justify deprioritizing those jobs so aggressively.

%{\bf Low overhead.}
%We assume the model checker can identify malloc-recycle data races with little or no overhead beyond what's already associated with data race detection.
%Our MC already tracks the heap state, so we implement this check for free with a simple generation counter.

{\bf Shared memory thread communication.}
We assume that the only way for two threads' transitions to affect each other's behaviour, should they be reordered,
is through either shared memory or a correctly-instrumented sync API.
Both DPOR and data-race detection rely on this assumption,
as any other way for threads to affect each other's behaviour would invisibly reduce independence and break soundness.
For brevity in these proofs, we refer to all thread communication as shared memory,
and assume that other mechanisms, such as system calls that access the filesystem, could be instrumented to fit the same model.

{\bf Schedule nondeterminism only.}
We discount the possibility of other types of nondeterminism,
such as program input nondeterminism (including randomness/timestamps) or
store-buffer nondeterminism on weak-memory architectures.
We refer the reader to \cite{klee,portend} for related work on the former, and to \cite{tsopso} for the latter.

\subsection{Assumptions for Iterative Deepening convergence}

{\bf Locks are correct.}
Because hybrid data-race detection uses lockset analysis to conclude that many access pairs could not possibly race,
we assume that no preemption during a lock-protected critical section could cause a contending thread to make meaningful progress.
This extends to use of disabling/enabling interrupts during kernel-space testing, which we model as a single global lock,
although we do not model RCU \cite{rcu}.
Our data-race analysis also models the behaviour of r/w locks and 1-initialized semaphores (the latter heuristically), although this is tangential to the proof.

Should the user wish to verify these properties of locks,
they could either run a locking test separately (as we suggest in our paper),
which would cheaply test the locks to an extent limited by the separate test case;
or they could remove the data-race analysis's lockset tracking,
which would expensively test all the main program's required locking properties in tandem with the program itself.

{\bf No forcibly blocking other threads.}
We assume the sync API behaves much like pthreads and/or message-passing:
that is, a thread can cause itself to block on some condition,
and other threads may wake it up,
but there exists no primitive which one thread can use to revoke another thread's runnable status when that other thread is not itself using the sync API concurrently.

{\bf Halting.}
From a certain perspective, proving convergence is the same as proving completeness,
which should be impossible for any runtime analysis of a Turing-complete language \cite{entscheidungsproblem}.
However, being already limited by practical real-world CPU budgets,
we are already accepting that many tests will time out.
We are concerned with proving the limited case that when \quicksand~{\em does} terminate with all state spaces completed, the verification is sound.
Inversely, during our proof, when we show that Iterative Deepening will eventually converge to an arbitrary buggy interleaving, we assume that no intermediate state spaces contain nontermination bugs.
If they do, for the sake of testing/verification, we are satisfied with finding that bug instead.
In this proof, we assume that the MC's heuristic infinite loop checker has no false negatives;
i.e., it will never get stuck forever in an infinite loop without identifying a bug, while occasionally producing false positives.
% , such as for a program trying to find a counterexample for Sally's Conjecture.
% TODO: cite some unproved mathematical conjecture by a woman, as an example of an undecidable program?


\subsection{Assumptions for malloc-recycle soundness}

{\bf Malloc is a magic black box.}
We assume the malloc implementation is correct (e.g., it won't double-allocate blocks), although we don't assume any implementation details such as a tendency to reuse blocks or allocate adjacent ones.
%particular allocation pattern regarding adjacency/coalescing/reuse.
In fact, in our experiments we instruct our MC to ignore all potential PPs which might be inserted on malloc's internal mutex;
in essence treating it as a ``magic primitive'', because we are not interested in verifying its implementation.
Furthermore, we configured DPOR to ignore the internal heap metadata accesses
when tracking shared memory conflicts to achieve greater state space reduction
(i.e., if the only consequence of reordering two transitions is malloc returning different addresses, we consider them independent).
%for the purpose of avoiding malloc-only shared memory conflicts.
This is not without consequences; see section~\ref{sec:owned}.

{\bf Sharing heap addresses.}
Finally, we assume that the only way the program can obtain heap addresses is through the return value of malloc().
Because we are testing C programs, any bizarre violations of this assumption are technically possible,
but should you wish to check for bugs like this,
%we would recommend a data-flow analysis which is much cheaper than model checking anyway.
symbolic execution \cite{klee} would be more appropriate.

For Section \ref{sec:proof}, we further assume a malloced block's address cannot be obtained through arithmetic on the address of a different block; in Section \ref{sec:owned} we show how to account for this case by relaxing the previous ``black box'' assumption.

\section{Proof of Iterative Deepening convergence}

We seek to prove that, given enough time, Iterative Deepening offers the same coverage of thread interleavings that could be achieved by preempting at every single instruction.
We'll call the latter the {\em na\"{i}ve state space}, and call the condition of testing all its interleavings {\em convergence}.
(It could also be called soundness, from the perspective of viewing Iterative Deepening as a search re-ordering heuristic that doesn't miss any interleavings.)
Hence, our convergence statement is as follows:

\begin{theorem}[Total verification]
	If Iterative Deepening fully saturates its set of data-race PPs and completes all associated state spaces,
	it serves as a verification of all possible thread schedules of the given test program.
	\label{thm:totalverif}
\end{theorem}

The contrapositive statement offers more structure for an easier proof\footnote{
For the reader who likes to avoid non-constructive proofs, % \cite{vargomax}, % TODO: uncomment for camready/TR
note that we are using the constructive contrapositive direction:
Theorem~\ref{thm:convergence} is $A \rightarrow B$ and Theorem~\ref{thm:totalverif} is $\neg B \rightarrow \neg A$.}:

%DPOR + Iterative Deepening will eventually test every nonequivalent thread interleaving that would be exposed by preempting at every single instruction
\begin{theorem}[Convergence of Iterative Deepening]
	If a bug is exposed by an interleaving in the na\"{i}ve state space, Iterative Deepening will eventually test an equivalent interleaving which exposes the same bug.
	\label{thm:convergence}
\end{theorem}

\newcommand\ppnext[1]{\ensuremath{\mathsf{next}(#1)}}
\newcommand\ppinstr[1]{\ensuremath{\mathsf{instr}(#1)}}
\newcommand\ppothers[1]{\ensuremath{\mathsf{others}(#1)}}
\newcommand\pppfx[1]{\ensuremath{\mathsf{pfx}(#1)}}
\newcommand\conflicts[2]{\ensuremath{\mathsf{conflicts}(#1,#2)}}
\newcommand\pai{\ensuremath{p_{\alpha{}i}}}
\newcommand\tai{\ensuremath{t_{\alpha{}i}}}
Let
\[
	\mathcal{I} = \{(t_{\alpha{}1}, p_{\alpha{}1}), (t_{\beta{}1}, p_{\beta{}1}), ... (t_{\alpha{}n}, p_{\alpha{}n}), ...\}
\]
be an interleaving trace, where the element $(\tai, \pai)$ represents the $i$th transition $t$ of thread $\alpha$, and the associated preemption point $p$.
We will use the following notation:
\begin{itemize}
	\item $\ppnext{\pai}$ indicates the next transition by the same thread, $t_{\alpha{}(i+1)}$,
	\item $\ppinstr{\pai}$ indicates the first instruction executed during $t_{\alpha{}(i+1)}$. and
	\item $\ppothers{\pai}$ indicates the set of all transitions by other threads between $\tai$ and $t_{\alpha{}(i+1)}$.
	\item $\conflicts{t_\alpha,t_\beta}$ indicates the set of memory access pairs between $t_\alpha$ and $t_\beta$ such that among each pair, one access belongs to each thread $\alpha$ and $\beta$, the address is the same, and at least one is a write; i.e., precisely the shared memory conflict relation required by DPOR to determine dependent interleavings.
\end{itemize}

%%%%%%%%%%%%%%%%%%%%%%%%%%%%%%%%%%%%%%%%%%%%%%%%%%%%%%%%%%%%%%%%%%%%%%%%%%%%%%%%

\subsection{Equivalence of irrelevant PPs}

Iterative Deepening will not, of course, preempt on exactly the same instructions that an arbitrary naive interleaving would,
as it is only capable of preempting on data races and sync API boundaries.
Thus, our first task is to show that all naive interleavings have an equivalent interleaving with only data-race and sync API PPs.

\begin{definition}[Relevant preemption point]
	We say a PP $\pai$ is {\em relevant} if either $\ppnext{\pai}$ is a sync API boundary (including a voluntary yield),
	or if
	\[
		\conflicts{\ppothers{\pai}}{\ppinstr{\pai}} \ne \emptyset
	\]
	i.e., the instruction by thread $\alpha$ immediately after $\pai$ has a memory conflict with some other thread interleaved during $\pai$.
\end{definition}
\begin{definition}[Fully-relevant interleaving]
	An interleaving comprised only of relevant PPs.
\end{definition}

\begin{lemma}
	For any interleaving in the na\"{i}ve state space, there exists an equivalent interleaving which uses only relevant PPs.
	%data-race PPs and sync API PPs.
	\label{lem:equivalent}
\end{lemma}

\begin{proof}
	Let $\pai$ be the first irrelevant PP of a naive interleaving $\mathcal{I}$.
	We ask, did some thread among $\ppothers{\pai}$ conflict with any instruction from $\ppnext{\pai}$, even though there was no conflict with $\ppinstr{\pai}$?
	\begin{itemize}
			% rho for relevant.
		\item If $\conflicts{\ppothers{\pai}}{\ppnext{\pai}} \ne \emptyset$, or if $\tai$ contains a sync API boundary instruction that's not already a PP,
			then let $\rho$ be either the first instruction by $\ppnext{\pai}$ among the conflicts or the first sync API instruction, whichever comes first.
			Let $\pppfx{\rho}$ denote the instruction sequence between $\ppinstr{\pai}$ and $\rho$, including the former but not the latter.
			Now, we output the new interleaving:

			\begin{tabular}{lll}
				\\
				$\mathcal{I}' = \{...,$&$(\tai \cup \pppfx{\rho}, \pai'),$& \\
									   &$\ppothers{\pai},$& \\
					       &$(t_{\alpha{}(i+1)}) \setminus \pppfx{\rho}, p_{\alpha{}(i+1)}),$ & $ ...\}$ \\
				\\
			\end{tabular}

			In other words, we have simply reordered $\ppothers{\pai}$ to between $\pppfx{\rho}$ and $\rho$, removing the irrelevant $\pai$ and adding a new PP $\pai'$, which is relevant by the construction of $\rho$.
			We know it must be possible to reorder $\ppothers{\pai}$ to after $\alpha$'s execution because no synchronization is done during $\pppfx{\rho}$, hence the other threads' runnability cannot be affected.
			Likewise $\pppfx{\rho}$ is independent with $\ppothers{\pai}$, so by the soundness of DPOR, $\mathcal{I}' \equiv \mathcal{I}$.
		\item Otherwise, $\conflicts{\ppothers{\pai}}{\ppnext{\pai}} = \emptyset$, and no instructions among $\ppnext{\pai}$ are a sync API boundary.
			Then we output the new interleaving:
			\[
				\mathcal{I}' = \{..., (\tai \cup t_{\alpha(i+1)}, p_{\alpha(i+1)}), \ppothers{\pai},...\}
			\]
			In other words, we have reordered {\ppothers{\pai}} with the entire next transition by thread $\alpha$.
			We know this must be possible for the same reasons as above, and again, the transitions are independent, so $\mathcal{I}' \equiv \mathcal{I}$.
					       %&$(t_{\alpha{}(i+1)}) \setminus \pppfx{\rho}, p_{\alpha{}(i+1)}),$ & $ ...\}$ \\
	\end{itemize}
	This constitutes an algorithm for inductively converting (or removing) all irrelevant PPs to relevant ones.
\end{proof}

%%%%%%%%%%%%%%%%%%%%%%%%%%%%%%%%%%%%%%%%%%%%%%%%%%%%%%%%%%%%%%%%%%%%%%%%%%%%%%%%

\subsection{Saturation of data-race PPs}

Now we must show that, starting from a sync-PP-only state space,
Iterative Deepening will eventually interleave threads sufficiently to expose all data races which are used as PPs in any buggy fully-relevant interleaving.
The challenge of this proof is that some data-race candidates are nondeterministic to find;
i.e., they may be hidden in an obscure control flow path which requires a prior preemption on a different data race to expose,
as shown in Figure~\ref{fig:nondet-dr}.
Hence, the maximal state space of statically-available sync API PPs will not necessarily uncover all possible data-race PPs;
Iterative Deepening may need to iterate through some data-race state spaces before finding certain nondeterministic races.

\begin{figure}[t]
	\small
	\begin{tabular}{rl}
	& \multicolumn{1}{c}{~\texttt{int x = 0, y = 0;~~~~~~~~~~~~~~~~}} \\
	& \multicolumn{1}{c}{\texttt{bool t1\_x = false, t1\_y = false;}} \\
	& \multicolumn{1}{c}{\texttt{bool t2\_x = false, t2\_y = false;}} \\
\\
		& Thread 1 (Thread 2 similar, with {\tt t1} and {\tt t2} vars swapped) \\
	1 & \texttt{\hilight{brickred}{x = x + 1;}} \\
	2 & \texttt{t1\_x = true;} \\
	3 & \texttt{\hilight{commentblue}{// "if x raced"}} \\
	4 & \texttt{if (x == 1 \&\& t2\_x) \{} \\
	5 & \texttt{~~~~\hilight{brickred}{y = y + 1;}} \\
	6 & \texttt{~~~~t1\_y = true;} \\
	7 & \texttt{~~~~\hilight{commentblue}{// "if y raced"}} \\
	8 & \texttt{~~~~if (y == 1 \&\& t2\_y) \{} \\
	9 & \texttt{~~~~~~~~panic();} \\
	10 & \texttt{~~~~\}} \\
	11 & \texttt{\}} \\
	\end{tabular}
	\caption{Not all data races will immediately be uncovered by interleaving threads around sync API PPs alone. Here, preempting during the race on line 1 is necessary to force the program into the control flow which can race on line 5.}
	\label{fig:nondet-dr}
\end{figure}

\begin{definition}[Reachable data race]
A data race candidate which will be identified by a MC configured to preempt only on locking API boundaries,
or transitively also configured to use data-race PPs of other
%already
reachable data races.
\end{definition}

\begin{definition}[Reachable preemption point]
A PP $\pai$ such that either $\ppinstr{\pai}$ is part of a reachable data race, or a sync API boundary.
\end{definition}

\begin{lemma}
	All PPs of any fully-relevant interleaving are reachable.
	\label{lem:saturation}
\end{lemma}

\newcommand\pbh{\ensuremath{p_{\beta{}h}}}
\newcommand\paj{\ensuremath{p_{\alpha{}j}}}
\newcommand\pbk{\ensuremath{p_{\beta{}k}}}
\newcommand\tbh{\ensuremath{t_{\beta{}h}}}
\newcommand\coalesce[1]{\ensuremath{\mathsf{coalesce}(#1)}}

\begin{proof}
Let $\mathcal{I} = \{(t_{\alpha{}1}, p_{\alpha{}1}), ...\}$ be the fully-relevant interleaving and PP sequence in the premise of the lemma which exposed a bug.
We proceed by induction on the PPs according to the order of their preemptions.
(Note that this is not necessarily the same as the order of the racing instructions $\ppinstr\pai$, which occur in $\ppnext\pai$, not in \tai.)
For both the base case and inductive step, we know (vacuously, for the former) that for each other PP $\pbh$ with $\tbh \prec \tai \in \mathcal{I}$, $\pbh$ is reachable.
We must show that a data race involving $\ppinstr\pai$
is reachable, and we are not allowed to use $\pai$ until we find the data race between $\ppinstr\pai$ and $\ppothers\pai$.

{\bf Coalescing not-yet-reachable data race PPs.}
Consider the alternate interleaving prefix:
\[
	\mathcal{J} = \{..., (\tai \cup \ppnext\pai', \paj), \ppothers\pai' \}
\]
where we have reordered $\ppnext\pai$ to before the execution of $\ppothers\pai$.
Here, $\paj$ is the first sync API PP in $\alpha$ after $\pai$ (as $p_{\alpha(i+1)}$ may be a not-yet-reachable data race PP),
and $\ppnext\pai'$ may include transitions of $\alpha$ beyond $\ppnext\pai$ itself.
%and for brevity, we will abbreviate the union of transitions as $\tak$.
Then, $\ppothers\pai'$ are the other threads' transitions from our target interleaving.
except they may be altered by the presence of some {\em other} data race in $\ppnext\pai'$ occurring after $\ppinstr\pai$.
Likewise, $\ppnext\pai'$ may be altered by some other data race in $\ppothers\pai'$.
The only certainty so far is that $\ppnext\pai'$ begins with $\ppinstr\pai$.

It would be straightforward to show that any other data races in $\ppnext\pai'$, which would be discovered in this interleaving,
could be reordered to after $\ppothers\pai'$,
eventually transforming $\ppnext\pai'$ to contain no conflicting memory accesses but $\ppinstr\pai$ itself.
Unfortunately, $\mathcal{J}$ is not necessarily reachable, as $\ppothers\pai'$ still includes any data-race PPs from $\ppothers\pai$,
which were ordered after $\pai$ in $\mathcal{I}$ and hence not covered by the inductive assumption.
Hence, we must perform the same ``coalescing'' for each thread in $\ppothers\pai'$ that we did for $\alpha$,
which we will abbreviate:
\[
	\mathcal{K} = \{..., (\tai \cup \ppnext\pai', \paj), \coalesce{\ppothers\pai'} \}
\]
For each other thread $\beta$, $\coalesce{\ppothers\pai'}$ will contain a single transition,
starting with the first instruction of $\beta$'s corresponding transition in $\ppothers\pai$,
and ending with the next sync API PP in $\beta$, which we'll call $\pbk$.
Again, the only certainly-executed instruction by $\alpha$ after $\tai$ is $\ppinstr\pai$.
It is even possible that neither $\ppinstr\pai$ nor any other instruction by thread $\alpha$ will conflict with any other thread,
until a different data race PP between two other threads is discovered,
as shown in Figure~\ref{fig:threethreads}.

It follows from the inductive hypothesis (which covers the prefix indicated by ``$...$''),
and from sync API PPs being reachable by definition,
that $\mathcal{K}$ is reachable.

\begin{figure}[t]
	TODO % TODO
	\caption{In this example, even though thread 1's data-race PP must occur first in a buggy interleaving, it cannot be reached until after Iterative Deepening first reaches the later-occurring data-race PP between threads 2 and 3.}
	\label{fig:threethreads}
\end{figure}

\newcommand\tbk{\ensuremath{t_{\beta{}k}}}
\newcommand\tbka{\ensuremath{t_{\beta{}k1}}}
\newcommand\tbkb{\ensuremath{t_{\beta{}k2}}}
\newcommand\tgl{\ensuremath{t_{\gamma{}l}}}

{\bf Finding the next reachable data-race PP.}
Now, we show by induction that a data race on $\ppinstr\pai$ will be reachable.
We assume that a state space $\mathcal{S}$ containing $\mathcal{K}$,
plus $n$ unique data race PPs among $\ppnext\pai'$ and $\ppothers\pai$,
will be reachable.
(In the base case, the number of new data-race PPs is 0, and $\mathcal{K}$'s reachability is justified above.)
We must show that in this state space, if the $\ppinstr\pai$ data race is not reachable,
a new unique data-race PP will instead be reachable.

By the definition of fully-relevant interleaving, $\mathcal{I}$ guarantees that some other thread $\omega$ can execute a data-racing instruction with $\ppinstr\pai$.
%
By the soundness of DPOR, if a program behaviour is possible by interleaving threads at the boundaries of
the given transitions, that interleaving will be tested.
By contrapositive,
because $\omega$'s data-racing instruction was not tested in $\mathcal{S}$, one or more necessary sites of interleaving must be in the middle of some transition, rather than at its boundaries.

However, we do not get a {\em single} such transition for free.
We have arrived at the crux of the proof:
to show that there cannot be multiple data-race PPs which must both be enabled before either data race can be identified.
Such a circular dependency seems intuitively impossible, but to actually find the ``first reachable'' PP is not straightforward.
We require that in $\mathcal{S}$, there exists a single transition $\tbk$ that can alone be split into $\{\tbka,\tbkb\}$,
and some other communicating transition $\tgl$ which conflicts with $\tbkb$.
%i.e., $\{\tbka, \tgl', \tbkb'\}$ exposes new program behaviour.

We show this by contradiction\footnote{
Regrettably, we could not devise a constructive algorithm for this part of the proof.
We leave actually identifying the first-reachable data-race PP to future work,
and trust that merely showing one must exist will satisfy the demands of modern verification.}:
%
Assume that for all $\tbk \in \mathcal{S}$,
and all possible points $\pbk'$ at which to split it into $\{\tbka,\tbkb\}$,
and all other non-MHB transitions $\tgl$,
$\tgl$ has no shared memory conflicts with $\tbkb$.
%
Let $\mathcal{S}' = \mathcal{S} \cup \pbk'$, i.e., the state space obtained by adding any such $\pbk'$ to $\mathcal{S}$'s PP-set.
By the soundness of DPOR, because any such $\tgl$ is independent with $\tbkb$,
\[
	\{..., \tbka, \tbkb, \tgl\}
	\equiv
	\{..., \tbka, \tgl', \tbkb'\}
\]
Hence, $\mathcal{S} \equiv \mathcal{S}'$. % This jump to full-state-space-equivalence seems a bit big, but the middle steps are all just more equivalence, obviously.
Then, the above assumption also applies to $\mathcal{S}'$,
which shows that for any {\em pair} of transitions such as $\tbk$, adding two new PPs cannot expose new program behaviour.
Hence, inductively, no set of new PPs of any size would expose new program behaviour not already exposed in $\mathcal{S}$.
However, the instruction by $\omega$ which conflicts with $\ppinstr\pai$ was not observed in $\mathcal{S}$,
so we have our contradiction.

Hence, if $\mathcal{S}$ does not expose the $\ppinstr\pai$ data race directly,
there must exist some transitions $\tbk = \{\tbka,\tbkb\}$ and $\tgl$ such that $\tgl$ shared-memory-conflicts with $\tbkb$ and could be interleaved immediately before it.
By the maximal state space assumption, all sync API PPs are already enabled, so the locksets of $\beta$ and $\gamma$ cannot overlap and there is no MHB relation.
%%% more verbose version
%so the locks held by $\beta$ cannot change during $\tbk$,
%and if $\tgl$ can be reordered to the middle of $\tbk$, the locksets of $\beta$ and $\gamma$ cannot overlap,
%and likewise there is no sync-enforced MHB relation.
Hence the memory conflict between $\tgl$ and $\tbkb$ will be identified as a data-race.
Finally, because $\tbk$ was not previously split by a PP, the data-race was not already discovered.

{\bf Reaching $\mathbf{p_{\alpha{}i}}$.}
Hence either a data-race with $\ppinstr\pai$ will be identified,
or an infinite/nonterminating sequence of other unique data-race PPs will be identified,
but by the halting assumption, this would constitute an infinite loop bug, which is sufficient.
(Alternatively, for any program with a finitely-sized instruction listing, the number of unique instruction pairs is finite.)
Hence by the ``next reachable data-race'' induction, $\pai$ is reachable.
Hence by the ``$\pai$ is reachable'' induction, all PPs in $\mathcal{I}$ are reachable.
\end{proof}

%%%%%%%%%%%%%%%%%%%%%%%%%%%%%%%%%%%%%%%%%%%%%%%%%%%%%%%%%%%%%%%%%%%%%%%%%%%%%%%%

\subsection{Conclusion}

\setcounter{theorem}{1}
\begin{theorem}[Convergence of Iterative Deepening]
	If a bug is exposed by an interleaving in the na\"{i}ve state space, Iterative Deepening will eventually test an interleaving which exposes the same bug.
\end{theorem}

\begin{proof}
By Lemma~\ref{lem:equivalent}, there must be some equivalent fully-relevant interleaving.
%the na\"{i}ve interleaving must be equivalent to some interleaving built of data-race PPs and sync API PPs.
By Lemma~\ref{lem:saturation}, Iterative Deepening will eventually discover and enable all necessary data-race PPs, with sync API PPs being enabled by the maximal state space assumption\footnote{
		In the paper we noted that when a bug is found, \quicksand~cancels all superset state spaces, anticipating that they would find the same bug.
		Because we are concerned with verifying the absence of bugs, we say this also suffices for convergence, but for brevity, we will leave this condition implicit.}.
Then DPOR will find the buggy interleaving within this state space.
\end{proof}

\section{Proof of malloc-recycle soundness}
\label{sec:proof}

We seek to prove that ignoring malloc-recycle data race candidates cannot cause DPOR + Iterative Deepening to miss a bug
that could be found by using the race as a PP predicate. Our soundness statement is as follows:

\begin{theorem}[Soundness of eliminating malloc-recycle races]
	If a malloc-recycle data race is not a false positive, DPOR will reorder threads such that either the same accesses will still race without fitting the malloc-recycle pattern, or a use-after-free bug will be reported immediately.
\end{theorem}

Though Figures~\ref{fig:recycle-bug} and \ref{fig:recycle-goal} show example programs, they do not capture all possible cases of how a true data race can fit the malloc-recycle pattern.
We proceed by establishing what must be true of any such program, then casing on the ambiguous possibilities, and showing that PPs will exist where we need them to reorder the threads.

For certain, there must be an access in one thread, followed by a free and malloc (we'll call them ``middle free'' and ``middle malloc''), each possibly from either thread, followed by an access from the other.
If the data race is not a false positive, then the second access must not change locations based on the middle malloc's return value.
WLOG, we say that thread 1 (T1) does the first access, called $a_1$, and thread 2 (T2) does the second, $a_2$.

\begin{lemma}
	If DPOR will reorder $a_2$ to before $a_1$, and the location of access $a_2$ doesn't change,
	then a non-malloc-recycle data race or a use-after-free bug will be identified.
	\label{lem:reorder}
\end{lemma}
\begin{proof}
By case on which threads the middle free and middle malloc came from.
\begin{itemize}
	\item T1 free, T2 malloc (as shown in Figure~\ref{fig:recycle-bug}). The malloc will go with $a_2$ to before the free, and because the allocation of concern has not been freed yet, will return a different value. Hence $a_1$ and $a_2$ will be in the same allocation; hence the race is not malloc-recycle anymore.
	\item T1 free, T1 malloc. Same as above, but the malloc does not move. The middle malloc will still recycle the memory, but $a_2$ now occurs before then, being in the same, older, allocation.
	\item T2 free, T2 malloc. Both the free and re-malloc will occur before either $a_1$ or $a_2$. The memory will be recycled and both accesses will appear to be in the later allocation.
	\item T2 free, T1 malloc. The free gets reordered earlier, the malloc stays put, and the accesses go in between. This will be a use-after-free bug.
\end{itemize}
If either the middle free or middle malloc came from a third thread, the case is the same as if it belonged to T1.
\end{proof}

The keen reader might ask here, what if T1 contains some extra spurious malloc calls (not related to $a_1$) that affect what T2's malloc returns after being reordered?
These could at best either cause {\tt x}'s memory to be recycled differently (not affecting the proof), or not at all (which simply causes immediate use-after-free).
%If such a malloc could affect $a_2$, the data race would be a false positive after all.
In general, extra spurious mallocs that could affect $a_1$ or $a_2$ could only convert the program back into a false-positive scenario;
and extra spurious synchronization events could only make it more easy to find PPs we need to trigger the reordering.
So we can safely assume the only relevant events are the ones we mention explicitly.

By our last assumption, there must also be an ``original malloc'' which allocated the block to begin with.
We must ask, which thread did the malloc which returned $a_1$'s address in the first place?
Our last assumption provides that the other thread must obtain that address through some communication mechanism (which we'll reason about later).

\subsection{T2 originally malloced x}

\begin{lemma}[Greedo]
	If T2 originally malloced the block containing {\tt x}, DPOR will reorder the threads to uncover a non-malloc-recycle race or a use-after-free bug.
	\label{lem:greedo}
\end{lemma}
\begin{proof}
Because T1 had the first access, there was a thread switch between the original malloc and $a_1$, as well as between $a_1$ and $a_2$. By Definition~\ref{def:transition}, each switch will be a PP.
By Definition~\ref{def:dpor}, DPOR will reorder $a_2$ to before $a_1$,
and because T1 is not involved in the logic determining $a_2$, the access's location stays the same.
Lemma~\ref{lem:reorder} finishes.
\end{proof}

This lemma also applies if a third thread was responsible for this malloc, as there would still be a thread switch in the same spot.

\subsection{T1 originally malloced x}

\begin{lemma}[Han]
	If T1 originally malloced the block containing {\tt x}, DPOR will reorder the threads to uncover a non-malloc-recycle race or a use-after-free bug.
	\label{lem:han}
\end{lemma}

\begin{proof}
We must guarantee there will be a PP during T1 before its $a_1$ access, but after whatever action it took to communicate the heap address to T2\footnote{
Note why we assert the publish action must come before $a_1$: otherwise, T2 couldn't be reordered to race $a_2$ with $a_1$ before T1 communicated the address, and it would be a false positive after all.}.

%\begin{figure}[t]
%	\small
%\begin{tabular}{rll}
%	& Thread 1 & Thread 2 \\
%	0 & \texttt{lock();} & \\
%	1 & \texttt{publish(x1);} & \\
%	3 & \texttt{\hilight{brickred}{x1->foo = ...;}} & \\
%	4 & \texttt{\hilight{olivegreen}{free}(x1);} \\
%	0 & \texttt{unlock();} & \\
%	0 & & \texttt{lock();} \\
%	2 & & \texttt{x2 = get\_published\_x();} \\
%	0 & & \texttt{unlock();} \\
%	5 & & \texttt{// x's memory recycled} \\
%	6 & & \texttt{y~=~\hilight{olivegreen}{malloc}(sizeof *y);} \\
%	7 & & \texttt{\hilight{brickred}{x2->foo = ...;}} \\
%\end{tabular}
%\caption{If the locksets are the same in T1 during publish and x1, you can't avoid the data race on publish+get.}
%\label{fig:recycle-bug}
%\end{figure}

If there was a synchronization event between the publish action and $a_1$,
then the maximal state space assumption provides the necessary PP, and we are done.
%We ask: was there a synchronization event (unlock or message-pass) between the publish action and $a_1$?
%If so, then by the maximal state space assumption, the event will be a PP, and we are done.
%If not,
Otherwise, T1's lockset will be the same during publish and during $a_1$ (and the MHB-ness cannot change).
For T1's publish to reach T2, they must access the same memory ({\em outside} of the block containing $a_1$; T2 doesn't have that yet),
which we'll call $p$.
Hence, $p$ must be a data race of its own.
%(whether by writing {\tt x}'s address directly into the shared memory, or by sharing a pointer to a data structure containing {\tt x} -- doesn't matter which)

\begin{figure}[t]
	\small
\begin{tabular}{rll}
	& {\bf Thread 1} & {\bf Thread 2} \\

	%1 & \texttt{p1 = \hilight{olivegreen}{malloc}(...);} & \\
	1 & \texttt{\hilight{brickred}{p1->ptr = x1;}} & \\
	2 & \texttt{publish(p1);} & \\
	3 & \texttt{\hilight{olivegreen}{free}(p1);} & \\
	4 & \texttt{\hilight{brickred}{x1->foo = ...;}} & \\
	5 & \texttt{\hilight{olivegreen}{free}(x1);} \\


	6 & & \texttt{p2 = get\_published\_p();} \\
	7 & & \texttt{\hilight{commentblue}{// p's memory recycled}} \\
	8 & & \texttt{q = \hilight{olivegreen}{malloc}(sizeof *q);} \\
	9 & & \texttt{\hilight{brickred}{x2 = p2->ptr;}} \\


	10 & & \texttt{\hilight{commentblue}{// x's memory recycled}} \\
	11 & & \texttt{y~=~\hilight{olivegreen}{malloc}(sizeof *y);} \\
	12 & & \texttt{\hilight{brickred}{x2->foo = ...;}} \\
\end{tabular}
\caption{If the accesses used to publish {\tt x}'s address are a data race, their PPs may also be eliminated under the malloc-recycle pattern. Induction on the pointer structure leading to {\tt x} handles this case.}
\label{fig:induction}
\end{figure}

Because $p$ may also be a malloc-recycle data race,
as shown in Figure~\ref{fig:induction},
we do not necessarily get the PP for free.
In this case we need to prove that DPOR will likewise reorder any intermediate malloc-recycle pattern to generate the PP we need\footnote{
In \quicksand, data race PPs are not used immediately, but rather generate new state spaces to explore in the future. Anyway, under the maximal state space assumption, we will get to it eventually.}.
We handle this with induction on T2's pointer chain leading to {\tt x}.

\newcommand\publish[1]{$p_{#1}$}
\begin{itemize}
	\item For the base case, the publish location \publish{0} is either in global memory, or shared directly using synchronization.
		Non-heap memory data races are not subject to the malloc-recycle pattern, so will always get a data-race PP,
		and use of synchronization always gets a PP in the maximal state space.
	\item For the inductive step, a pointer \publish{n} is published in some heap memory \publish{n-1}\texttt{->ptr},
		and we assume that however \publish{n-1} is shared to T2,
		there will be a PP there
		% where \publish{n-1} is shared
		sufficient to make the \publish{n-1}\texttt{->ptr} access not malloc-recycle after DPOR.
		%DPOR will test some interleaving in which T1's
		Hence a data race PP will be generated on the \publish{n-1}\texttt{->ptr} access,
		and by Definition~\ref{def:dpor} and Lemma~\ref{lem:reorder},
		DPOR will reorder T1's and T2's subsequent accesses to \publish{n} sufficiently to place a PP on them.
		% i.e., so that the p_n access no longer appears to be malloc-recycle.
\end{itemize}

Hence, even if the accesses by which T1 shares {\tt x} with T2 appear in a different malloc-recycle pattern,
a PP will be identified on the publish location $p$,
%Hence by Definition~\ref{def:dpor}
and DPOR will reorder T2's execution to just after the publish action.
% XXX: Landslide does not do this! It only puts a PP immediately *before* the access. That's incompatible with this, which needs you to put one both before and after.
As long as T2's execution occurs after the publish, it will receive the same value for its $a_2$, so the location of the data race does not change.
Lemma~\ref{lem:reorder} concludes.
%Hence for any type of publish action, there will be a PP between it and $a_1$, hence DPOR will reorder $a_2$ to before $a_1$ (without changing the address at which $a_2$ occurs), and Lemma~\ref{lem:reorder} finishes.
\end{proof}

%We will also say that T1 does the initial malloc which first returns $a_1$'s address; although certainly T2 or even a third thread could have malloced it, that would involve a thread switch to thread 1, and hence a PP by Definition~\ref{def:transition}.
%As we will see, the main challenge of the proof is showing that enough PPs will exist at appropriate points, so adding more PPs by considering other threads for the initial malloc can only make the proof easier.

\subsection{Conclusion}

\setcounter{theorem}{2}
\begin{theorem}[Soundness of eliminating malloc-recycle races]
	\label{thm:recycle}
	If a malloc-recycle data race is not a false positive, DPOR will reorder threads such that either the same accesses will still race without fitting the malloc-recycle pattern, or a use-after-free bug will be reported immediately.
\end{theorem}
\begin{proof}
	Between Lemmas \ref{lem:greedo} and \ref{lem:han}, all cases of possible program structure are covered.
\end{proof}

%%%%%%%%%%%%%%%%%%%%%%%%%%%%%%%%%%%%%%%%%%%%%%%%%%%%%%%%%%%%%%%%%%%%%%%%%%%%%%%%

\subsection{Heap overruns}
\label{sec:owned}

If we relax the ``sharing heap addresses'' assumption, there is another way to share the allocation's address without T1 and T2 communicating outside of $a_1$/$a_2$.
One thread can overrun a {\em different} heap block adjacent to the one containing $a_1$/$a_2$ (call them the ``neighbour block'' and ``real block'' respectively).
Figure~\ref{fig:overrun} shows an example.
Heap overrun bugs are quite serious \cite{eternal-war}, so we do not wish to exclude them from our proof.

In our evaluation, we used the full ``sharing heap addresses'' assumption to heuristically reduce state space size,
skipping reorderings which could only change the addresses allocated by $\mathsf{malloc}$.
This restricted our tests' scope to exclude such heap-overflow bugs.
We consider this justified because recent techniques \cite{sparc-ssm} can find such bugs quickly without the need for data-race PPs.
However, for users wishing to test for this class of bug and concurrency bugs simultaneously,
we show now how to strengthen the configuration of DPOR to cover the weakened assumption.

Note also that even if malloc-recycle candidates are {\em not} suppressed,
a DPOR which ignores malloc's internal metadata accesses would still be unsound with respect to these bugs.
We illustrate this in Figure~\ref{fig:overrun-notenough}:
even with PPs in arbitrary places, the two threads' transitions conflict only on $\mathsf{malloc}$'s internal metadata,
and hence DPOR would not attempt to reorder them.
Hence, our proofs so far show that suppressing malloc-recycle candidates is sound with respect to the classes of bugs which DPOR is already sound to.

\begin{figure}[t]
	\small
\begin{tabular}{rll}
	& {\bf Thread 1} & {\bf Thread 2} \\
	1 & & \texttt{z = malloc(42);} \\
	2 & & \texttt{\hilight{commentblue}{// TODO bounds check??}} \\
	3 & & \texttt{x2 = \&z[50];} \\
	4 & \texttt{x1 = malloc(...);} & \\
	5 & \texttt{\hilight{brickred}{x1->foo = ...;}} & \\
	6 & \texttt{\hilight{olivegreen}{free}(x1);} \\
	7 & & \texttt{\hilight{commentblue}{// x's memory recycled}} \\
	8 & & \texttt{y~=~\hilight{olivegreen}{malloc}(sizeof *y);} \\
	9 & & \texttt{\hilight{brickred}{x2->foo = ...;}} \\
\end{tabular}
\caption{Final possibility for how T2 can share T1's allocation address, and probably a security vulnerability to boot!}
\label{fig:overrun}
\end{figure}


% For shown it is in complexity class, not possible to complete,
To soundly suppress heap-overrun malloc-recycle candidates,
%we must consider the cases where T1 mallocs the real block and T2 overflows a neighbour, and vice versa.
we must strengthen our configuration of DPOR as follows:
accesses from $\mathsf{malloc}$'s internal implementation are not ignored when computing shared memory conflicts,
and the synchronization API PP predicates are extended to include the start and end of each $\mathsf{malloc}$ and $\mathsf{free}$ call.
%\footnote{
%	As a practical matter, in a large codebase where parts of it are trusted to avoid heap overflows, while others are not,
%	a user could use {\sc Landslide}'s {\tt within\_function} command to restrict such PPs, as we describe in the main paper.
%}.

\begin{lemma}
	If T1 and T2 each malloced neighbouring blocks, and collided based on pointer arithmetic involving no shared memory accesses,
	%did not communicate via any publish action $p$,
	%and the DPOR configuration is strengthened as above,
	DPOR will reorder the threads to uncover a non-malloc-recycle data race.
	\label{lem:leia} % "there is another"
\end{lemma}

\begin{proof}
WLOG, let T1's access in the original malloc-recycle race occur first.
We require that our strengthened DPOR will reorder T2's racing access to before T1's, such that both still occur on the same address.
There will be a PP in T1's execution between its $\mathsf{malloc}$ call and the subsequent racing access.
If there are no other memory conflicts between T1 and T2, then by the soundness of DPOR, this PP suffices to reorder without changing the address.
Otherwise, let $p$ be the latest conflicting access by T1 before its access $a_1$.
By the same inductive reasoning as we used in Lemma~\ref{lem:han}, Iterative Deepening will add a data-race PP on $p$.
As T1 has no further conflicts between $p$ and $a_1$, T2's $a_2$ will be reordered between them without changing the address.
Lemma~\ref{lem:reorder} finishes.
\end{proof}

Theorem~\ref{thm:recycle} is modified to simply include this lemma in addition to Lemmas~\ref{lem:greedo} and \ref{lem:han}.

%This spells QED so we are done.
\begin{figure}[t]
	\small
\begin{tabular}{rll}
	& {\bf Thread 1} & {\bf Thread 2} \\
	1 & & \texttt{z = malloc(42);} \\
	2 & & \texttt{\hilight{commentblue}{// TODO bounds check??}} \\
	3 & & \texttt{x2 = \&z[50];} \\
	4 & & \texttt{y~=~\hilight{olivegreen}{malloc}(sizeof *y);} \\
	5 & & \texttt{\hilight{brickred}{x2->foo = ...;}} \\
	6 & \texttt{x1 = malloc(...);} & \\
	7 & \texttt{\hilight{brickred}{x1->foo = ...;}} & \\
	8 & \texttt{\hilight{olivegreen}{free}(x1);} \\
\end{tabular}
\caption{Without a PP between lines 4 and 5 of Figure~\ref{fig:overrun}, this is the only alternate interleaving DPOR would explore. The mallocs have been reordered and may no longer collide, which wrongly appears to be a false positive.}
\label{fig:overrun-notenough}
\end{figure}


\begin{figure}[t]
	\small
\begin{tabular}{rll}
	& {\bf Thread 1} & {\bf Thread 2} \\
	1 & & \texttt{z = malloc(42);} \\
	2 & & \texttt{\hilight{commentblue}{// TODO bounds check??}} \\
	3 & & \texttt{x2 = \&z[50];} \\
	4 & \texttt{x1 = malloc(...);} & \\
	5 & & \texttt{y~=~\hilight{olivegreen}{malloc}(sizeof *y);} \\
	6 & & \texttt{\hilight{brickred}{x2->foo = ...;}} \\
	7 & \texttt{\hilight{brickred}{x1->foo = ...;}} & \\
	8 & \texttt{\hilight{olivegreen}{free}(x1);} \\
\end{tabular}
	\caption{Goal interleaving of Figure~\ref{fig:overrun}. To ensure collision, the sequence of malloc calls producing $a_1$ and $a_2$ must not be disrupted compared to the original interleaving.}
	%, compared to the trace which originally exposed the malloc-recycle race.}
\label{fig:overrun-goal}
\end{figure}


% TODO CAMREADY
% We suspect that DPOR + pure-HB will always eventually find all data races uncovered by MHB.
% for the purpose of iterative deepening, MHB is
% better because it will find the data races much sooner, rather than requiring you
% to explore the maximal state space before the race can be uncovered

\section{Acknowledgements}

{\em removed for double blind review}
%Thanks to Michael J. ``Sully'' Sullivan and Carlo Angiuli, who double-checked the proof and helped with the flow of explanation.

\bibliographystyle{abbrvnat}
\bibliography{../citations}{}
\end{document}
