\section{Introduction}

% blah blah trite opening sentence
%As parallelism becomes ever more important for achieving high performance in modern-day programs,
%so too do advanced concurrency testing techniques become important for verifying the correctness of those programs.
Concurrency bugs are notoriously hard to find and reproduce because they only appear in specific thread interleavings, which arise at random during normal program execution.
{\em Stateless model checking} \cite{verisoft} offers a method for finding such bugs,
or verifying their absence,
%by systematically executing a program along as many distinct interleavings as possible,
by forcing a program to execute each distinct interleaving,
capturing and controlling this nondeterminism using a finite state space.
Unfortunately, these state spaces explode exponentially in the size of the input program.
Reduction techniques such as Dynamic Partial Order Reduction \cite{dpor} and Maximal Causality Reduction \cite{mcr} expand the limits of feasible test completion,
and search ordering strategies such as Iterative Context Bounding \cite{chess-icb} allow bugs to be found sooner in a given exploration should they exist.

% Can I even make a claim this broad to begin with?
However, all stateless model checkers to date are bound by a fixed set of {\em preemption points}: code locations that define the granularity at which threads interleave.
Choosing these preemption points is a tradeoff between schedule coverage and feasibility of completion.
For example, \textsc{CHESS} \cite{chess} preempts only on synchronization operations and library calls, which can miss lock-free shared memory races.
%
On the other hand, SPIN \cite{spin}
is able to preempt threads around any shared memory access. Such fine granularity would automatically check if each data race is a real bug, but makes full state space completion intractable for even modestly-sized tests.
%
This work shows how to avoid making that tradeoff decision in advance.

We present \quicksand,
a model checking framework for deciding at runtime which preemption points to test,
according to which resulting state spaces are most likely to fit a prescribed CPU budget.
It uses data-race analysis to dynamically find new preemption points which expose bugs not reachable by preempting on API calls alone.
When prior work would time out on large tests by trying several preemption points simultaneously,
\quicksand~tests smaller state spaces based on subsets of those points, which often finds the same bugs sooner.
On the other hand, when the CPU budget is large enough to fully test all data-race preemption points,
we prove that this constitutes a total verification of all possible thread schedules.
Short of deciding in advance to preempt on every single instruction \cite{spin}, this was not previously possible without data-race preemption points.

We evaluate \quicksand~by testing \numstudence~student thread libraries and kernels from the undergraduate OS classes at Carnegie Mellon, Berkeley, and the University of Chicago.
We show many advantages over both conventional stateless model checking and single-pass data-race analysis.
We find that many of the bugs found by the conventional approach can be found faster using subsets of preemption points,
and that data-race preemption points quickly expose many new bugs that prior model checkers could not find at all.

Our contributions are as follows:
\begin{enumerate}
	\item {\em Iterative Deepening}, a new technique for combining data-race analysis with stateless model checking, and \quicksand, an open-source implementation of the technique;
	\item A proof of convergence, showing that should it be possible in the given CPU budget,
		fully testing every discovered data-race preemption point is equivalent to testing all possible thread schedules;
	\item A new tactic for eliminating one class of false-positive data races,
		which cannot soundly be used in a single-pass analysis,
		but which we prove correct when used with Iterative Deepening;
		%, unsound in single-pass analysis but which we prove sound when used with Iterative Deepening;
	\item A large evaluation in which \quicksand~compares favorably to stand-alone data-race detection and stateless model checking approaches, finding new bugs that would be missed by either alone.
\end{enumerate}

The remainder of the paper is organized as follows.
\sect{\ref{sec:overview}} reviews the background on stateless model checking and data-race analysis,
\sect{\ref{sec:design}} and \sect{\ref{sec:implementation}} discuss our design and implementation of Iterative Deepening,
\sect{\ref{sec:soundness}} presents our proofs of convergence and of the soundness of our new false-positive data-race tactic,
\sect{\ref{sec:eval}} presents our evaluation,
\sect{\ref{sec:future}} discusses limitations and future work,
\sect{\ref{sec:related}} surveys the related work,
and \sect{\ref{sec:conclusion}} concludes.
