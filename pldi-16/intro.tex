\section{Introduction}

% blah blah trite opening sentence
%As parallelism becomes ever more important for achieving high performance in modern-day programs,
%so too do advanced concurrency testing techniques become important for verifying the correctness of those programs.
Concurrency bugs are notoriously hard to find and reproduce because they only appear in specific thread interleavings, which arise at random during normal program execution.
{\em Stateless model checking} \cite{verisoft} offers a method for finding such bugs,
or verifying their absence,
%by systematically executing a program along as many distinct interleavings as possible,
by forcing a program to execute each distinct interleaving,
capturing and controlling this nondeterminism using a finite state space.
Unfortunately, these state spaces explode exponentially in the size of the input program.
Reduction techniques such as Dynamic Partial Order Reduction \cite{dpor} and Maximal Causality Reduction \cite{mcr} expand the limits of feasible test completion,
and search ordering strategies such as Iterative Context Bounding \cite{chess-icb} allow bugs to be found sooner in a given exploration should they exist.

% Can I even make a claim this broad to begin with?
However, all stateless model checkers to date are bound by a fixed set of {\em preemption points}: code locations that define the granularity at which threads interleave.
% TODO: You repeat the following 1.5 sentence in the related work section; maybe you can eliminate this redundancy to save space?
For example, \textsc{CHESS} \cite{chess} preempts only on synchronization operations and library calls, which can miss lock-free shared memory races.
It provides an additional data-race analysis to report any violations of this model;
however, data-race analyses are prone to report false positives
%and benign races
which require annotations or imprecise heuristics to suppress \cite{racerx,tsan,datacollider}.
%
On the other hand, SPIN \cite{spin}
is able to preempt threads around any shared memory access. Such fine granularity would automatically check if each data race is a real bug, but makes full state space completion intractable for even modestly-sized tests.
%
Configuring a model checker is a tradeoff between schedule coverage and feasibility of completion.
This work shows how to avoid making that tradeoff decision in advance.

% TODO: ttuttle says this is too much of a jump -- that exactly what "subsets" means is not well defined (i.e, lock vs unlock, while NOT addressing the within/without_function problem)

We present \quicksand, a framework for {\em Iterative Deepening} of preemption points during stateless model checking.
Named after the analogous technique in chess AI \cite{iterative-deepening-chess-ai}, our approach likewise makes progressively deeper searches of the state space until a given CPU budget is exhausted.
Rather than attempting to search a single state space with every available preemption point enabled (e.g., preempting on every pthread API call),
\quicksand~tests many different state spaces corresponding to subsets of those points, managing a model checker instance to explore each one.
It estimates the size of each state space to decide when long-running instances should be suspended, and dynamically generates new state spaces based on data race analysis.

We evaluated \quicksand~by testing \numstudence~student thread libraries and kernels from the undergraduate OS classes at Carnegie Mellon, Berkeley, and the University of Chicago.
We show that \quicksand~finds more bugs than the conventional stateless model checking approach given the same CPU budget,
and furthermore that adding data-race preemption points quickly exposes bugs missed by even the ``maximal'' state space of the conventional approach.

This paper's contributions are as follows:
\begin{enumerate}
	\item Iterative Deepening, a new technique for combining data-race analysis with stateless model checking, and \quicksand, an open source implementation of the technique;
	\item A new tactic for eliminating one class of false-positive data races,
		which cannot soundly be used in a single-pass analysis,
		but which we prove correct when used with Iterative Deepening;
		%, unsound in single-pass analysis but which we prove sound when used with Iterative Deepening;
	%\item Techniques for detecting and flattening cyclic state spaces resulting from ad-hoc while-loop synchronization % TODO: is there actually room for this in the paper?
	\item A large evaluation in which \quicksand~compares favorably to stand-alone data-race detection and stateless model checking approaches.
\end{enumerate}

The remainder of the paper is organized as follows.
\sect{\ref{sec:design}} discusses the background and design of Iterative Deepening,
\sect{\ref{sec:implementation}} explains \quicksand's approach to implementing it,
\sect{\ref{sec:eval}} presents our evaluation,
\sect{\ref{sec:future}} discusses limitations and future work,
\sect{\ref{sec:related}} surveys the related work,
and \sect{\ref{sec:conclusion}} concludes.
