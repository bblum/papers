\section{Implementation}
\label{sec:implementation}

\subsection{Landslide}
\label{sec:landslide}

We chose \landslide~\cite{landslide} as our stateless model checker due to its ability to trace program execution at the granularity of individual instructions and memory accesses, which dynamic data-race detection requires.
\landslide~implements DPOR \cite{dpor},
%\cite{dpor},
state space estimation \cite{estimation}, and a hybrid lockset/happens-before data-race analysis \cite{hybriddatarace}.
It avoids state space cycles (e.g. ad-hoc synchronization with {\tt yield} or even {\tt xchg} loops) with a heuristic similar to Fair-Bounded Search \cite{bpor}.
% this line can be cut if space is needed
%It can test both user-level and kernel-level code, although is limited to timer-driven nondeterminism.
% joshua wants "segfault" to be "memory access error (i.e., segmentation fault, or bus error)"
Its bug-detection metrics include assertion failure, deadlock, segfault, heap checking (like Valgrind~\cite{valgrind}), and a heuristic infinite loop/livelock check.

{\bf Restricting PPs with stack trace predicates.}
Most MC tools preempt indiscriminately on any sync API call, regardless of the call-site.
However, when testing a particular module in a large codebase,
the user is likely uninterested in PPs arising from other modules.
\landslide~provides the {\tt within\_function} configuration command for a user to identify which call-sites matter most.
Before inserting a PP, \landslide~requires at least one argument to {\tt within\_function} to appear in the current thread's stack trace.
%The {\tt without\_function} directive works similarly, but as a blacklist.
The {\tt without\_function} directive is the dual of {\tt within\_function}, indicating a blacklist.
Multiple invocations can be used; later ones take precedence.
%\cite{landslide} provides further detail on this feature.

{\bf Data races in lock implementations.}
Data race tools in prior work \cite{tsan,portend} recognize the implementations of sync primitives to avoid spuriously flagging memory accesses resulting from the lock implementation itself.
The assumption that the locks are already correct enables productive data-race analysis on the rest of the codebase.
Otherwise, with testing limited to one execution,
%even if one wishes to test for lock bugs,
data-race analysis would flag every access pair in the lock implementation. %, requiring human attention to verify.
However, Iterative Deepening removes the need for human attention to verify them. %can automatically verify a large quantity of data-race candidates as benign.
Hence, we extended \landslide~with a custom option to change the lock-set tracking to include accesses from {\tt mutex\_lock()} and {\tt mutex\_unlock()} in the data race analysis.
(Accesses from other sync functions, such as {\tt cond\_wait()}, would either be included already, or be protected by an internal mutex.)

\subsection{Quicksand}

\quicksand~is an independent program that wraps the execution of several \landslide~instances.
The implementation is roughly 3000 lines of C.
%It uses a thread pool to schedule the available state spaces,
%sorting such jobs according to their status among a running queue, pending queue, and suspended queue.
%Jobs are further prioritized by number of PPs, ETA, and whether they include data-race PPs.
The interface to \landslide~ has two parts. %, which any similar MC could implement, has two parts.
First, when starting each job, \quicksand~creates a configuration file declaring which PPs to use,
% can lose this line due to space
among other options such as mutex-testing mode,
passed as an argument to \landslide.
Then, a dedicated \quicksand~thread communicates with the \landslide~process via message-passing. %on a FIFO pipe.
\landslide~messages \quicksand~after testing each new interleaving to report updated progress and ETA,
and whenever it finds a new data-race candidate or bug.
\quicksand~in turn replies whether to suspend/resume due to too high ETA, or quit due to timeout.
We suspend jobs simply by making \landslide~wait on a message-passing call.
Should \quicksand~later resume a suspended job, we send a message to continue,
resuming the \landslide~instance where it left off.
%otherwise, we resume it only after time runs out, causing it to exit.

